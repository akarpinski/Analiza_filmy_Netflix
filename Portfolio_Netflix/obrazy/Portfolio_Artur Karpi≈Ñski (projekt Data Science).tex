\documentclass[11pt]{article}

    \usepackage[breakable]{tcolorbox}
    \usepackage{parskip} % Stop auto-indenting (to mimic markdown behaviour)
    
    \usepackage{iftex}
    \ifPDFTeX
    	\usepackage[T1]{fontenc}
    	\usepackage{mathpazo}
    \else
    	\usepackage{fontspec}
    \fi

    % Basic figure setup, for now with no caption control since it's done
    % automatically by Pandoc (which extracts ![](path) syntax from Markdown).
    \usepackage{graphicx}
    % Maintain compatibility with old templates. Remove in nbconvert 6.0
    \let\Oldincludegraphics\includegraphics
    % Ensure that by default, figures have no caption (until we provide a
    % proper Figure object with a Caption API and a way to capture that
    % in the conversion process - todo).
    \usepackage{caption}
    \DeclareCaptionFormat{nocaption}{}
    \captionsetup{format=nocaption,aboveskip=0pt,belowskip=0pt}

    \usepackage{float}
    \floatplacement{figure}{H} % forces figures to be placed at the correct location
    \usepackage{xcolor} % Allow colors to be defined
    \usepackage{enumerate} % Needed for markdown enumerations to work
    \usepackage{geometry} % Used to adjust the document margins
    \usepackage{amsmath} % Equations
    \usepackage{amssymb} % Equations
    \usepackage{textcomp} % defines textquotesingle
    % Hack from http://tex.stackexchange.com/a/47451/13684:
    \AtBeginDocument{%
        \def\PYZsq{\textquotesingle}% Upright quotes in Pygmentized code
    }
    \usepackage{upquote} % Upright quotes for verbatim code
    \usepackage{eurosym} % defines \euro
    \usepackage[mathletters]{ucs} % Extended unicode (utf-8) support
    \usepackage{fancyvrb} % verbatim replacement that allows latex
    \usepackage{grffile} % extends the file name processing of package graphics 
                         % to support a larger range
    \makeatletter % fix for old versions of grffile with XeLaTeX
    \@ifpackagelater{grffile}{2019/11/01}
    {
      % Do nothing on new versions
    }
    {
      \def\Gread@@xetex#1{%
        \IfFileExists{"\Gin@base".bb}%
        {\Gread@eps{\Gin@base.bb}}%
        {\Gread@@xetex@aux#1}%
      }
    }
    \makeatother
    \usepackage[Export]{adjustbox} % Used to constrain images to a maximum size
    \adjustboxset{max size={0.9\linewidth}{0.9\paperheight}}

    % The hyperref package gives us a pdf with properly built
    % internal navigation ('pdf bookmarks' for the table of contents,
    % internal cross-reference links, web links for URLs, etc.)
    \usepackage{hyperref}
    % The default LaTeX title has an obnoxious amount of whitespace. By default,
    % titling removes some of it. It also provides customization options.
    \usepackage{titling}
    \usepackage{longtable} % longtable support required by pandoc >1.10
    \usepackage{booktabs}  % table support for pandoc > 1.12.2
    \usepackage[inline]{enumitem} % IRkernel/repr support (it uses the enumerate* environment)
    \usepackage[normalem]{ulem} % ulem is needed to support strikethroughs (\sout)
                                % normalem makes italics be italics, not underlines
    \usepackage{mathrsfs}
    

    
    % Colors for the hyperref package
    \definecolor{urlcolor}{rgb}{0,.145,.698}
    \definecolor{linkcolor}{rgb}{.71,0.21,0.01}
    \definecolor{citecolor}{rgb}{.12,.54,.11}

    % ANSI colors
    \definecolor{ansi-black}{HTML}{3E424D}
    \definecolor{ansi-black-intense}{HTML}{282C36}
    \definecolor{ansi-red}{HTML}{E75C58}
    \definecolor{ansi-red-intense}{HTML}{B22B31}
    \definecolor{ansi-green}{HTML}{00A250}
    \definecolor{ansi-green-intense}{HTML}{007427}
    \definecolor{ansi-yellow}{HTML}{DDB62B}
    \definecolor{ansi-yellow-intense}{HTML}{B27D12}
    \definecolor{ansi-blue}{HTML}{208FFB}
    \definecolor{ansi-blue-intense}{HTML}{0065CA}
    \definecolor{ansi-magenta}{HTML}{D160C4}
    \definecolor{ansi-magenta-intense}{HTML}{A03196}
    \definecolor{ansi-cyan}{HTML}{60C6C8}
    \definecolor{ansi-cyan-intense}{HTML}{258F8F}
    \definecolor{ansi-white}{HTML}{C5C1B4}
    \definecolor{ansi-white-intense}{HTML}{A1A6B2}
    \definecolor{ansi-default-inverse-fg}{HTML}{FFFFFF}
    \definecolor{ansi-default-inverse-bg}{HTML}{000000}

    % common color for the border for error outputs.
    \definecolor{outerrorbackground}{HTML}{FFDFDF}

    % commands and environments needed by pandoc snippets
    % extracted from the output of `pandoc -s`
    \providecommand{\tightlist}{%
      \setlength{\itemsep}{0pt}\setlength{\parskip}{0pt}}
    \DefineVerbatimEnvironment{Highlighting}{Verbatim}{commandchars=\\\{\}}
    % Add ',fontsize=\small' for more characters per line
    \newenvironment{Shaded}{}{}
    \newcommand{\KeywordTok}[1]{\textcolor[rgb]{0.00,0.44,0.13}{\textbf{{#1}}}}
    \newcommand{\DataTypeTok}[1]{\textcolor[rgb]{0.56,0.13,0.00}{{#1}}}
    \newcommand{\DecValTok}[1]{\textcolor[rgb]{0.25,0.63,0.44}{{#1}}}
    \newcommand{\BaseNTok}[1]{\textcolor[rgb]{0.25,0.63,0.44}{{#1}}}
    \newcommand{\FloatTok}[1]{\textcolor[rgb]{0.25,0.63,0.44}{{#1}}}
    \newcommand{\CharTok}[1]{\textcolor[rgb]{0.25,0.44,0.63}{{#1}}}
    \newcommand{\StringTok}[1]{\textcolor[rgb]{0.25,0.44,0.63}{{#1}}}
    \newcommand{\CommentTok}[1]{\textcolor[rgb]{0.38,0.63,0.69}{\textit{{#1}}}}
    \newcommand{\OtherTok}[1]{\textcolor[rgb]{0.00,0.44,0.13}{{#1}}}
    \newcommand{\AlertTok}[1]{\textcolor[rgb]{1.00,0.00,0.00}{\textbf{{#1}}}}
    \newcommand{\FunctionTok}[1]{\textcolor[rgb]{0.02,0.16,0.49}{{#1}}}
    \newcommand{\RegionMarkerTok}[1]{{#1}}
    \newcommand{\ErrorTok}[1]{\textcolor[rgb]{1.00,0.00,0.00}{\textbf{{#1}}}}
    \newcommand{\NormalTok}[1]{{#1}}
    
    % Additional commands for more recent versions of Pandoc
    \newcommand{\ConstantTok}[1]{\textcolor[rgb]{0.53,0.00,0.00}{{#1}}}
    \newcommand{\SpecialCharTok}[1]{\textcolor[rgb]{0.25,0.44,0.63}{{#1}}}
    \newcommand{\VerbatimStringTok}[1]{\textcolor[rgb]{0.25,0.44,0.63}{{#1}}}
    \newcommand{\SpecialStringTok}[1]{\textcolor[rgb]{0.73,0.40,0.53}{{#1}}}
    \newcommand{\ImportTok}[1]{{#1}}
    \newcommand{\DocumentationTok}[1]{\textcolor[rgb]{0.73,0.13,0.13}{\textit{{#1}}}}
    \newcommand{\AnnotationTok}[1]{\textcolor[rgb]{0.38,0.63,0.69}{\textbf{\textit{{#1}}}}}
    \newcommand{\CommentVarTok}[1]{\textcolor[rgb]{0.38,0.63,0.69}{\textbf{\textit{{#1}}}}}
    \newcommand{\VariableTok}[1]{\textcolor[rgb]{0.10,0.09,0.49}{{#1}}}
    \newcommand{\ControlFlowTok}[1]{\textcolor[rgb]{0.00,0.44,0.13}{\textbf{{#1}}}}
    \newcommand{\OperatorTok}[1]{\textcolor[rgb]{0.40,0.40,0.40}{{#1}}}
    \newcommand{\BuiltInTok}[1]{{#1}}
    \newcommand{\ExtensionTok}[1]{{#1}}
    \newcommand{\PreprocessorTok}[1]{\textcolor[rgb]{0.74,0.48,0.00}{{#1}}}
    \newcommand{\AttributeTok}[1]{\textcolor[rgb]{0.49,0.56,0.16}{{#1}}}
    \newcommand{\InformationTok}[1]{\textcolor[rgb]{0.38,0.63,0.69}{\textbf{\textit{{#1}}}}}
    \newcommand{\WarningTok}[1]{\textcolor[rgb]{0.38,0.63,0.69}{\textbf{\textit{{#1}}}}}
    
    
    % Define a nice break command that doesn't care if a line doesn't already
    % exist.
    \def\br{\hspace*{\fill} \\* }
    % Math Jax compatibility definitions
    \def\gt{>}
    \def\lt{<}
    \let\Oldtex\TeX
    \let\Oldlatex\LaTeX
    \renewcommand{\TeX}{\textrm{\Oldtex}}
    \renewcommand{\LaTeX}{\textrm{\Oldlatex}}
    % Document parameters
    % Document title
    \title{Portfolio\_Artur Karpiński (projekt Data Science)}
    
    
    
    
    
% Pygments definitions
\makeatletter
\def\PY@reset{\let\PY@it=\relax \let\PY@bf=\relax%
    \let\PY@ul=\relax \let\PY@tc=\relax%
    \let\PY@bc=\relax \let\PY@ff=\relax}
\def\PY@tok#1{\csname PY@tok@#1\endcsname}
\def\PY@toks#1+{\ifx\relax#1\empty\else%
    \PY@tok{#1}\expandafter\PY@toks\fi}
\def\PY@do#1{\PY@bc{\PY@tc{\PY@ul{%
    \PY@it{\PY@bf{\PY@ff{#1}}}}}}}
\def\PY#1#2{\PY@reset\PY@toks#1+\relax+\PY@do{#2}}

\expandafter\def\csname PY@tok@w\endcsname{\def\PY@tc##1{\textcolor[rgb]{0.73,0.73,0.73}{##1}}}
\expandafter\def\csname PY@tok@c\endcsname{\let\PY@it=\textit\def\PY@tc##1{\textcolor[rgb]{0.25,0.50,0.50}{##1}}}
\expandafter\def\csname PY@tok@cp\endcsname{\def\PY@tc##1{\textcolor[rgb]{0.74,0.48,0.00}{##1}}}
\expandafter\def\csname PY@tok@k\endcsname{\let\PY@bf=\textbf\def\PY@tc##1{\textcolor[rgb]{0.00,0.50,0.00}{##1}}}
\expandafter\def\csname PY@tok@kp\endcsname{\def\PY@tc##1{\textcolor[rgb]{0.00,0.50,0.00}{##1}}}
\expandafter\def\csname PY@tok@kt\endcsname{\def\PY@tc##1{\textcolor[rgb]{0.69,0.00,0.25}{##1}}}
\expandafter\def\csname PY@tok@o\endcsname{\def\PY@tc##1{\textcolor[rgb]{0.40,0.40,0.40}{##1}}}
\expandafter\def\csname PY@tok@ow\endcsname{\let\PY@bf=\textbf\def\PY@tc##1{\textcolor[rgb]{0.67,0.13,1.00}{##1}}}
\expandafter\def\csname PY@tok@nb\endcsname{\def\PY@tc##1{\textcolor[rgb]{0.00,0.50,0.00}{##1}}}
\expandafter\def\csname PY@tok@nf\endcsname{\def\PY@tc##1{\textcolor[rgb]{0.00,0.00,1.00}{##1}}}
\expandafter\def\csname PY@tok@nc\endcsname{\let\PY@bf=\textbf\def\PY@tc##1{\textcolor[rgb]{0.00,0.00,1.00}{##1}}}
\expandafter\def\csname PY@tok@nn\endcsname{\let\PY@bf=\textbf\def\PY@tc##1{\textcolor[rgb]{0.00,0.00,1.00}{##1}}}
\expandafter\def\csname PY@tok@ne\endcsname{\let\PY@bf=\textbf\def\PY@tc##1{\textcolor[rgb]{0.82,0.25,0.23}{##1}}}
\expandafter\def\csname PY@tok@nv\endcsname{\def\PY@tc##1{\textcolor[rgb]{0.10,0.09,0.49}{##1}}}
\expandafter\def\csname PY@tok@no\endcsname{\def\PY@tc##1{\textcolor[rgb]{0.53,0.00,0.00}{##1}}}
\expandafter\def\csname PY@tok@nl\endcsname{\def\PY@tc##1{\textcolor[rgb]{0.63,0.63,0.00}{##1}}}
\expandafter\def\csname PY@tok@ni\endcsname{\let\PY@bf=\textbf\def\PY@tc##1{\textcolor[rgb]{0.60,0.60,0.60}{##1}}}
\expandafter\def\csname PY@tok@na\endcsname{\def\PY@tc##1{\textcolor[rgb]{0.49,0.56,0.16}{##1}}}
\expandafter\def\csname PY@tok@nt\endcsname{\let\PY@bf=\textbf\def\PY@tc##1{\textcolor[rgb]{0.00,0.50,0.00}{##1}}}
\expandafter\def\csname PY@tok@nd\endcsname{\def\PY@tc##1{\textcolor[rgb]{0.67,0.13,1.00}{##1}}}
\expandafter\def\csname PY@tok@s\endcsname{\def\PY@tc##1{\textcolor[rgb]{0.73,0.13,0.13}{##1}}}
\expandafter\def\csname PY@tok@sd\endcsname{\let\PY@it=\textit\def\PY@tc##1{\textcolor[rgb]{0.73,0.13,0.13}{##1}}}
\expandafter\def\csname PY@tok@si\endcsname{\let\PY@bf=\textbf\def\PY@tc##1{\textcolor[rgb]{0.73,0.40,0.53}{##1}}}
\expandafter\def\csname PY@tok@se\endcsname{\let\PY@bf=\textbf\def\PY@tc##1{\textcolor[rgb]{0.73,0.40,0.13}{##1}}}
\expandafter\def\csname PY@tok@sr\endcsname{\def\PY@tc##1{\textcolor[rgb]{0.73,0.40,0.53}{##1}}}
\expandafter\def\csname PY@tok@ss\endcsname{\def\PY@tc##1{\textcolor[rgb]{0.10,0.09,0.49}{##1}}}
\expandafter\def\csname PY@tok@sx\endcsname{\def\PY@tc##1{\textcolor[rgb]{0.00,0.50,0.00}{##1}}}
\expandafter\def\csname PY@tok@m\endcsname{\def\PY@tc##1{\textcolor[rgb]{0.40,0.40,0.40}{##1}}}
\expandafter\def\csname PY@tok@gh\endcsname{\let\PY@bf=\textbf\def\PY@tc##1{\textcolor[rgb]{0.00,0.00,0.50}{##1}}}
\expandafter\def\csname PY@tok@gu\endcsname{\let\PY@bf=\textbf\def\PY@tc##1{\textcolor[rgb]{0.50,0.00,0.50}{##1}}}
\expandafter\def\csname PY@tok@gd\endcsname{\def\PY@tc##1{\textcolor[rgb]{0.63,0.00,0.00}{##1}}}
\expandafter\def\csname PY@tok@gi\endcsname{\def\PY@tc##1{\textcolor[rgb]{0.00,0.63,0.00}{##1}}}
\expandafter\def\csname PY@tok@gr\endcsname{\def\PY@tc##1{\textcolor[rgb]{1.00,0.00,0.00}{##1}}}
\expandafter\def\csname PY@tok@ge\endcsname{\let\PY@it=\textit}
\expandafter\def\csname PY@tok@gs\endcsname{\let\PY@bf=\textbf}
\expandafter\def\csname PY@tok@gp\endcsname{\let\PY@bf=\textbf\def\PY@tc##1{\textcolor[rgb]{0.00,0.00,0.50}{##1}}}
\expandafter\def\csname PY@tok@go\endcsname{\def\PY@tc##1{\textcolor[rgb]{0.53,0.53,0.53}{##1}}}
\expandafter\def\csname PY@tok@gt\endcsname{\def\PY@tc##1{\textcolor[rgb]{0.00,0.27,0.87}{##1}}}
\expandafter\def\csname PY@tok@err\endcsname{\def\PY@bc##1{\setlength{\fboxsep}{0pt}\fcolorbox[rgb]{1.00,0.00,0.00}{1,1,1}{\strut ##1}}}
\expandafter\def\csname PY@tok@kc\endcsname{\let\PY@bf=\textbf\def\PY@tc##1{\textcolor[rgb]{0.00,0.50,0.00}{##1}}}
\expandafter\def\csname PY@tok@kd\endcsname{\let\PY@bf=\textbf\def\PY@tc##1{\textcolor[rgb]{0.00,0.50,0.00}{##1}}}
\expandafter\def\csname PY@tok@kn\endcsname{\let\PY@bf=\textbf\def\PY@tc##1{\textcolor[rgb]{0.00,0.50,0.00}{##1}}}
\expandafter\def\csname PY@tok@kr\endcsname{\let\PY@bf=\textbf\def\PY@tc##1{\textcolor[rgb]{0.00,0.50,0.00}{##1}}}
\expandafter\def\csname PY@tok@bp\endcsname{\def\PY@tc##1{\textcolor[rgb]{0.00,0.50,0.00}{##1}}}
\expandafter\def\csname PY@tok@fm\endcsname{\def\PY@tc##1{\textcolor[rgb]{0.00,0.00,1.00}{##1}}}
\expandafter\def\csname PY@tok@vc\endcsname{\def\PY@tc##1{\textcolor[rgb]{0.10,0.09,0.49}{##1}}}
\expandafter\def\csname PY@tok@vg\endcsname{\def\PY@tc##1{\textcolor[rgb]{0.10,0.09,0.49}{##1}}}
\expandafter\def\csname PY@tok@vi\endcsname{\def\PY@tc##1{\textcolor[rgb]{0.10,0.09,0.49}{##1}}}
\expandafter\def\csname PY@tok@vm\endcsname{\def\PY@tc##1{\textcolor[rgb]{0.10,0.09,0.49}{##1}}}
\expandafter\def\csname PY@tok@sa\endcsname{\def\PY@tc##1{\textcolor[rgb]{0.73,0.13,0.13}{##1}}}
\expandafter\def\csname PY@tok@sb\endcsname{\def\PY@tc##1{\textcolor[rgb]{0.73,0.13,0.13}{##1}}}
\expandafter\def\csname PY@tok@sc\endcsname{\def\PY@tc##1{\textcolor[rgb]{0.73,0.13,0.13}{##1}}}
\expandafter\def\csname PY@tok@dl\endcsname{\def\PY@tc##1{\textcolor[rgb]{0.73,0.13,0.13}{##1}}}
\expandafter\def\csname PY@tok@s2\endcsname{\def\PY@tc##1{\textcolor[rgb]{0.73,0.13,0.13}{##1}}}
\expandafter\def\csname PY@tok@sh\endcsname{\def\PY@tc##1{\textcolor[rgb]{0.73,0.13,0.13}{##1}}}
\expandafter\def\csname PY@tok@s1\endcsname{\def\PY@tc##1{\textcolor[rgb]{0.73,0.13,0.13}{##1}}}
\expandafter\def\csname PY@tok@mb\endcsname{\def\PY@tc##1{\textcolor[rgb]{0.40,0.40,0.40}{##1}}}
\expandafter\def\csname PY@tok@mf\endcsname{\def\PY@tc##1{\textcolor[rgb]{0.40,0.40,0.40}{##1}}}
\expandafter\def\csname PY@tok@mh\endcsname{\def\PY@tc##1{\textcolor[rgb]{0.40,0.40,0.40}{##1}}}
\expandafter\def\csname PY@tok@mi\endcsname{\def\PY@tc##1{\textcolor[rgb]{0.40,0.40,0.40}{##1}}}
\expandafter\def\csname PY@tok@il\endcsname{\def\PY@tc##1{\textcolor[rgb]{0.40,0.40,0.40}{##1}}}
\expandafter\def\csname PY@tok@mo\endcsname{\def\PY@tc##1{\textcolor[rgb]{0.40,0.40,0.40}{##1}}}
\expandafter\def\csname PY@tok@ch\endcsname{\let\PY@it=\textit\def\PY@tc##1{\textcolor[rgb]{0.25,0.50,0.50}{##1}}}
\expandafter\def\csname PY@tok@cm\endcsname{\let\PY@it=\textit\def\PY@tc##1{\textcolor[rgb]{0.25,0.50,0.50}{##1}}}
\expandafter\def\csname PY@tok@cpf\endcsname{\let\PY@it=\textit\def\PY@tc##1{\textcolor[rgb]{0.25,0.50,0.50}{##1}}}
\expandafter\def\csname PY@tok@c1\endcsname{\let\PY@it=\textit\def\PY@tc##1{\textcolor[rgb]{0.25,0.50,0.50}{##1}}}
\expandafter\def\csname PY@tok@cs\endcsname{\let\PY@it=\textit\def\PY@tc##1{\textcolor[rgb]{0.25,0.50,0.50}{##1}}}

\def\PYZbs{\char`\\}
\def\PYZus{\char`\_}
\def\PYZob{\char`\{}
\def\PYZcb{\char`\}}
\def\PYZca{\char`\^}
\def\PYZam{\char`\&}
\def\PYZlt{\char`\<}
\def\PYZgt{\char`\>}
\def\PYZsh{\char`\#}
\def\PYZpc{\char`\%}
\def\PYZdl{\char`\$}
\def\PYZhy{\char`\-}
\def\PYZsq{\char`\'}
\def\PYZdq{\char`\"}
\def\PYZti{\char`\~}
% for compatibility with earlier versions
\def\PYZat{@}
\def\PYZlb{[}
\def\PYZrb{]}
\makeatother


    % For linebreaks inside Verbatim environment from package fancyvrb. 
    \makeatletter
        \newbox\Wrappedcontinuationbox 
        \newbox\Wrappedvisiblespacebox 
        \newcommand*\Wrappedvisiblespace {\textcolor{red}{\textvisiblespace}} 
        \newcommand*\Wrappedcontinuationsymbol {\textcolor{red}{\llap{\tiny$\m@th\hookrightarrow$}}} 
        \newcommand*\Wrappedcontinuationindent {3ex } 
        \newcommand*\Wrappedafterbreak {\kern\Wrappedcontinuationindent\copy\Wrappedcontinuationbox} 
        % Take advantage of the already applied Pygments mark-up to insert 
        % potential linebreaks for TeX processing. 
        %        {, <, #, %, $, ' and ": go to next line. 
        %        _, }, ^, &, >, - and ~: stay at end of broken line. 
        % Use of \textquotesingle for straight quote. 
        \newcommand*\Wrappedbreaksatspecials {% 
            \def\PYGZus{\discretionary{\char`\_}{\Wrappedafterbreak}{\char`\_}}% 
            \def\PYGZob{\discretionary{}{\Wrappedafterbreak\char`\{}{\char`\{}}% 
            \def\PYGZcb{\discretionary{\char`\}}{\Wrappedafterbreak}{\char`\}}}% 
            \def\PYGZca{\discretionary{\char`\^}{\Wrappedafterbreak}{\char`\^}}% 
            \def\PYGZam{\discretionary{\char`\&}{\Wrappedafterbreak}{\char`\&}}% 
            \def\PYGZlt{\discretionary{}{\Wrappedafterbreak\char`\<}{\char`\<}}% 
            \def\PYGZgt{\discretionary{\char`\>}{\Wrappedafterbreak}{\char`\>}}% 
            \def\PYGZsh{\discretionary{}{\Wrappedafterbreak\char`\#}{\char`\#}}% 
            \def\PYGZpc{\discretionary{}{\Wrappedafterbreak\char`\%}{\char`\%}}% 
            \def\PYGZdl{\discretionary{}{\Wrappedafterbreak\char`\$}{\char`\$}}% 
            \def\PYGZhy{\discretionary{\char`\-}{\Wrappedafterbreak}{\char`\-}}% 
            \def\PYGZsq{\discretionary{}{\Wrappedafterbreak\textquotesingle}{\textquotesingle}}% 
            \def\PYGZdq{\discretionary{}{\Wrappedafterbreak\char`\"}{\char`\"}}% 
            \def\PYGZti{\discretionary{\char`\~}{\Wrappedafterbreak}{\char`\~}}% 
        } 
        % Some characters . , ; ? ! / are not pygmentized. 
        % This macro makes them "active" and they will insert potential linebreaks 
        \newcommand*\Wrappedbreaksatpunct {% 
            \lccode`\~`\.\lowercase{\def~}{\discretionary{\hbox{\char`\.}}{\Wrappedafterbreak}{\hbox{\char`\.}}}% 
            \lccode`\~`\,\lowercase{\def~}{\discretionary{\hbox{\char`\,}}{\Wrappedafterbreak}{\hbox{\char`\,}}}% 
            \lccode`\~`\;\lowercase{\def~}{\discretionary{\hbox{\char`\;}}{\Wrappedafterbreak}{\hbox{\char`\;}}}% 
            \lccode`\~`\:\lowercase{\def~}{\discretionary{\hbox{\char`\:}}{\Wrappedafterbreak}{\hbox{\char`\:}}}% 
            \lccode`\~`\?\lowercase{\def~}{\discretionary{\hbox{\char`\?}}{\Wrappedafterbreak}{\hbox{\char`\?}}}% 
            \lccode`\~`\!\lowercase{\def~}{\discretionary{\hbox{\char`\!}}{\Wrappedafterbreak}{\hbox{\char`\!}}}% 
            \lccode`\~`\/\lowercase{\def~}{\discretionary{\hbox{\char`\/}}{\Wrappedafterbreak}{\hbox{\char`\/}}}% 
            \catcode`\.\active
            \catcode`\,\active 
            \catcode`\;\active
            \catcode`\:\active
            \catcode`\?\active
            \catcode`\!\active
            \catcode`\/\active 
            \lccode`\~`\~ 	
        }
    \makeatother

    \let\OriginalVerbatim=\Verbatim
    \makeatletter
    \renewcommand{\Verbatim}[1][1]{%
        %\parskip\z@skip
        \sbox\Wrappedcontinuationbox {\Wrappedcontinuationsymbol}%
        \sbox\Wrappedvisiblespacebox {\FV@SetupFont\Wrappedvisiblespace}%
        \def\FancyVerbFormatLine ##1{\hsize\linewidth
            \vtop{\raggedright\hyphenpenalty\z@\exhyphenpenalty\z@
                \doublehyphendemerits\z@\finalhyphendemerits\z@
                \strut ##1\strut}%
        }%
        % If the linebreak is at a space, the latter will be displayed as visible
        % space at end of first line, and a continuation symbol starts next line.
        % Stretch/shrink are however usually zero for typewriter font.
        \def\FV@Space {%
            \nobreak\hskip\z@ plus\fontdimen3\font minus\fontdimen4\font
            \discretionary{\copy\Wrappedvisiblespacebox}{\Wrappedafterbreak}
            {\kern\fontdimen2\font}%
        }%
        
        % Allow breaks at special characters using \PYG... macros.
        \Wrappedbreaksatspecials
        % Breaks at punctuation characters . , ; ? ! and / need catcode=\active 	
        \OriginalVerbatim[#1,codes*=\Wrappedbreaksatpunct]%
    }
    \makeatother

    % Exact colors from NB
    \definecolor{incolor}{HTML}{303F9F}
    \definecolor{outcolor}{HTML}{D84315}
    \definecolor{cellborder}{HTML}{CFCFCF}
    \definecolor{cellbackground}{HTML}{F7F7F7}
    
    % prompt
    \makeatletter
    \newcommand{\boxspacing}{\kern\kvtcb@left@rule\kern\kvtcb@boxsep}
    \makeatother
    \newcommand{\prompt}[4]{
        {\ttfamily\llap{{\color{#2}[#3]:\hspace{3pt}#4}}\vspace{-\baselineskip}}
    }
    

    
    % Prevent overflowing lines due to hard-to-break entities
    \sloppy 
    % Setup hyperref package
    \hypersetup{
      breaklinks=true,  % so long urls are correctly broken across lines
      colorlinks=true,
      urlcolor=urlcolor,
      linkcolor=linkcolor,
      citecolor=citecolor,
      }
    % Slightly bigger margins than the latex defaults
    
    \geometry{verbose,tmargin=1in,bmargin=1in,lmargin=1in,rmargin=1in}
    
    

\begin{document}
    
    \maketitle
    
    

    
    \hypertarget{studia-podyplomowe-inux17cynieria-danych-data-science}{%
\subsubsection{Studia podyplomowe: Inżynieria Danych -- Data
Science}\label{studia-podyplomowe-inux17cynieria-danych-data-science}}

PRACA ZALICZENIOWA z Politechniki Gdańskiej \#\#\#\# autor: Artur
Karpiński

    \hypertarget{analiza-danych-filmowych-na-platformie-netflix}{%
\subsubsection{Analiza danych filmowych na platformie
Netflix}\label{analiza-danych-filmowych-na-platformie-netflix}}

Projekt oparty jest na wybranych danych dotyczących Systemu Rekomendacji
Filmów. Nie wykorzystuję indywidualnych danych związanych z
użytkownikami. W odróżnieniu od autora danych nie chodzi mi więc o
rekomendowanie, reklamowanie tytułów widzom na podstawie zebranych o
nich danych, prefencji. Wybrane dane: - tytuły filmów - czas wydania,
najważniejsze gatunki filmów - oceny i daty jej wystawienia

Na podstawie zebranych danych szukałem, analizowałem i pokazywałem
zależności między tytułami, gatunkami filmów, ich ocenami i
popularnością wśród widzów na przestrzeni czasu. Korzystałem z otwartych
danych z platformy Kaggle.

    \hypertarget{czyszczenie-danych-bez-plikuxf3w-cleaning}{%
\subsubsection{Czyszczenie danych (bez plików
cleaning)}\label{czyszczenie-danych-bez-plikuxf3w-cleaning}}

Wykorzystywać będziemy dane dwóch plików: 1. movies -- informacje
dotyczące filmów 2. ratings -- informacje dotyczące ocen tych filmów

    \begin{tcolorbox}[breakable, size=fbox, boxrule=1pt, pad at break*=1mm,colback=cellbackground, colframe=cellborder]
\prompt{In}{incolor}{1}{\boxspacing}
\begin{Verbatim}[commandchars=\\\{\}]
\PY{k+kn}{import} \PY{n+nn}{numpy} \PY{k}{as} \PY{n+nn}{np}
\PY{k+kn}{import} \PY{n+nn}{pandas} \PY{k}{as} \PY{n+nn}{pd}
\PY{k+kn}{import} \PY{n+nn}{matplotlib}\PY{n+nn}{.}\PY{n+nn}{pyplot} \PY{k}{as} \PY{n+nn}{plt}
\PY{k+kn}{import} \PY{n+nn}{seaborn} \PY{k}{as} \PY{n+nn}{sns}
\PY{k+kn}{import} \PY{n+nn}{sklearn}
\end{Verbatim}
\end{tcolorbox}

    \begin{tcolorbox}[breakable, size=fbox, boxrule=1pt, pad at break*=1mm,colback=cellbackground, colframe=cellborder]
\prompt{In}{incolor}{2}{\boxspacing}
\begin{Verbatim}[commandchars=\\\{\}]
\PY{c+c1}{\PYZsh{} informacje o filmach}

\PY{n}{df\PYZus{}movies} \PY{o}{=} \PY{n}{pd}\PY{o}{.}\PY{n}{read\PYZus{}csv}\PY{p}{(}\PY{l+s+s1}{\PYZsq{}}\PY{l+s+s1}{./data/movies.csv}\PY{l+s+s1}{\PYZsq{}}\PY{p}{)}
\end{Verbatim}
\end{tcolorbox}

    \begin{tcolorbox}[breakable, size=fbox, boxrule=1pt, pad at break*=1mm,colback=cellbackground, colframe=cellborder]
\prompt{In}{incolor}{3}{\boxspacing}
\begin{Verbatim}[commandchars=\\\{\}]
\PY{c+c1}{\PYZsh{} tabela zawiera więcej informacji niż ma kolumn}
\PY{c+c1}{\PYZsh{} w kolumnach więcej niż jedna informacja (np. rok, tytuł)}

\PY{n}{df\PYZus{}movies}\PY{o}{.}\PY{n}{head}\PY{p}{(}\PY{p}{)}
\end{Verbatim}
\end{tcolorbox}

            \begin{tcolorbox}[breakable, size=fbox, boxrule=.5pt, pad at break*=1mm, opacityfill=0]
\prompt{Out}{outcolor}{3}{\boxspacing}
\begin{Verbatim}[commandchars=\\\{\}]
   movieId                               title  \textbackslash{}
0        1                    Toy Story (1995)
1        2                      Jumanji (1995)
2        3             Grumpier Old Men (1995)
3        4            Waiting to Exhale (1995)
4        5  Father of the Bride Part II (1995)

                                        genres
0  Adventure|Animation|Children|Comedy|Fantasy
1                   Adventure|Children|Fantasy
2                               Comedy|Romance
3                         Comedy|Drama|Romance
4                                       Comedy
\end{Verbatim}
\end{tcolorbox}
        
    \begin{tcolorbox}[breakable, size=fbox, boxrule=1pt, pad at break*=1mm,colback=cellbackground, colframe=cellborder]
\prompt{In}{incolor}{4}{\boxspacing}
\begin{Verbatim}[commandchars=\\\{\}]
\PY{c+c1}{\PYZsh{} aby wyodrębnić tytuł i rok produkcji, string jako wzorzec wyrażenia regularnego poprzedzony r}
\PY{c+c1}{\PYZsh{} środku cztery liczby, expand=True, aby wzorzec z list string podzielić na kolumny}
\PY{c+c1}{\PYZsh{} po podziale tytułu na kolumny, usunięcie zbędnej, zmiana nazw kolumn}

\PY{n}{df\PYZus{}movies}\PY{p}{[}\PY{l+s+s1}{\PYZsq{}}\PY{l+s+s1}{title}\PY{l+s+s1}{\PYZsq{}}\PY{p}{]}\PY{o}{.}\PY{n}{str}\PY{o}{.}\PY{n}{split}\PY{p}{(}\PY{l+s+sa}{r}\PY{l+s+s1}{\PYZsq{}}\PY{l+s+s1}{(}\PY{l+s+s1}{\PYZbs{}}\PY{l+s+s1}{(}\PY{l+s+s1}{\PYZbs{}}\PY{l+s+s1}{d}\PY{l+s+s1}{\PYZbs{}}\PY{l+s+s1}{d}\PY{l+s+s1}{\PYZbs{}}\PY{l+s+s1}{d}\PY{l+s+s1}{\PYZbs{}}\PY{l+s+s1}{d}\PY{l+s+s1}{\PYZbs{}}\PY{l+s+s1}{))}\PY{l+s+s1}{\PYZsq{}}\PY{p}{,} \PY{n}{expand}\PY{o}{=}\PY{k+kc}{True}\PY{p}{)} \PYZbs{}
    \PY{o}{.}\PY{n}{drop}\PY{p}{(}\PY{n}{columns}\PY{o}{=}\PY{l+m+mi}{2}\PY{p}{)} \PYZbs{}
    \PY{o}{.}\PY{n}{rename}\PY{p}{(}\PY{n}{columns}\PY{o}{=}\PY{p}{\PYZob{}}\PY{l+m+mi}{0}\PY{p}{:} \PY{l+s+s1}{\PYZsq{}}\PY{l+s+s1}{title}\PY{l+s+s1}{\PYZsq{}}\PY{p}{,} \PY{l+m+mi}{1}\PY{p}{:} \PY{l+s+s1}{\PYZsq{}}\PY{l+s+s1}{year}\PY{l+s+s1}{\PYZsq{}}\PY{p}{\PYZcb{}}\PY{p}{)} \PYZbs{}
    \PY{o}{.}\PY{n}{head}\PY{p}{(}\PY{p}{)}
\end{Verbatim}
\end{tcolorbox}

            \begin{tcolorbox}[breakable, size=fbox, boxrule=.5pt, pad at break*=1mm, opacityfill=0]
\prompt{Out}{outcolor}{4}{\boxspacing}
\begin{Verbatim}[commandchars=\\\{\}]
                          title    year
0                    Toy Story   (1995)
1                      Jumanji   (1995)
2             Grumpier Old Men   (1995)
3            Waiting to Exhale   (1995)
4  Father of the Bride Part II   (1995)
\end{Verbatim}
\end{tcolorbox}
        
    \begin{tcolorbox}[breakable, size=fbox, boxrule=1pt, pad at break*=1mm,colback=cellbackground, colframe=cellborder]
\prompt{In}{incolor}{5}{\boxspacing}
\begin{Verbatim}[commandchars=\\\{\}]
\PY{c+c1}{\PYZsh{} \PYZus{}zmienna pomocnicza \PYZus{}df i usunięcie nawiasów przy year}

\PY{n}{\PYZus{}df} \PY{o}{=} \PY{n}{df\PYZus{}movies}\PY{p}{[}\PY{l+s+s1}{\PYZsq{}}\PY{l+s+s1}{title}\PY{l+s+s1}{\PYZsq{}}\PY{p}{]}\PY{o}{.}\PY{n}{str}\PY{o}{.}\PY{n}{split}\PY{p}{(}\PY{l+s+sa}{r}\PY{l+s+s1}{\PYZsq{}}\PY{l+s+s1}{(}\PY{l+s+s1}{\PYZbs{}}\PY{l+s+s1}{(}\PY{l+s+s1}{\PYZbs{}}\PY{l+s+s1}{d}\PY{l+s+s1}{\PYZbs{}}\PY{l+s+s1}{d}\PY{l+s+s1}{\PYZbs{}}\PY{l+s+s1}{d}\PY{l+s+s1}{\PYZbs{}}\PY{l+s+s1}{d}\PY{l+s+s1}{\PYZbs{}}\PY{l+s+s1}{))}\PY{l+s+s1}{\PYZsq{}}\PY{p}{,} \PY{n}{expand}\PY{o}{=}\PY{k+kc}{True}\PY{p}{)} \PYZbs{}
    \PY{o}{.}\PY{n}{drop}\PY{p}{(}\PY{n}{columns}\PY{o}{=}\PY{l+m+mi}{2}\PY{p}{)} \PYZbs{}
    \PY{o}{.}\PY{n}{rename}\PY{p}{(}\PY{n}{columns}\PY{o}{=}\PY{p}{\PYZob{}}\PY{l+m+mi}{0}\PY{p}{:} \PY{l+s+s1}{\PYZsq{}}\PY{l+s+s1}{title}\PY{l+s+s1}{\PYZsq{}}\PY{p}{,} \PY{l+m+mi}{1}\PY{p}{:} \PY{l+s+s1}{\PYZsq{}}\PY{l+s+s1}{year}\PY{l+s+s1}{\PYZsq{}}\PY{p}{\PYZcb{}}\PY{p}{)}
\PY{n}{df\PYZus{}movies}\PY{p}{[}\PY{l+s+s1}{\PYZsq{}}\PY{l+s+s1}{title}\PY{l+s+s1}{\PYZsq{}}\PY{p}{]} \PY{o}{=} \PY{n}{\PYZus{}df}\PY{p}{[}\PY{l+s+s1}{\PYZsq{}}\PY{l+s+s1}{title}\PY{l+s+s1}{\PYZsq{}}\PY{p}{]}
\PY{n}{df\PYZus{}movies}\PY{p}{[}\PY{l+s+s1}{\PYZsq{}}\PY{l+s+s1}{year}\PY{l+s+s1}{\PYZsq{}}\PY{p}{]} \PY{o}{=} \PY{n}{\PYZus{}df}\PY{p}{[}\PY{l+s+s1}{\PYZsq{}}\PY{l+s+s1}{year}\PY{l+s+s1}{\PYZsq{}}\PY{p}{]}\PY{o}{.}\PY{n}{str}\PY{o}{.}\PY{n}{strip}\PY{p}{(}\PY{l+s+s1}{\PYZsq{}}\PY{l+s+s1}{()}\PY{l+s+s1}{\PYZsq{}}\PY{p}{)}
\PY{k}{del} \PY{n}{\PYZus{}df}
\PY{n}{df\PYZus{}movies}\PY{o}{.}\PY{n}{head}\PY{p}{(}\PY{p}{)}
\end{Verbatim}
\end{tcolorbox}

            \begin{tcolorbox}[breakable, size=fbox, boxrule=.5pt, pad at break*=1mm, opacityfill=0]
\prompt{Out}{outcolor}{5}{\boxspacing}
\begin{Verbatim}[commandchars=\\\{\}]
   movieId                         title  \textbackslash{}
0        1                    Toy Story
1        2                      Jumanji
2        3             Grumpier Old Men
3        4            Waiting to Exhale
4        5  Father of the Bride Part II

                                        genres  year
0  Adventure|Animation|Children|Comedy|Fantasy  1995
1                   Adventure|Children|Fantasy  1995
2                               Comedy|Romance  1995
3                         Comedy|Drama|Romance  1995
4                                       Comedy  1995
\end{Verbatim}
\end{tcolorbox}
        
    \begin{tcolorbox}[breakable, size=fbox, boxrule=1pt, pad at break*=1mm,colback=cellbackground, colframe=cellborder]
\prompt{In}{incolor}{6}{\boxspacing}
\begin{Verbatim}[commandchars=\\\{\}]
\PY{c+c1}{\PYZsh{} lista i liczba brakujących elementów w year (68)}

\PY{n}{df\PYZus{}movies}\PY{p}{[}\PY{n}{df\PYZus{}movies}\PY{p}{[}\PY{l+s+s1}{\PYZsq{}}\PY{l+s+s1}{year}\PY{l+s+s1}{\PYZsq{}}\PY{p}{]}\PY{o}{.}\PY{n}{isna}\PY{p}{(}\PY{p}{)}\PY{p}{]}
\end{Verbatim}
\end{tcolorbox}

            \begin{tcolorbox}[breakable, size=fbox, boxrule=.5pt, pad at break*=1mm, opacityfill=0]
\prompt{Out}{outcolor}{6}{\boxspacing}
\begin{Verbatim}[commandchars=\\\{\}]
       movieId                                             title  \textbackslash{}
10603    40697                                         Babylon 5
15674    79607           Millions Game, The (Das Millionenspiel)
17376    87442  Bicycle, Spoon, Apple (Bicicleta, cullera, poma)
22471   107434                      Diplomatic Immunity (2009– )
22782   108548                      Big Bang Theory, The (2007-)
{\ldots}        {\ldots}                                               {\ldots}
34013   150655                                    Aimy in a Cage
34014   150657                                       Trophy Kids
34058   150868                               Jasne Błękitne Okna
34068   150888                                 Mr. Kuka's Advice
34186   151655                                            Hundra

                             genres  year
10603                        Sci-Fi  None
15674  Action|Drama|Sci-Fi|Thriller  None
17376                   Documentary  None
22471                        Comedy  None
22782                        Comedy  None
{\ldots}                             {\ldots}   {\ldots}
34013          Drama|Fantasy|Sci-Fi  None
34014                   Documentary  None
34058            (no genres listed)  None
34068            (no genres listed)  None
34186             Adventure|Fantasy  None

[68 rows x 4 columns]
\end{Verbatim}
\end{tcolorbox}
        
    \begin{tcolorbox}[breakable, size=fbox, boxrule=1pt, pad at break*=1mm,colback=cellbackground, colframe=cellborder]
\prompt{In}{incolor}{7}{\boxspacing}
\begin{Verbatim}[commandchars=\\\{\}]
\PY{c+c1}{\PYZsh{} null nie jest konwertowalny do liczby}
\PY{c+c1}{\PYZsh{} brakujące wartości zastępujemy \PYZhy{}1, bo taki rok na pewno nie istnieje}

\PY{n}{df\PYZus{}movies}\PY{p}{[}\PY{l+s+s1}{\PYZsq{}}\PY{l+s+s1}{year}\PY{l+s+s1}{\PYZsq{}}\PY{p}{]} \PY{o}{=} \PY{n}{df\PYZus{}movies}\PY{p}{[}\PY{l+s+s1}{\PYZsq{}}\PY{l+s+s1}{year}\PY{l+s+s1}{\PYZsq{}}\PY{p}{]}\PY{o}{.}\PY{n}{fillna}\PY{p}{(}\PY{o}{\PYZhy{}}\PY{l+m+mi}{1}\PY{p}{)}\PY{o}{.}\PY{n}{astype}\PY{p}{(}\PY{n}{np}\PY{o}{.}\PY{n}{int16}\PY{p}{)}
\PY{n}{df\PYZus{}movies}\PY{p}{[}\PY{l+s+s1}{\PYZsq{}}\PY{l+s+s1}{year}\PY{l+s+s1}{\PYZsq{}}\PY{p}{]}\PY{o}{.}\PY{n}{dtype}
\end{Verbatim}
\end{tcolorbox}

            \begin{tcolorbox}[breakable, size=fbox, boxrule=.5pt, pad at break*=1mm, opacityfill=0]
\prompt{Out}{outcolor}{7}{\boxspacing}
\begin{Verbatim}[commandchars=\\\{\}]
dtype('int16')
\end{Verbatim}
\end{tcolorbox}
        
    \begin{tcolorbox}[breakable, size=fbox, boxrule=1pt, pad at break*=1mm,colback=cellbackground, colframe=cellborder]
\prompt{In}{incolor}{8}{\boxspacing}
\begin{Verbatim}[commandchars=\\\{\}]
\PY{c+c1}{\PYZsh{} lista i liczba brakujących elementów w year (teraz puste wyeliminowane)}

\PY{n}{df\PYZus{}movies}\PY{p}{[}\PY{n}{df\PYZus{}movies}\PY{p}{[}\PY{l+s+s1}{\PYZsq{}}\PY{l+s+s1}{year}\PY{l+s+s1}{\PYZsq{}}\PY{p}{]}\PY{o}{.}\PY{n}{isna}\PY{p}{(}\PY{p}{)}\PY{p}{]}
\end{Verbatim}
\end{tcolorbox}

            \begin{tcolorbox}[breakable, size=fbox, boxrule=.5pt, pad at break*=1mm, opacityfill=0]
\prompt{Out}{outcolor}{8}{\boxspacing}
\begin{Verbatim}[commandchars=\\\{\}]
Empty DataFrame
Columns: [movieId, title, genres, year]
Index: []
\end{Verbatim}
\end{tcolorbox}
        
    \begin{tcolorbox}[breakable, size=fbox, boxrule=1pt, pad at break*=1mm,colback=cellbackground, colframe=cellborder]
\prompt{In}{incolor}{9}{\boxspacing}
\begin{Verbatim}[commandchars=\\\{\}]
\PY{n}{df\PYZus{}movies}\PY{o}{.}\PY{n}{head}\PY{p}{(}\PY{p}{)}
\end{Verbatim}
\end{tcolorbox}

            \begin{tcolorbox}[breakable, size=fbox, boxrule=.5pt, pad at break*=1mm, opacityfill=0]
\prompt{Out}{outcolor}{9}{\boxspacing}
\begin{Verbatim}[commandchars=\\\{\}]
   movieId                         title  \textbackslash{}
0        1                    Toy Story
1        2                      Jumanji
2        3             Grumpier Old Men
3        4            Waiting to Exhale
4        5  Father of the Bride Part II

                                        genres  year
0  Adventure|Animation|Children|Comedy|Fantasy  1995
1                   Adventure|Children|Fantasy  1995
2                               Comedy|Romance  1995
3                         Comedy|Drama|Romance  1995
4                                       Comedy  1995
\end{Verbatim}
\end{tcolorbox}
        
    \begin{tcolorbox}[breakable, size=fbox, boxrule=1pt, pad at break*=1mm,colback=cellbackground, colframe=cellborder]
\prompt{In}{incolor}{10}{\boxspacing}
\begin{Verbatim}[commandchars=\\\{\}]
\PY{c+c1}{\PYZsh{} genres (rodzaje filmów) w jednej kolumnie trudne wykorzystania (analizy, filtrowania itp.)}
\PY{c+c1}{\PYZsh{} \PYZsq{}no genres listed\PYZsq{} (brak rodzaju filmu) zamieniamy na pusty string bez kreski}
\PY{c+c1}{\PYZsh{} każdy wpis do małej litery}
\PY{c+c1}{\PYZsh{} dzielimy wpisy względem pionowej kreski}

\PY{n}{df\PYZus{}movies}\PY{p}{[}\PY{l+s+s1}{\PYZsq{}}\PY{l+s+s1}{genres}\PY{l+s+s1}{\PYZsq{}}\PY{p}{]} \PY{o}{=} \PY{n}{df\PYZus{}movies}\PY{p}{[}\PY{l+s+s1}{\PYZsq{}}\PY{l+s+s1}{genres}\PY{l+s+s1}{\PYZsq{}}\PY{p}{]}\PY{o}{.}\PY{n}{replace}\PY{p}{(}\PY{p}{\PYZob{}}\PY{l+s+s1}{\PYZsq{}}\PY{l+s+s1}{(no genres listed)}\PY{l+s+s1}{\PYZsq{}}\PY{p}{:} \PY{l+s+s1}{\PYZsq{}}\PY{l+s+s1}{\PYZsq{}}\PY{p}{\PYZcb{}}\PY{p}{)} \PYZbs{}
    \PY{o}{.}\PY{n}{str}\PY{o}{.}\PY{n}{lower}\PY{p}{(}\PY{p}{)} \PYZbs{}
    \PY{o}{.}\PY{n}{str}\PY{o}{.}\PY{n}{split}\PY{p}{(}\PY{l+s+s1}{\PYZsq{}}\PY{l+s+s1}{|}\PY{l+s+s1}{\PYZsq{}}\PY{p}{)}
\PY{n}{df\PYZus{}movies}
\end{Verbatim}
\end{tcolorbox}

            \begin{tcolorbox}[breakable, size=fbox, boxrule=.5pt, pad at break*=1mm, opacityfill=0]
\prompt{Out}{outcolor}{10}{\boxspacing}
\begin{Verbatim}[commandchars=\\\{\}]
       movieId                         title  \textbackslash{}
0            1                    Toy Story
1            2                      Jumanji
2            3             Grumpier Old Men
3            4            Waiting to Exhale
4            5  Father of the Bride Part II
{\ldots}        {\ldots}                           {\ldots}
34203   151697                   Grand Slam
34204   151701                   Bloodmoney
34205   151703         The Butterfly Circus
34206   151709                         Zero
34207   151711        The 2000 Year Old Man

                                                  genres  year
0      [adventure, animation, children, comedy, fantasy]  1995
1                         [adventure, children, fantasy]  1995
2                                      [comedy, romance]  1995
3                               [comedy, drama, romance]  1995
4                                               [comedy]  1995
{\ldots}                                                  {\ldots}   {\ldots}
34203                                         [thriller]  1967
34204                                                 []  2010
34205                                            [drama]  2009
34206                                    [drama, sci-fi]  2015
34207                                                 []  1975

[34208 rows x 4 columns]
\end{Verbatim}
\end{tcolorbox}
        
    \begin{tcolorbox}[breakable, size=fbox, boxrule=1pt, pad at break*=1mm,colback=cellbackground, colframe=cellborder]
\prompt{In}{incolor}{11}{\boxspacing}
\begin{Verbatim}[commandchars=\\\{\}]
\PY{c+c1}{\PYZsh{} teraz wyczyszczenie tabeli rating (oceny)}
\PY{c+c1}{\PYZsh{} wybieramy jedynie potrzebne kolumny}

\PY{n}{df\PYZus{}ratings} \PY{o}{=} \PY{n}{pd}\PY{o}{.}\PY{n}{read\PYZus{}csv}\PY{p}{(}\PY{l+s+s1}{\PYZsq{}}\PY{l+s+s1}{./data/ratings.csv}\PY{l+s+s1}{\PYZsq{}}\PY{p}{,} \PY{n}{usecols}\PY{o}{=}\PY{p}{[}\PY{l+s+s1}{\PYZsq{}}\PY{l+s+s1}{movieId}\PY{l+s+s1}{\PYZsq{}}\PY{p}{,} \PY{l+s+s1}{\PYZsq{}}\PY{l+s+s1}{rating}\PY{l+s+s1}{\PYZsq{}}\PY{p}{,} \PY{l+s+s1}{\PYZsq{}}\PY{l+s+s1}{timestamp}\PY{l+s+s1}{\PYZsq{}}\PY{p}{]}\PY{p}{)}

\PY{n}{df\PYZus{}ratings}\PY{o}{.}\PY{n}{head}\PY{p}{(}\PY{p}{)}
\end{Verbatim}
\end{tcolorbox}

            \begin{tcolorbox}[breakable, size=fbox, boxrule=.5pt, pad at break*=1mm, opacityfill=0]
\prompt{Out}{outcolor}{11}{\boxspacing}
\begin{Verbatim}[commandchars=\\\{\}]
   movieId  rating   timestamp
0      169     2.5  1204927694
1     2471     3.0  1204927438
2    48516     5.0  1204927435
3     2571     3.5  1436165433
4   109487     4.0  1436165496
\end{Verbatim}
\end{tcolorbox}
        
    \begin{tcolorbox}[breakable, size=fbox, boxrule=1pt, pad at break*=1mm,colback=cellbackground, colframe=cellborder]
\prompt{In}{incolor}{12}{\boxspacing}
\begin{Verbatim}[commandchars=\\\{\}]
\PY{c+c1}{\PYZsh{} konwertujemy te liczby do obiektów typu datetime}
\PY{c+c1}{\PYZsh{} wykorzystujemy funkcję pd.to\PYZus{}datetime (sekundy jako jednostki)}

\PY{n}{df\PYZus{}ratings}\PY{p}{[}\PY{l+s+s1}{\PYZsq{}}\PY{l+s+s1}{timestamp}\PY{l+s+s1}{\PYZsq{}}\PY{p}{]} \PY{o}{=} \PY{n}{pd}\PY{o}{.}\PY{n}{to\PYZus{}datetime}\PY{p}{(}\PY{n}{df\PYZus{}ratings}\PY{p}{[}\PY{l+s+s1}{\PYZsq{}}\PY{l+s+s1}{timestamp}\PY{l+s+s1}{\PYZsq{}}\PY{p}{]}\PY{p}{,} \PY{n}{unit}\PY{o}{=}\PY{l+s+s1}{\PYZsq{}}\PY{l+s+s1}{s}\PY{l+s+s1}{\PYZsq{}}\PY{p}{)}
\PY{n}{df\PYZus{}ratings}\PY{o}{.}\PY{n}{head}\PY{p}{(}\PY{p}{)}
\end{Verbatim}
\end{tcolorbox}

            \begin{tcolorbox}[breakable, size=fbox, boxrule=.5pt, pad at break*=1mm, opacityfill=0]
\prompt{Out}{outcolor}{12}{\boxspacing}
\begin{Verbatim}[commandchars=\\\{\}]
   movieId  rating           timestamp
0      169     2.5 2008-03-07 22:08:14
1     2471     3.0 2008-03-07 22:03:58
2    48516     5.0 2008-03-07 22:03:55
3     2571     3.5 2015-07-06 06:50:33
4   109487     4.0 2015-07-06 06:51:36
\end{Verbatim}
\end{tcolorbox}
        
    \begin{tcolorbox}[breakable, size=fbox, boxrule=1pt, pad at break*=1mm,colback=cellbackground, colframe=cellborder]
\prompt{In}{incolor}{13}{\boxspacing}
\begin{Verbatim}[commandchars=\\\{\}]
\PY{c+c1}{\PYZsh{} sprawdzamy czy timestamp zawiera jakieś informacje czasowe}
\PY{c+c1}{\PYZsh{} to liczba sekund od jakiegoś umownego roku}

\PY{n}{df\PYZus{}ratings}\PY{p}{[}\PY{l+s+s1}{\PYZsq{}}\PY{l+s+s1}{timestamp}\PY{l+s+s1}{\PYZsq{}}\PY{p}{]}
\end{Verbatim}
\end{tcolorbox}

            \begin{tcolorbox}[breakable, size=fbox, boxrule=.5pt, pad at break*=1mm, opacityfill=0]
\prompt{Out}{outcolor}{13}{\boxspacing}
\begin{Verbatim}[commandchars=\\\{\}]
0         2008-03-07 22:08:14
1         2008-03-07 22:03:58
2         2008-03-07 22:03:55
3         2015-07-06 06:50:33
4         2015-07-06 06:51:36
                  {\ldots}
1009994   1997-05-10 19:17:26
1009995   1997-01-19 11:54:11
1009996   1997-01-18 19:24:30
1009997   1997-01-19 17:40:53
1009998   1997-01-19 17:40:53
Name: timestamp, Length: 1009999, dtype: datetime64[ns]
\end{Verbatim}
\end{tcolorbox}
        
    \begin{tcolorbox}[breakable, size=fbox, boxrule=1pt, pad at break*=1mm,colback=cellbackground, colframe=cellborder]
\prompt{In}{incolor}{14}{\boxspacing}
\begin{Verbatim}[commandchars=\\\{\}]
\PY{c+c1}{\PYZsh{} ocena najniższa}

\PY{n}{df\PYZus{}ratings}\PY{p}{[}\PY{l+s+s1}{\PYZsq{}}\PY{l+s+s1}{rating}\PY{l+s+s1}{\PYZsq{}}\PY{p}{]}\PY{o}{.}\PY{n}{min}\PY{p}{(}\PY{p}{)}
\end{Verbatim}
\end{tcolorbox}

            \begin{tcolorbox}[breakable, size=fbox, boxrule=.5pt, pad at break*=1mm, opacityfill=0]
\prompt{Out}{outcolor}{14}{\boxspacing}
\begin{Verbatim}[commandchars=\\\{\}]
0.5
\end{Verbatim}
\end{tcolorbox}
        
    \begin{tcolorbox}[breakable, size=fbox, boxrule=1pt, pad at break*=1mm,colback=cellbackground, colframe=cellborder]
\prompt{In}{incolor}{15}{\boxspacing}
\begin{Verbatim}[commandchars=\\\{\}]
\PY{c+c1}{\PYZsh{} ocena najniższa}

\PY{n}{df\PYZus{}ratings}\PY{p}{[}\PY{l+s+s1}{\PYZsq{}}\PY{l+s+s1}{rating}\PY{l+s+s1}{\PYZsq{}}\PY{p}{]}\PY{o}{.}\PY{n}{max}\PY{p}{(}\PY{p}{)}
\end{Verbatim}
\end{tcolorbox}

            \begin{tcolorbox}[breakable, size=fbox, boxrule=.5pt, pad at break*=1mm, opacityfill=0]
\prompt{Out}{outcolor}{15}{\boxspacing}
\begin{Verbatim}[commandchars=\\\{\}]
5.0
\end{Verbatim}
\end{tcolorbox}
        
    \hypertarget{analiza-danych}{%
\subsubsection{Analiza danych}\label{analiza-danych}}

\hypertarget{filmy-z-platformy-netflix.}{%
\paragraph{Filmy z platformy
Netflix.}\label{filmy-z-platformy-netflix.}}

    \begin{tcolorbox}[breakable, size=fbox, boxrule=1pt, pad at break*=1mm,colback=cellbackground, colframe=cellborder]
\prompt{In}{incolor}{16}{\boxspacing}
\begin{Verbatim}[commandchars=\\\{\}]
\PY{c+c1}{\PYZsh{} oczyszczone dane o filmach}

\PY{n}{df\PYZus{}movies}\PY{o}{.}\PY{n}{head}\PY{p}{(}\PY{p}{)}
\end{Verbatim}
\end{tcolorbox}

            \begin{tcolorbox}[breakable, size=fbox, boxrule=.5pt, pad at break*=1mm, opacityfill=0]
\prompt{Out}{outcolor}{16}{\boxspacing}
\begin{Verbatim}[commandchars=\\\{\}]
   movieId                         title  \textbackslash{}
0        1                    Toy Story
1        2                      Jumanji
2        3             Grumpier Old Men
3        4            Waiting to Exhale
4        5  Father of the Bride Part II

                                              genres  year
0  [adventure, animation, children, comedy, fantasy]  1995
1                     [adventure, children, fantasy]  1995
2                                  [comedy, romance]  1995
3                           [comedy, drama, romance]  1995
4                                           [comedy]  1995
\end{Verbatim}
\end{tcolorbox}
        
    \begin{tcolorbox}[breakable, size=fbox, boxrule=1pt, pad at break*=1mm,colback=cellbackground, colframe=cellborder]
\prompt{In}{incolor}{17}{\boxspacing}
\begin{Verbatim}[commandchars=\\\{\}]
\PY{c+c1}{\PYZsh{} liczba filmów w poszczególnych latach}
\PY{c+c1}{\PYZsh{} loc[0:] filtruje filmy bez przypisanego roku (\PYZhy{}1)}

\PY{n}{df\PYZus{}movies}\PY{p}{[}\PY{p}{[}\PY{l+s+s1}{\PYZsq{}}\PY{l+s+s1}{movieId}\PY{l+s+s1}{\PYZsq{}}\PY{p}{,} \PY{l+s+s1}{\PYZsq{}}\PY{l+s+s1}{year}\PY{l+s+s1}{\PYZsq{}}\PY{p}{]}\PY{p}{]}\PY{o}{.}\PY{n}{groupby}\PY{p}{(}\PY{l+s+s1}{\PYZsq{}}\PY{l+s+s1}{year}\PY{l+s+s1}{\PYZsq{}}\PY{p}{)}\PY{o}{.}\PY{n}{count}\PY{p}{(}\PY{p}{)}
\end{Verbatim}
\end{tcolorbox}

            \begin{tcolorbox}[breakable, size=fbox, boxrule=.5pt, pad at break*=1mm, opacityfill=0]
\prompt{Out}{outcolor}{17}{\boxspacing}
\begin{Verbatim}[commandchars=\\\{\}]
       movieId
year
-1          68
 1874        1
 1878        1
 1887        1
 1888        2
{\ldots}        {\ldots}
 2012     1387
 2013     1476
 2014     1420
 2015     1036
 2016       23

[130 rows x 1 columns]
\end{Verbatim}
\end{tcolorbox}
        
    \begin{tcolorbox}[breakable, size=fbox, boxrule=1pt, pad at break*=1mm,colback=cellbackground, colframe=cellborder]
\prompt{In}{incolor}{18}{\boxspacing}
\begin{Verbatim}[commandchars=\\\{\}]
\PY{c+c1}{\PYZsh{} liczba filmów w poszczególnych latach}
\PY{c+c1}{\PYZsh{} nie do końca zdefiniowany koniec (nie wszystkie dane z ostatniego roku)}

\PY{n}{df\PYZus{}movies}\PY{p}{[}\PY{p}{[}\PY{l+s+s1}{\PYZsq{}}\PY{l+s+s1}{movieId}\PY{l+s+s1}{\PYZsq{}}\PY{p}{,} \PY{l+s+s1}{\PYZsq{}}\PY{l+s+s1}{year}\PY{l+s+s1}{\PYZsq{}}\PY{p}{]}\PY{p}{]}\PY{o}{.}\PY{n}{groupby}\PY{p}{(}\PY{l+s+s1}{\PYZsq{}}\PY{l+s+s1}{year}\PY{l+s+s1}{\PYZsq{}}\PY{p}{)}\PY{o}{.}\PY{n}{count}\PY{p}{(}\PY{p}{)}\PY{o}{.}\PY{n}{loc}\PY{p}{[}\PY{l+m+mi}{0}\PY{p}{:}\PY{l+m+mi}{2014}\PY{p}{]}\PY{o}{.}\PY{n}{plot}\PY{p}{(}\PY{n}{figsize}\PY{o}{=}\PY{p}{(}\PY{l+m+mi}{12}\PY{p}{,} \PY{l+m+mi}{3}\PY{p}{)}\PY{p}{)}
\PY{n}{plt}\PY{o}{.}\PY{n}{title}\PY{p}{(}\PY{l+s+s1}{\PYZsq{}}\PY{l+s+s1}{Liczba wszystkich filmów do roku 2014}\PY{l+s+s1}{\PYZsq{}}\PY{p}{)}
\PY{n}{df\PYZus{}movies}\PY{p}{[}\PY{p}{[}\PY{l+s+s1}{\PYZsq{}}\PY{l+s+s1}{movieId}\PY{l+s+s1}{\PYZsq{}}\PY{p}{,} \PY{l+s+s1}{\PYZsq{}}\PY{l+s+s1}{year}\PY{l+s+s1}{\PYZsq{}}\PY{p}{]}\PY{p}{]}\PY{o}{.}\PY{n}{groupby}\PY{p}{(}\PY{l+s+s1}{\PYZsq{}}\PY{l+s+s1}{year}\PY{l+s+s1}{\PYZsq{}}\PY{p}{)}\PY{o}{.}\PY{n}{count}\PY{p}{(}\PY{p}{)}\PY{o}{.}\PY{n}{loc}\PY{p}{[}\PY{l+m+mi}{2000}\PY{p}{:}\PY{l+m+mi}{2014}\PY{p}{]}\PY{o}{.}\PY{n}{plot}\PY{p}{(}\PY{n}{figsize}\PY{o}{=}\PY{p}{(}\PY{l+m+mi}{12}\PY{p}{,} \PY{l+m+mi}{3}\PY{p}{)}\PY{p}{)}
\PY{n}{plt}\PY{o}{.}\PY{n}{title}\PY{p}{(}\PY{l+s+s1}{\PYZsq{}}\PY{l+s+s1}{Liczba filmów w latach 2000 – 2014}\PY{l+s+s1}{\PYZsq{}}\PY{p}{)}
\end{Verbatim}
\end{tcolorbox}

            \begin{tcolorbox}[breakable, size=fbox, boxrule=.5pt, pad at break*=1mm, opacityfill=0]
\prompt{Out}{outcolor}{18}{\boxspacing}
\begin{Verbatim}[commandchars=\\\{\}]
Text(0.5, 1.0, 'Liczba filmów w latach 2000 – 2014')
\end{Verbatim}
\end{tcolorbox}
        
    \begin{center}
    \adjustimage{max size={0.9\linewidth}{0.9\paperheight}}{output_21_1.png}
    \end{center}
    { \hspace*{\fill} \\}
    
    \begin{center}
    \adjustimage{max size={0.9\linewidth}{0.9\paperheight}}{output_21_2.png}
    \end{center}
    { \hspace*{\fill} \\}
    
    \hypertarget{cel-badanie-zwiux105zkuxf3w-miux119dzy-latami-produkcji-gatunkami-filmuxf3w-ich-ocenux105}{%
\paragraph{Cel: badanie związków między latami produkcji, gatunkami
filmów, ich
oceną}\label{cel-badanie-zwiux105zkuxf3w-miux119dzy-latami-produkcji-gatunkami-filmuxf3w-ich-ocenux105}}

    \begin{tcolorbox}[breakable, size=fbox, boxrule=1pt, pad at break*=1mm,colback=cellbackground, colframe=cellborder]
\prompt{In}{incolor}{19}{\boxspacing}
\begin{Verbatim}[commandchars=\\\{\}]
\PY{c+c1}{\PYZsh{} za pomocą funkcji lambda konwersja do listy każdego rzędu gatunków filmów }

\PY{n}{df\PYZus{}movies}\PY{p}{[}\PY{l+s+s1}{\PYZsq{}}\PY{l+s+s1}{genres}\PY{l+s+s1}{\PYZsq{}}\PY{p}{]}\PY{o}{.}\PY{n}{apply}\PY{p}{(}\PY{k}{lambda} \PY{n}{x}\PY{p}{:} \PY{n+nb}{list}\PY{p}{(}\PY{n}{x}\PY{p}{)}\PY{p}{)}
\end{Verbatim}
\end{tcolorbox}

            \begin{tcolorbox}[breakable, size=fbox, boxrule=.5pt, pad at break*=1mm, opacityfill=0]
\prompt{Out}{outcolor}{19}{\boxspacing}
\begin{Verbatim}[commandchars=\\\{\}]
0        [adventure, animation, children, comedy, fantasy]
1                           [adventure, children, fantasy]
2                                        [comedy, romance]
3                                 [comedy, drama, romance]
4                                                 [comedy]
                               {\ldots}
34203                                           [thriller]
34204                                                   []
34205                                              [drama]
34206                                      [drama, sci-fi]
34207                                                   []
Name: genres, Length: 34208, dtype: object
\end{Verbatim}
\end{tcolorbox}
        
    \begin{tcolorbox}[breakable, size=fbox, boxrule=1pt, pad at break*=1mm,colback=cellbackground, colframe=cellborder]
\prompt{In}{incolor}{20}{\boxspacing}
\begin{Verbatim}[commandchars=\\\{\}]
\PY{c+c1}{\PYZsh{} dzięki explode na kolumnie genres gatunki ułożą się jeden pod drugim}

\PY{n}{df\PYZus{}movies}\PY{p}{[}\PY{l+s+s1}{\PYZsq{}}\PY{l+s+s1}{genres}\PY{l+s+s1}{\PYZsq{}}\PY{p}{]} \PY{o}{=} \PY{n}{df\PYZus{}movies}\PY{p}{[}\PY{l+s+s1}{\PYZsq{}}\PY{l+s+s1}{genres}\PY{l+s+s1}{\PYZsq{}}\PY{p}{]}\PY{o}{.}\PY{n}{apply}\PY{p}{(}\PY{k}{lambda} \PY{n}{x}\PY{p}{:} \PY{n+nb}{list}\PY{p}{(}\PY{n}{x}\PY{p}{)}\PY{p}{)}
\PY{n}{df1} \PY{o}{=} \PY{n}{df\PYZus{}movies}\PY{o}{.}\PY{n}{explode}\PY{p}{(}\PY{l+s+s1}{\PYZsq{}}\PY{l+s+s1}{genres}\PY{l+s+s1}{\PYZsq{}}\PY{p}{)}
\PY{n}{df1}
\end{Verbatim}
\end{tcolorbox}

            \begin{tcolorbox}[breakable, size=fbox, boxrule=.5pt, pad at break*=1mm, opacityfill=0]
\prompt{Out}{outcolor}{20}{\boxspacing}
\begin{Verbatim}[commandchars=\\\{\}]
       movieId                   title     genres  year
0            1              Toy Story   adventure  1995
0            1              Toy Story   animation  1995
0            1              Toy Story    children  1995
0            1              Toy Story      comedy  1995
0            1              Toy Story     fantasy  1995
{\ldots}        {\ldots}                     {\ldots}        {\ldots}   {\ldots}
34204   151701             Bloodmoney              2010
34205   151703   The Butterfly Circus       drama  2009
34206   151709                   Zero       drama  2015
34206   151709                   Zero      sci-fi  2015
34207   151711  The 2000 Year Old Man              1975

[66668 rows x 4 columns]
\end{Verbatim}
\end{tcolorbox}
        
    \begin{tcolorbox}[breakable, size=fbox, boxrule=1pt, pad at break*=1mm,colback=cellbackground, colframe=cellborder]
\prompt{In}{incolor}{21}{\boxspacing}
\begin{Verbatim}[commandchars=\\\{\}]
\PY{c+c1}{\PYZsh{} ilość gatunków filmów w każdym zbiorze}

\PY{n}{df1}\PY{p}{[}\PY{l+s+s1}{\PYZsq{}}\PY{l+s+s1}{genres}\PY{l+s+s1}{\PYZsq{}}\PY{p}{]}\PY{o}{.}\PY{n}{value\PYZus{}counts}\PY{p}{(}\PY{p}{)}
\end{Verbatim}
\end{tcolorbox}

            \begin{tcolorbox}[breakable, size=fbox, boxrule=.5pt, pad at break*=1mm, opacityfill=0]
\prompt{Out}{outcolor}{21}{\boxspacing}
\begin{Verbatim}[commandchars=\\\{\}]
drama          15774
comedy         10124
thriller        5300
romance         4875
action          4445
crime           3446
horror          3365
documentary     3040
adventure       2763
sci-fi          2156
mystery         1837
fantasy         1692
children        1609
animation       1387
war             1345
                1145
musical         1052
western          779
film-noir        338
imax             196
Name: genres, dtype: int64
\end{Verbatim}
\end{tcolorbox}
        
    \begin{tcolorbox}[breakable, size=fbox, boxrule=1pt, pad at break*=1mm,colback=cellbackground, colframe=cellborder]
\prompt{In}{incolor}{22}{\boxspacing}
\begin{Verbatim}[commandchars=\\\{\}]
\PY{c+c1}{\PYZsh{} wykres horyzontalny (kind=\PYZsq{}barth\PYZsq{}) od wartości najmniejszej}

\PY{n}{df1}\PY{p}{[}\PY{l+s+s1}{\PYZsq{}}\PY{l+s+s1}{genres}\PY{l+s+s1}{\PYZsq{}}\PY{p}{]}\PY{o}{.}\PY{n}{value\PYZus{}counts}\PY{p}{(}\PY{p}{)}\PY{o}{.}\PY{n}{sort\PYZus{}values}\PY{p}{(}\PY{n}{ascending}\PY{o}{=}\PY{k+kc}{True}\PY{p}{)}\PY{o}{.}\PY{n}{plot}\PY{p}{(}\PY{n}{kind}\PY{o}{=}\PY{l+s+s1}{\PYZsq{}}\PY{l+s+s1}{barh}\PY{l+s+s1}{\PYZsq{}}\PY{p}{,} \PY{n}{figsize}\PY{o}{=}\PY{p}{(}\PY{l+m+mi}{12}\PY{p}{,} \PY{l+m+mi}{5}\PY{p}{)}\PY{p}{)}
\PY{n}{plt}\PY{o}{.}\PY{n}{title}\PY{p}{(}\PY{l+s+s1}{\PYZsq{}}\PY{l+s+s1}{Ilość filmów w gatunkach}\PY{l+s+s1}{\PYZsq{}}\PY{p}{)}
\end{Verbatim}
\end{tcolorbox}

            \begin{tcolorbox}[breakable, size=fbox, boxrule=.5pt, pad at break*=1mm, opacityfill=0]
\prompt{Out}{outcolor}{22}{\boxspacing}
\begin{Verbatim}[commandchars=\\\{\}]
Text(0.5, 1.0, 'Ilość filmów w gatunkach')
\end{Verbatim}
\end{tcolorbox}
        
    \begin{center}
    \adjustimage{max size={0.9\linewidth}{0.9\paperheight}}{output_26_1.png}
    \end{center}
    { \hspace*{\fill} \\}
    
    \begin{tcolorbox}[breakable, size=fbox, boxrule=1pt, pad at break*=1mm,colback=cellbackground, colframe=cellborder]
\prompt{In}{incolor}{23}{\boxspacing}
\begin{Verbatim}[commandchars=\\\{\}]
\PY{c+c1}{\PYZsh{} często filmy są przydzielone do wielu kategorii}
\PY{c+c1}{\PYZsh{} dzięki \PYZsq{}value\PYZus{}counts\PYZsq{} ilość tych wystąpień}

\PY{n}{df\PYZus{}movies}\PY{p}{[}\PY{l+s+s1}{\PYZsq{}}\PY{l+s+s1}{genres}\PY{l+s+s1}{\PYZsq{}}\PY{p}{]}\PY{o}{.}\PY{n}{apply}\PY{p}{(}\PY{n+nb}{len}\PY{p}{)}\PY{o}{.}\PY{n}{value\PYZus{}counts}\PY{p}{(}\PY{p}{)}
\end{Verbatim}
\end{tcolorbox}

            \begin{tcolorbox}[breakable, size=fbox, boxrule=.5pt, pad at break*=1mm, opacityfill=0]
\prompt{Out}{outcolor}{23}{\boxspacing}
\begin{Verbatim}[commandchars=\\\{\}]
1     14372
2     10719
3      6432
4      2023
5       537
6        98
7        21
8         5
10        1
Name: genres, dtype: int64
\end{Verbatim}
\end{tcolorbox}
        
    \begin{tcolorbox}[breakable, size=fbox, boxrule=1pt, pad at break*=1mm,colback=cellbackground, colframe=cellborder]
\prompt{In}{incolor}{24}{\boxspacing}
\begin{Verbatim}[commandchars=\\\{\}]
\PY{c+c1}{\PYZsh{} wykres słupkowy (kind=\PYZsq{}bar\PYZsq{})}

\PY{n}{df\PYZus{}movies}\PY{p}{[}\PY{l+s+s1}{\PYZsq{}}\PY{l+s+s1}{genres}\PY{l+s+s1}{\PYZsq{}}\PY{p}{]}\PY{o}{.}\PY{n}{apply}\PY{p}{(}\PY{n+nb}{len}\PY{p}{)}\PY{o}{.}\PY{n}{value\PYZus{}counts}\PY{p}{(}\PY{p}{)}\PY{o}{.}\PY{n}{plot}\PY{p}{(}\PY{n}{kind}\PY{o}{=}\PY{l+s+s1}{\PYZsq{}}\PY{l+s+s1}{bar}\PY{l+s+s1}{\PYZsq{}}\PY{p}{,} \PY{n}{figsize}\PY{o}{=}\PY{p}{(}\PY{l+m+mi}{12}\PY{p}{,} \PY{l+m+mi}{3}\PY{p}{)}\PY{p}{)}
\PY{n}{plt}\PY{o}{.}\PY{n}{title}\PY{p}{(}\PY{l+s+s1}{\PYZsq{}}\PY{l+s+s1}{Liczba gatunków przydzielonych filmom}\PY{l+s+s1}{\PYZsq{}}\PY{p}{)}
\end{Verbatim}
\end{tcolorbox}

            \begin{tcolorbox}[breakable, size=fbox, boxrule=.5pt, pad at break*=1mm, opacityfill=0]
\prompt{Out}{outcolor}{24}{\boxspacing}
\begin{Verbatim}[commandchars=\\\{\}]
Text(0.5, 1.0, 'Liczba gatunków przydzielonych filmom')
\end{Verbatim}
\end{tcolorbox}
        
    \begin{center}
    \adjustimage{max size={0.9\linewidth}{0.9\paperheight}}{output_28_1.png}
    \end{center}
    { \hspace*{\fill} \\}
    
    \begin{tcolorbox}[breakable, size=fbox, boxrule=1pt, pad at break*=1mm,colback=cellbackground, colframe=cellborder]
\prompt{In}{incolor}{25}{\boxspacing}
\begin{Verbatim}[commandchars=\\\{\}]
\PY{c+c1}{\PYZsh{} filmy po 1940 roku}

\PY{n}{df2} \PY{o}{=} \PY{n}{df1}\PY{p}{[}\PY{n}{df1}\PY{p}{[}\PY{l+s+s1}{\PYZsq{}}\PY{l+s+s1}{year}\PY{l+s+s1}{\PYZsq{}}\PY{p}{]} \PY{o}{\PYZgt{}} \PY{l+m+mi}{1940}\PY{p}{]}
\PY{n}{df2}
\end{Verbatim}
\end{tcolorbox}

            \begin{tcolorbox}[breakable, size=fbox, boxrule=.5pt, pad at break*=1mm, opacityfill=0]
\prompt{Out}{outcolor}{25}{\boxspacing}
\begin{Verbatim}[commandchars=\\\{\}]
       movieId                   title     genres  year
0            1              Toy Story   adventure  1995
0            1              Toy Story   animation  1995
0            1              Toy Story    children  1995
0            1              Toy Story      comedy  1995
0            1              Toy Story     fantasy  1995
{\ldots}        {\ldots}                     {\ldots}        {\ldots}   {\ldots}
34204   151701             Bloodmoney              2010
34205   151703   The Butterfly Circus       drama  2009
34206   151709                   Zero       drama  2015
34206   151709                   Zero      sci-fi  2015
34207   151711  The 2000 Year Old Man              1975

[63231 rows x 4 columns]
\end{Verbatim}
\end{tcolorbox}
        
    \begin{tcolorbox}[breakable, size=fbox, boxrule=1pt, pad at break*=1mm,colback=cellbackground, colframe=cellborder]
\prompt{In}{incolor}{26}{\boxspacing}
\begin{Verbatim}[commandchars=\\\{\}]
\PY{c+c1}{\PYZsh{} tworzenie podgrup dla filmów w poszczególnych dekadach}
\PY{c+c1}{\PYZsh{} dla grupowania filmów po przedziałach w poszczególnych latach funkcja \PYZsq{}cut\PYZsq{}}
\PY{c+c1}{\PYZsh{} \PYZsq{}bins\PYZsq{} to kieszenie,\PYZsq{}range\PYZsq{} to zakres}
\PY{c+c1}{\PYZsh{} przedział dla lat 1940\PYZhy{}2020 co 10 lat}

\PY{n}{pd}\PY{o}{.}\PY{n}{cut}\PY{p}{(}\PY{n}{df2}\PY{p}{[}\PY{l+s+s1}{\PYZsq{}}\PY{l+s+s1}{year}\PY{l+s+s1}{\PYZsq{}}\PY{p}{]}\PY{p}{,} \PY{n}{bins}\PY{o}{=}\PY{n+nb}{range}\PY{p}{(}\PY{l+m+mi}{1940}\PY{p}{,} \PY{l+m+mi}{2030}\PY{p}{,} \PY{l+m+mi}{10}\PY{p}{)}\PY{p}{,} \PY{n}{labels}\PY{o}{=}\PY{p}{[}\PY{l+s+sa}{f}\PY{l+s+s2}{\PYZdq{}}\PY{l+s+si}{\PYZob{}}\PY{n}{x}\PY{l+s+si}{\PYZcb{}}\PY{l+s+s2}{s}\PY{l+s+s2}{\PYZdq{}} \PYZbs{}
    \PY{k}{for} \PY{n}{x} \PY{o+ow}{in} \PY{n+nb}{range}\PY{p}{(}\PY{l+m+mi}{40}\PY{p}{,} \PY{l+m+mi}{100}\PY{p}{,} \PY{l+m+mi}{10}\PY{p}{)}\PY{p}{]} \PY{o}{+} \PY{p}{[}\PY{l+s+s1}{\PYZsq{}}\PY{l+s+s1}{2000s}\PY{l+s+s1}{\PYZsq{}}\PY{p}{,} \PY{l+s+s1}{\PYZsq{}}\PY{l+s+s1}{2010s}\PY{l+s+s1}{\PYZsq{}}\PY{p}{]}\PY{p}{)}
\end{Verbatim}
\end{tcolorbox}

            \begin{tcolorbox}[breakable, size=fbox, boxrule=.5pt, pad at break*=1mm, opacityfill=0]
\prompt{Out}{outcolor}{26}{\boxspacing}
\begin{Verbatim}[commandchars=\\\{\}]
0          90s
0          90s
0          90s
0          90s
0          90s
         {\ldots}
34204    2000s
34205    2000s
34206    2010s
34206    2010s
34207      70s
Name: year, Length: 63231, dtype: category
Categories (8, object): ['40s' < '50s' < '60s' < '70s' < '80s' < '90s' < '2000s'
< '2010s']
\end{Verbatim}
\end{tcolorbox}
        
    \begin{tcolorbox}[breakable, size=fbox, boxrule=1pt, pad at break*=1mm,colback=cellbackground, colframe=cellborder]
\prompt{In}{incolor}{27}{\boxspacing}
\begin{Verbatim}[commandchars=\\\{\}]
\PY{c+c1}{\PYZsh{} liczba filmów w czasach, dekadach}

\PY{n}{df2}\PY{p}{[}\PY{l+s+s1}{\PYZsq{}}\PY{l+s+s1}{times}\PY{l+s+s1}{\PYZsq{}}\PY{p}{]} \PY{o}{=} \PY{n}{pd}\PY{o}{.}\PY{n}{cut}\PY{p}{(}\PY{n}{df2}\PY{p}{[}\PY{l+s+s1}{\PYZsq{}}\PY{l+s+s1}{year}\PY{l+s+s1}{\PYZsq{}}\PY{p}{]}\PY{p}{,} \PY{n}{bins}\PY{o}{=}\PY{n+nb}{range}\PY{p}{(}\PY{l+m+mi}{1940}\PY{p}{,} \PY{l+m+mi}{2030}\PY{p}{,} \PY{l+m+mi}{10}\PY{p}{)}\PY{p}{,} \PYZbs{}
    \PY{n}{labels}\PY{o}{=}\PY{p}{[}\PY{l+s+sa}{f}\PY{l+s+s2}{\PYZdq{}}\PY{l+s+si}{\PYZob{}}\PY{n}{x}\PY{l+s+si}{\PYZcb{}}\PY{l+s+s2}{s}\PY{l+s+s2}{\PYZdq{}} \PY{k}{for} \PY{n}{x} \PY{o+ow}{in} \PY{n+nb}{range}\PY{p}{(}\PY{l+m+mi}{40}\PY{p}{,} \PY{l+m+mi}{100}\PY{p}{,} \PY{l+m+mi}{10}\PY{p}{)}\PY{p}{]} \PY{o}{+} \PY{p}{[}\PY{l+s+s1}{\PYZsq{}}\PY{l+s+s1}{2000s}\PY{l+s+s1}{\PYZsq{}}\PY{p}{,} \PY{l+s+s1}{\PYZsq{}}\PY{l+s+s1}{2010s}\PY{l+s+s1}{\PYZsq{}}\PY{p}{]}\PY{p}{)}
\PY{n}{df2}
\end{Verbatim}
\end{tcolorbox}

    \begin{Verbatim}[commandchars=\\\{\}]
<ipython-input-27-8b65ba67fd43>:3: SettingWithCopyWarning:
A value is trying to be set on a copy of a slice from a DataFrame.
Try using .loc[row\_indexer,col\_indexer] = value instead

See the caveats in the documentation: https://pandas.pydata.org/pandas-
docs/stable/user\_guide/indexing.html\#returning-a-view-versus-a-copy
  df2['times'] = pd.cut(df2['year'], bins=range(1940, 2030, 10), \textbackslash{}
    \end{Verbatim}

            \begin{tcolorbox}[breakable, size=fbox, boxrule=.5pt, pad at break*=1mm, opacityfill=0]
\prompt{Out}{outcolor}{27}{\boxspacing}
\begin{Verbatim}[commandchars=\\\{\}]
       movieId                   title     genres  year  times
0            1              Toy Story   adventure  1995    90s
0            1              Toy Story   animation  1995    90s
0            1              Toy Story    children  1995    90s
0            1              Toy Story      comedy  1995    90s
0            1              Toy Story     fantasy  1995    90s
{\ldots}        {\ldots}                     {\ldots}        {\ldots}   {\ldots}    {\ldots}
34204   151701             Bloodmoney              2010  2000s
34205   151703   The Butterfly Circus       drama  2009  2000s
34206   151709                   Zero       drama  2015  2010s
34206   151709                   Zero      sci-fi  2015  2010s
34207   151711  The 2000 Year Old Man              1975    70s

[63231 rows x 5 columns]
\end{Verbatim}
\end{tcolorbox}
        
    \begin{tcolorbox}[breakable, size=fbox, boxrule=1pt, pad at break*=1mm,colback=cellbackground, colframe=cellborder]
\prompt{In}{incolor}{28}{\boxspacing}
\begin{Verbatim}[commandchars=\\\{\}]
\PY{c+c1}{\PYZsh{} jak zmieniała się (wzrastała) liczba filmów w poszczególnych gatunkach}

\PY{n}{df2}\PY{p}{[}\PY{p}{[}\PY{l+s+s1}{\PYZsq{}}\PY{l+s+s1}{movieId}\PY{l+s+s1}{\PYZsq{}}\PY{p}{,} \PY{l+s+s1}{\PYZsq{}}\PY{l+s+s1}{genres}\PY{l+s+s1}{\PYZsq{}}\PY{p}{,} \PY{l+s+s1}{\PYZsq{}}\PY{l+s+s1}{times}\PY{l+s+s1}{\PYZsq{}}\PY{p}{]}\PY{p}{]}
\end{Verbatim}
\end{tcolorbox}

            \begin{tcolorbox}[breakable, size=fbox, boxrule=.5pt, pad at break*=1mm, opacityfill=0]
\prompt{Out}{outcolor}{28}{\boxspacing}
\begin{Verbatim}[commandchars=\\\{\}]
       movieId     genres  times
0            1  adventure    90s
0            1  animation    90s
0            1   children    90s
0            1     comedy    90s
0            1    fantasy    90s
{\ldots}        {\ldots}        {\ldots}    {\ldots}
34204   151701             2000s
34205   151703      drama  2000s
34206   151709      drama  2010s
34206   151709     sci-fi  2010s
34207   151711               70s

[63231 rows x 3 columns]
\end{Verbatim}
\end{tcolorbox}
        
    \begin{tcolorbox}[breakable, size=fbox, boxrule=1pt, pad at break*=1mm,colback=cellbackground, colframe=cellborder]
\prompt{In}{incolor}{29}{\boxspacing}
\begin{Verbatim}[commandchars=\\\{\}]
\PY{c+c1}{\PYZsh{} grupowanie po parach gatunek\PYZhy{}czas zliczając \PYZsq{}movieId\PYZsq{}}
\PY{c+c1}{\PYZsh{} zmiana nazwy \PYZsq{}movieId\PYZsq{} na \PYZsq{}decades\PYZsq{}}
\PY{c+c1}{\PYZsh{} gdy gdzieś none to zastępujemy 0, zmieniamy typ na int}
\PY{c+c1}{\PYZsh{} unstack pozwala rozerwać długi DataFrame i pogrupować indeksy jeden obok drugiego}
\PY{c+c1}{\PYZsh{} jednym indeksem jest \PYZsq{}genres\PYZsq{} filmu, a drugim \PYZsq{}decades\PYZsq{}}

\PY{n}{df3} \PY{o}{=} \PY{n}{df2}\PY{p}{[}\PY{p}{[}\PY{l+s+s1}{\PYZsq{}}\PY{l+s+s1}{movieId}\PY{l+s+s1}{\PYZsq{}}\PY{p}{,} \PY{l+s+s1}{\PYZsq{}}\PY{l+s+s1}{genres}\PY{l+s+s1}{\PYZsq{}}\PY{p}{,} \PY{l+s+s1}{\PYZsq{}}\PY{l+s+s1}{times}\PY{l+s+s1}{\PYZsq{}}\PY{p}{]}\PY{p}{]} \PYZbs{}
    \PY{o}{.}\PY{n}{groupby}\PY{p}{(}\PY{n}{by}\PY{o}{=}\PY{p}{[}\PY{l+s+s1}{\PYZsq{}}\PY{l+s+s1}{genres}\PY{l+s+s1}{\PYZsq{}}\PY{p}{,} \PY{l+s+s1}{\PYZsq{}}\PY{l+s+s1}{times}\PY{l+s+s1}{\PYZsq{}}\PY{p}{]}\PY{p}{)} \PYZbs{}
    \PY{o}{.}\PY{n}{count}\PY{p}{(}\PY{p}{)} \PYZbs{}
    \PY{o}{.}\PY{n}{rename}\PY{p}{(}\PY{n}{columns}\PY{o}{=}\PY{p}{\PYZob{}}\PY{l+s+s1}{\PYZsq{}}\PY{l+s+s1}{movieId}\PY{l+s+s1}{\PYZsq{}}\PY{p}{:} \PY{l+s+s1}{\PYZsq{}}\PY{l+s+s1}{decades}\PY{l+s+s1}{\PYZsq{}}\PY{p}{\PYZcb{}}\PY{p}{)} \PYZbs{}
    \PY{o}{.}\PY{n}{fillna}\PY{p}{(}\PY{l+m+mi}{0}\PY{p}{)} \PYZbs{}
    \PY{o}{.}\PY{n}{astype}\PY{p}{(}\PY{n+nb}{int}\PY{p}{)} \PYZbs{}
    \PY{o}{.}\PY{n}{unstack}\PY{p}{(}\PY{p}{)}
\PY{n}{df3}
\end{Verbatim}
\end{tcolorbox}

            \begin{tcolorbox}[breakable, size=fbox, boxrule=.5pt, pad at break*=1mm, opacityfill=0]
\prompt{Out}{outcolor}{29}{\boxspacing}
\begin{Verbatim}[commandchars=\\\{\}]
            decades
times           40s  50s   60s   70s   80s   90s 2000s 2010s
genres
                 27   45    68   128   106   168   206   293
action           76  100   231   446   531   754  1360   857
adventure        84  166   215   215   332   398   743   458
animation        27   19    45    66   144   221   497   329
children         37   36    78   111   172   344   500   298
comedy          307  339   582   667  1097  1806  2975  1757
crime           176  189   195   348   290   531  1011   520
documentary      27   22    75   126   120   328  1250  1040
drama           650  862  1008  1141  1297  2498  4699  2695
fantasy          47   47    95   105   215   269   555   288
film-noir       168   86     6     6    17    14    18     7
horror           52  117   188   371   448   388  1040   673
imax              0    0     0     0     2    21    70   103
musical         111  115    89    93    93   108   236    79
mystery         132   62   102   191   128   223   584   291
romance         301  296   231   208   378   834  1448   670
sci-fi           14  128   145   177   312   340   560   430
thriller        154  145   197   416   436   849  1847  1158
war             133  156   163   102   120   125   321   130
western          79  160   151   112    35    54    66    38
\end{Verbatim}
\end{tcolorbox}
        
    \begin{tcolorbox}[breakable, size=fbox, boxrule=1pt, pad at break*=1mm,colback=cellbackground, colframe=cellborder]
\prompt{In}{incolor}{30}{\boxspacing}
\begin{Verbatim}[commandchars=\\\{\}]
\PY{c+c1}{\PYZsh{} rozwój gatunków na przestrzeni czasu}
\PY{c+c1}{\PYZsh{} heatmap dla pokazania trendów produkcji w poszczególnych latach}
\PY{c+c1}{\PYZsh{} płótno wykresu przez \PYZsq{}subplots\PYZsq{}, mapa kolorów przez \PYZsq{}cmap\PYZsq{}}
\PY{c+c1}{\PYZsh{} \PYZsq{}annot=True\PYZsq{} dla wyświetleń liczbowych, \PYZsq{}fmt\PYZsq{} dla usunięcia ułamków}
\PY{c+c1}{\PYZsh{} \PYZsq{}linewidth\PYZsq{} dla rozdzielenia linii}

\PY{n}{fig}\PY{p}{,} \PY{n}{ax} \PY{o}{=} \PY{n}{plt}\PY{o}{.}\PY{n}{subplots}\PY{p}{(}\PY{n}{figsize}\PY{o}{=}\PY{p}{(}\PY{l+m+mi}{8}\PY{p}{,}\PY{l+m+mi}{8}\PY{p}{)}\PY{p}{)}
\PY{n}{sns}\PY{o}{.}\PY{n}{heatmap}\PY{p}{(}\PY{n}{df3}\PY{p}{,} \PY{n}{cmap}\PY{o}{=}\PY{l+s+s1}{\PYZsq{}}\PY{l+s+s1}{hot}\PY{l+s+s1}{\PYZsq{}}\PY{p}{,} \PY{n}{annot}\PY{o}{=}\PY{k+kc}{True}\PY{p}{,} \PY{n}{fmt}\PY{o}{=}\PY{l+s+s1}{\PYZsq{}}\PY{l+s+s1}{1}\PY{l+s+s1}{\PYZsq{}}\PY{p}{,} \PY{n}{linewidth}\PY{o}{=}\PY{l+m+mf}{0.5}\PY{p}{)}
\PY{n}{plt}\PY{o}{.}\PY{n}{title}\PY{p}{(}\PY{l+s+s1}{\PYZsq{}}\PY{l+s+s1}{Popularność gatunków w czasie}\PY{l+s+s1}{\PYZsq{}}\PY{p}{)}
\end{Verbatim}
\end{tcolorbox}

            \begin{tcolorbox}[breakable, size=fbox, boxrule=.5pt, pad at break*=1mm, opacityfill=0]
\prompt{Out}{outcolor}{30}{\boxspacing}
\begin{Verbatim}[commandchars=\\\{\}]
Text(0.5, 1.0, 'Popularność gatunków w czasie')
\end{Verbatim}
\end{tcolorbox}
        
    \begin{center}
    \adjustimage{max size={0.9\linewidth}{0.9\paperheight}}{output_34_1.png}
    \end{center}
    { \hspace*{\fill} \\}
    
    \hypertarget{cel-badanie-ocen-poszczeguxf3lnych-filmuxf3w}{%
\paragraph{Cel: badanie ocen poszczególnych
filmów}\label{cel-badanie-ocen-poszczeguxf3lnych-filmuxf3w}}

Potem wnioski z korelacji (zależności) między filmami, a ich ocenami

    \begin{tcolorbox}[breakable, size=fbox, boxrule=1pt, pad at break*=1mm,colback=cellbackground, colframe=cellborder]
\prompt{In}{incolor}{31}{\boxspacing}
\begin{Verbatim}[commandchars=\\\{\}]
\PY{n}{df\PYZus{}ratings}\PY{o}{.}\PY{n}{head}\PY{p}{(}\PY{p}{)}
\end{Verbatim}
\end{tcolorbox}

            \begin{tcolorbox}[breakable, size=fbox, boxrule=.5pt, pad at break*=1mm, opacityfill=0]
\prompt{Out}{outcolor}{31}{\boxspacing}
\begin{Verbatim}[commandchars=\\\{\}]
   movieId  rating           timestamp
0      169     2.5 2008-03-07 22:08:14
1     2471     3.0 2008-03-07 22:03:58
2    48516     5.0 2008-03-07 22:03:55
3     2571     3.5 2015-07-06 06:50:33
4   109487     4.0 2015-07-06 06:51:36
\end{Verbatim}
\end{tcolorbox}
        
    \begin{tcolorbox}[breakable, size=fbox, boxrule=1pt, pad at break*=1mm,colback=cellbackground, colframe=cellborder]
\prompt{In}{incolor}{32}{\boxspacing}
\begin{Verbatim}[commandchars=\\\{\}]
\PY{c+c1}{\PYZsh{} cel: badanie związków między latami produkcji, gatunkami filmów, ich oceną}

\PY{n}{df\PYZus{}ratings}\PY{o}{.}\PY{n}{info}\PY{p}{(}\PY{p}{)}
\end{Verbatim}
\end{tcolorbox}

    \begin{Verbatim}[commandchars=\\\{\}]
<class 'pandas.core.frame.DataFrame'>
RangeIndex: 1009999 entries, 0 to 1009998
Data columns (total 3 columns):
 \#   Column     Non-Null Count    Dtype
---  ------     --------------    -----
 0   movieId    1009999 non-null  int64
 1   rating     1009999 non-null  float64
 2   timestamp  1009999 non-null  datetime64[ns]
dtypes: datetime64[ns](1), float64(1), int64(1)
memory usage: 23.1 MB
    \end{Verbatim}

    \begin{tcolorbox}[breakable, size=fbox, boxrule=1pt, pad at break*=1mm,colback=cellbackground, colframe=cellborder]
\prompt{In}{incolor}{33}{\boxspacing}
\begin{Verbatim}[commandchars=\\\{\}]
\PY{c+c1}{\PYZsh{} widzowie oceniali dodając ilość punktów, gwiazdek}
\PY{c+c1}{\PYZsh{} oceny między 0.5 oraz 5.0}

\PY{n}{df\PYZus{}ratings}\PY{p}{[}\PY{l+s+s1}{\PYZsq{}}\PY{l+s+s1}{rating}\PY{l+s+s1}{\PYZsq{}}\PY{p}{]}\PY{o}{.}\PY{n}{min}\PY{p}{(}\PY{p}{)}
\end{Verbatim}
\end{tcolorbox}

            \begin{tcolorbox}[breakable, size=fbox, boxrule=.5pt, pad at break*=1mm, opacityfill=0]
\prompt{Out}{outcolor}{33}{\boxspacing}
\begin{Verbatim}[commandchars=\\\{\}]
0.5
\end{Verbatim}
\end{tcolorbox}
        
    \begin{tcolorbox}[breakable, size=fbox, boxrule=1pt, pad at break*=1mm,colback=cellbackground, colframe=cellborder]
\prompt{In}{incolor}{34}{\boxspacing}
\begin{Verbatim}[commandchars=\\\{\}]
\PY{n}{df\PYZus{}ratings}\PY{p}{[}\PY{l+s+s1}{\PYZsq{}}\PY{l+s+s1}{rating}\PY{l+s+s1}{\PYZsq{}}\PY{p}{]}\PY{o}{.}\PY{n}{max}\PY{p}{(}\PY{p}{)}
\end{Verbatim}
\end{tcolorbox}

            \begin{tcolorbox}[breakable, size=fbox, boxrule=.5pt, pad at break*=1mm, opacityfill=0]
\prompt{Out}{outcolor}{34}{\boxspacing}
\begin{Verbatim}[commandchars=\\\{\}]
5.0
\end{Verbatim}
\end{tcolorbox}
        
    \begin{tcolorbox}[breakable, size=fbox, boxrule=1pt, pad at break*=1mm,colback=cellbackground, colframe=cellborder]
\prompt{In}{incolor}{35}{\boxspacing}
\begin{Verbatim}[commandchars=\\\{\}]
\PY{c+c1}{\PYZsh{} rozkład nie jest równomierny, a średnia między 3 i 4}

\PY{n}{df\PYZus{}ratings}\PY{p}{[}\PY{l+s+s1}{\PYZsq{}}\PY{l+s+s1}{rating}\PY{l+s+s1}{\PYZsq{}}\PY{p}{]}\PY{o}{.}\PY{n}{plot}\PY{p}{(}\PY{n}{kind}\PY{o}{=}\PY{l+s+s1}{\PYZsq{}}\PY{l+s+s1}{hist}\PY{l+s+s1}{\PYZsq{}}\PY{p}{,} \PY{n}{bins}\PY{o}{=}\PY{l+m+mi}{10}\PY{p}{,} \PY{n}{width}\PY{o}{=}\PY{l+m+mf}{0.3}\PY{p}{)}
\PY{n}{plt}\PY{o}{.}\PY{n}{title}\PY{p}{(}\PY{l+s+s1}{\PYZsq{}}\PY{l+s+s1}{Liczba ocen filmów}\PY{l+s+s1}{\PYZsq{}}\PY{p}{)}
\end{Verbatim}
\end{tcolorbox}

            \begin{tcolorbox}[breakable, size=fbox, boxrule=.5pt, pad at break*=1mm, opacityfill=0]
\prompt{Out}{outcolor}{35}{\boxspacing}
\begin{Verbatim}[commandchars=\\\{\}]
Text(0.5, 1.0, 'Liczba ocen filmów')
\end{Verbatim}
\end{tcolorbox}
        
    \begin{center}
    \adjustimage{max size={0.9\linewidth}{0.9\paperheight}}{output_40_1.png}
    \end{center}
    { \hspace*{\fill} \\}
    
    \begin{tcolorbox}[breakable, size=fbox, boxrule=1pt, pad at break*=1mm,colback=cellbackground, colframe=cellborder]
\prompt{In}{incolor}{36}{\boxspacing}
\begin{Verbatim}[commandchars=\\\{\}]
\PY{c+c1}{\PYZsh{} ilość wystąpień ocen}
\PY{n}{df\PYZus{}ratings}\PY{p}{[}\PY{l+s+s1}{\PYZsq{}}\PY{l+s+s1}{rating}\PY{l+s+s1}{\PYZsq{}}\PY{p}{]}\PY{o}{.}\PY{n}{value\PYZus{}counts}\PY{p}{(}\PY{p}{)}
\end{Verbatim}
\end{tcolorbox}

            \begin{tcolorbox}[breakable, size=fbox, boxrule=.5pt, pad at break*=1mm, opacityfill=0]
\prompt{Out}{outcolor}{36}{\boxspacing}
\begin{Verbatim}[commandchars=\\\{\}]
4.0    271120
3.0    212607
5.0    147001
3.5    113442
4.5     77839
2.0     74139
2.5     47461
1.0     36688
1.5     15259
0.5     14443
Name: rating, dtype: int64
\end{Verbatim}
\end{tcolorbox}
        
    \begin{tcolorbox}[breakable, size=fbox, boxrule=1pt, pad at break*=1mm,colback=cellbackground, colframe=cellborder]
\prompt{In}{incolor}{37}{\boxspacing}
\begin{Verbatim}[commandchars=\\\{\}]
\PY{c+c1}{\PYZsh{} łączymy obie tabele (\PYZsq{}movies\PYZsq{},\PYZsq{}ratings\PYZsq{}) po \PYZsq{}movieId\PYZsq{}}
\PY{c+c1}{\PYZsh{} w tabeli \PYZsq{}ratings\PYZsq{} liczymy średnią, odchylenie standardowe,  liczbę ocen}
\PY{c+c1}{\PYZsh{} wybieramy tylko filmy z ponad 100 głosami widzów}
\PY{c+c1}{\PYZsh{} \PYZsq{}dfq\PYZsq{} zawiera oceny}

\PY{n}{dfq} \PY{o}{=} \PY{n}{df\PYZus{}ratings}\PY{p}{[}\PY{p}{[}\PY{l+s+s1}{\PYZsq{}}\PY{l+s+s1}{movieId}\PY{l+s+s1}{\PYZsq{}}\PY{p}{,} \PY{l+s+s1}{\PYZsq{}}\PY{l+s+s1}{rating}\PY{l+s+s1}{\PYZsq{}}\PY{p}{]}\PY{p}{]} \PYZbs{}
    \PY{o}{.}\PY{n}{groupby}\PY{p}{(}\PY{l+s+s1}{\PYZsq{}}\PY{l+s+s1}{movieId}\PY{l+s+s1}{\PYZsq{}}\PY{p}{)} \PYZbs{}
    \PY{o}{.}\PY{n}{agg}\PY{p}{(}\PY{p}{\PYZob{}}\PY{l+s+s1}{\PYZsq{}}\PY{l+s+s1}{rating}\PY{l+s+s1}{\PYZsq{}}\PY{p}{:} \PY{p}{[}\PY{l+s+s1}{\PYZsq{}}\PY{l+s+s1}{mean}\PY{l+s+s1}{\PYZsq{}}\PY{p}{,} \PY{l+s+s1}{\PYZsq{}}\PY{l+s+s1}{std}\PY{l+s+s1}{\PYZsq{}}\PY{p}{,} \PY{l+s+s1}{\PYZsq{}}\PY{l+s+s1}{count}\PY{l+s+s1}{\PYZsq{}}\PY{p}{]}\PY{p}{\PYZcb{}}\PY{p}{)}
\PY{n}{dfq} \PY{o}{=} \PY{n}{dfq}\PY{p}{[}\PY{n}{dfq}\PY{p}{[}\PY{l+s+s1}{\PYZsq{}}\PY{l+s+s1}{rating}\PY{l+s+s1}{\PYZsq{}}\PY{p}{]}\PY{p}{[}\PY{l+s+s1}{\PYZsq{}}\PY{l+s+s1}{count}\PY{l+s+s1}{\PYZsq{}}\PY{p}{]} \PY{o}{\PYZgt{}} \PY{l+m+mi}{100}\PY{p}{]}
\PY{n}{dfq}
\end{Verbatim}
\end{tcolorbox}

            \begin{tcolorbox}[breakable, size=fbox, boxrule=.5pt, pad at break*=1mm, opacityfill=0]
\prompt{Out}{outcolor}{37}{\boxspacing}
\begin{Verbatim}[commandchars=\\\{\}]
           rating
             mean       std count
movieId
1        3.875427  0.941717  2633
2        3.238537  0.965685  1025
3        3.211712  1.010954   666
4        2.819672  1.066131   122
5        3.088527  0.986589   706
{\ldots}           {\ldots}       {\ldots}   {\ldots}
122882   3.932735  1.031030   223
122886   3.950980  0.866053   102
122892   3.666667  1.021163   114
134130   3.976974  0.830580   152
134853   3.900943  1.079582   212

[2081 rows x 3 columns]
\end{Verbatim}
\end{tcolorbox}
        
    \begin{tcolorbox}[breakable, size=fbox, boxrule=1pt, pad at break*=1mm,colback=cellbackground, colframe=cellborder]
\prompt{In}{incolor}{38}{\boxspacing}
\begin{Verbatim}[commandchars=\\\{\}]
\PY{c+c1}{\PYZsh{} po przycięciu tabeli względem ilości widzów dalsze uproszczenie}
\PY{c+c1}{\PYZsh{} uproszczenie przez wybór tabeli z wartością średnią i odchyleniem standardowym}
\PY{c+c1}{\PYZsh{} funkcja \PYZsq{}concat\PYZsq{} łączy obiekty pandy wzdłuż określonej osi}

\PY{n}{dfq} \PY{o}{=} \PY{n}{pd}\PY{o}{.}\PY{n}{concat}\PY{p}{(}\PY{p}{[}\PY{n}{dfq}\PY{p}{[}\PY{l+s+s1}{\PYZsq{}}\PY{l+s+s1}{rating}\PY{l+s+s1}{\PYZsq{}}\PY{p}{]}\PY{p}{[}\PY{l+s+s1}{\PYZsq{}}\PY{l+s+s1}{mean}\PY{l+s+s1}{\PYZsq{}}\PY{p}{]}\PY{p}{,} \PY{n}{dfq}\PY{p}{[}\PY{l+s+s1}{\PYZsq{}}\PY{l+s+s1}{rating}\PY{l+s+s1}{\PYZsq{}}\PY{p}{]}\PY{p}{[}\PY{l+s+s1}{\PYZsq{}}\PY{l+s+s1}{std}\PY{l+s+s1}{\PYZsq{}}\PY{p}{]}\PY{p}{]}\PY{p}{,} \PY{n}{axis}\PY{o}{=}\PY{l+m+mi}{1}\PY{p}{)}\PY{o}{.}\PY{n}{reset\PYZus{}index}\PY{p}{(}\PY{p}{)}
\PY{n}{dfq}
\end{Verbatim}
\end{tcolorbox}

            \begin{tcolorbox}[breakable, size=fbox, boxrule=.5pt, pad at break*=1mm, opacityfill=0]
\prompt{Out}{outcolor}{38}{\boxspacing}
\begin{Verbatim}[commandchars=\\\{\}]
      movieId      mean       std
0           1  3.875427  0.941717
1           2  3.238537  0.965685
2           3  3.211712  1.010954
3           4  2.819672  1.066131
4           5  3.088527  0.986589
{\ldots}       {\ldots}       {\ldots}       {\ldots}
2076   122882  3.932735  1.031030
2077   122886  3.950980  0.866053
2078   122892  3.666667  1.021163
2079   134130  3.976974  0.830580
2080   134853  3.900943  1.079582

[2081 rows x 3 columns]
\end{Verbatim}
\end{tcolorbox}
        
    \begin{tcolorbox}[breakable, size=fbox, boxrule=1pt, pad at break*=1mm,colback=cellbackground, colframe=cellborder]
\prompt{In}{incolor}{39}{\boxspacing}
\begin{Verbatim}[commandchars=\\\{\}]
\PY{c+c1}{\PYZsh{} histogram histplot z biblioteki seaborn}
\PY{c+c1}{\PYZsh{} rozkład (przegląd) średnich ocen dla wszystkich filmów}

\PY{n}{sns}\PY{o}{.}\PY{n}{histplot}\PY{p}{(}\PY{n}{dfq}\PY{p}{[}\PY{l+s+s1}{\PYZsq{}}\PY{l+s+s1}{mean}\PY{l+s+s1}{\PYZsq{}}\PY{p}{]}\PY{p}{,} \PY{n}{bins}\PY{o}{=}\PY{n}{np}\PY{o}{.}\PY{n}{linspace}\PY{p}{(}\PY{l+m+mi}{0}\PY{p}{,} \PY{l+m+mi}{5}\PY{p}{,} \PY{n}{num}\PY{o}{=}\PY{l+m+mi}{26}\PY{p}{)}\PY{p}{)}
\PY{n}{plt}\PY{o}{.}\PY{n}{title}\PY{p}{(}\PY{l+s+s1}{\PYZsq{}}\PY{l+s+s1}{Rozkład średniej ocen filmów}\PY{l+s+s1}{\PYZsq{}}\PY{p}{)}
\end{Verbatim}
\end{tcolorbox}

            \begin{tcolorbox}[breakable, size=fbox, boxrule=.5pt, pad at break*=1mm, opacityfill=0]
\prompt{Out}{outcolor}{39}{\boxspacing}
\begin{Verbatim}[commandchars=\\\{\}]
Text(0.5, 1.0, 'Rozkład średniej ocen filmów')
\end{Verbatim}
\end{tcolorbox}
        
    \begin{center}
    \adjustimage{max size={0.9\linewidth}{0.9\paperheight}}{output_44_1.png}
    \end{center}
    { \hspace*{\fill} \\}
    
    \begin{tcolorbox}[breakable, size=fbox, boxrule=1pt, pad at break*=1mm,colback=cellbackground, colframe=cellborder]
\prompt{In}{incolor}{40}{\boxspacing}
\begin{Verbatim}[commandchars=\\\{\}]
\PY{c+c1}{\PYZsh{} najgorsze filmy z lewej strony wykresu}
\PY{c+c1}{\PYZsh{} dla nich wartości mniejsze niż w skrajnym kwantylu (pół procent)}
\PY{c+c1}{\PYZsh{} ze związku między mean i std widać, że oceny skrajnie negatywne}

\PY{n}{bad\PYZus{}movies} \PY{o}{=} \PY{n}{dfq}\PY{p}{[}\PY{n}{dfq}\PY{p}{[}\PY{l+s+s1}{\PYZsq{}}\PY{l+s+s1}{mean}\PY{l+s+s1}{\PYZsq{}}\PY{p}{]} \PY{o}{\PYZlt{}} \PY{n}{dfq}\PY{p}{[}\PY{l+s+s1}{\PYZsq{}}\PY{l+s+s1}{mean}\PY{l+s+s1}{\PYZsq{}}\PY{p}{]}\PY{o}{.}\PY{n}{quantile}\PY{p}{(}\PY{l+m+mf}{0.005}\PY{p}{)}\PY{p}{]}
\PY{n}{bad\PYZus{}movies}\PY{o}{.}\PY{n}{head}\PY{p}{(}\PY{p}{)}
\end{Verbatim}
\end{tcolorbox}

            \begin{tcolorbox}[breakable, size=fbox, boxrule=.5pt, pad at break*=1mm, opacityfill=0]
\prompt{Out}{outcolor}{40}{\boxspacing}
\begin{Verbatim}[commandchars=\\\{\}]
     movieId      mean       std
218      393  1.834906  0.994561
642     1381  1.892405  1.155631
684     1556  1.955882  1.091368
737     1707  1.694915  1.025391
752     1760  1.824324  1.164767
\end{Verbatim}
\end{tcolorbox}
        
    \begin{tcolorbox}[breakable, size=fbox, boxrule=1pt, pad at break*=1mm,colback=cellbackground, colframe=cellborder]
\prompt{In}{incolor}{41}{\boxspacing}
\begin{Verbatim}[commandchars=\\\{\}]
\PY{c+c1}{\PYZsh{} dla najlepszych filmów analogicznie}
\PY{c+c1}{\PYZsh{} zmiana nierówności i skrajmych kwantyli}
\PY{c+c1}{\PYZsh{} najlepsze filmy były z prawej strony wykresu}

\PY{n}{good\PYZus{}movies} \PY{o}{=} \PY{n}{dfq}\PY{p}{[}\PY{n}{dfq}\PY{p}{[}\PY{l+s+s1}{\PYZsq{}}\PY{l+s+s1}{mean}\PY{l+s+s1}{\PYZsq{}}\PY{p}{]} \PY{o}{\PYZgt{}} \PY{n}{dfq}\PY{p}{[}\PY{l+s+s1}{\PYZsq{}}\PY{l+s+s1}{mean}\PY{l+s+s1}{\PYZsq{}}\PY{p}{]}\PY{o}{.}\PY{n}{quantile}\PY{p}{(}\PY{l+m+mf}{0.995}\PY{p}{)}\PY{p}{]}
\PY{n}{good\PYZus{}movies}\PY{o}{.}\PY{n}{head}\PY{p}{(}\PY{p}{)}
\end{Verbatim}
\end{tcolorbox}

            \begin{tcolorbox}[breakable, size=fbox, boxrule=.5pt, pad at break*=1mm, opacityfill=0]
\prompt{Out}{outcolor}{41}{\boxspacing}
\begin{Verbatim}[commandchars=\\\{\}]
     movieId      mean       std
39        50  4.322271  0.779590
176      318  4.443080  0.734277
275      527  4.301654  0.838246
350      750  4.226719  0.873880
377      858  4.339376  0.867292
\end{Verbatim}
\end{tcolorbox}
        
    \begin{tcolorbox}[breakable, size=fbox, boxrule=1pt, pad at break*=1mm,colback=cellbackground, colframe=cellborder]
\prompt{In}{incolor}{42}{\boxspacing}
\begin{Verbatim}[commandchars=\\\{\}]
\PY{c+c1}{\PYZsh{} średnia ocen}
\PY{n}{dfq}\PY{p}{[}\PY{l+s+s1}{\PYZsq{}}\PY{l+s+s1}{mean}\PY{l+s+s1}{\PYZsq{}}\PY{p}{]}\PY{o}{.}\PY{n}{mean}\PY{p}{(}\PY{p}{)}
\end{Verbatim}
\end{tcolorbox}

            \begin{tcolorbox}[breakable, size=fbox, boxrule=.5pt, pad at break*=1mm, opacityfill=0]
\prompt{Out}{outcolor}{42}{\boxspacing}
\begin{Verbatim}[commandchars=\\\{\}]
3.454050593287785
\end{Verbatim}
\end{tcolorbox}
        
    \hypertarget{najgorsze-filmy}{%
\paragraph{Najgorsze filmy}\label{najgorsze-filmy}}

    \begin{tcolorbox}[breakable, size=fbox, boxrule=1pt, pad at break*=1mm,colback=cellbackground, colframe=cellborder]
\prompt{In}{incolor}{43}{\boxspacing}
\begin{Verbatim}[commandchars=\\\{\}]
\PY{c+c1}{\PYZsh{} poniżej lista najgorzej ocenianych filmów}
\PY{c+c1}{\PYZsh{} sortowanie względem średniej i w sposób rosnący}
\PY{c+c1}{\PYZsh{} przez \PYZsq{}merge\PYZsq{} dobieramy (scalamy) do \PYZsq{}bad\PYZus{}movies\PYZsq{} na indeksie \PYZsq{}movieId\PYZsq{}}

\PY{c+c1}{\PYZsh{} najgorszym był \PYZdq{}Battlefield Earth\PYZdq{}}

\PY{n}{dfm} \PY{o}{=} \PY{n}{df\PYZus{}movies}\PY{p}{[}\PY{p}{[}\PY{l+s+s1}{\PYZsq{}}\PY{l+s+s1}{movieId}\PY{l+s+s1}{\PYZsq{}}\PY{p}{,} \PY{l+s+s1}{\PYZsq{}}\PY{l+s+s1}{title}\PY{l+s+s1}{\PYZsq{}}\PY{p}{,} \PY{l+s+s1}{\PYZsq{}}\PY{l+s+s1}{genres}\PY{l+s+s1}{\PYZsq{}}\PY{p}{,} \PY{l+s+s1}{\PYZsq{}}\PY{l+s+s1}{year}\PY{l+s+s1}{\PYZsq{}}\PY{p}{]}\PY{p}{]}

\PY{n}{dfm}\PY{o}{.}\PY{n}{merge}\PY{p}{(}\PY{n}{bad\PYZus{}movies}\PY{p}{,} \PY{n}{on}\PY{o}{=}\PY{l+s+s1}{\PYZsq{}}\PY{l+s+s1}{movieId}\PY{l+s+s1}{\PYZsq{}}\PY{p}{)} \PYZbs{}
    \PY{o}{.}\PY{n}{sort\PYZus{}values}\PY{p}{(}\PY{n}{by}\PY{o}{=}\PY{l+s+s1}{\PYZsq{}}\PY{l+s+s1}{mean}\PY{l+s+s1}{\PYZsq{}}\PY{p}{,} \PY{n}{ascending}\PY{o}{=}\PY{k+kc}{True}\PY{p}{)} \PYZbs{}
    \PY{o}{.}\PY{n}{head}\PY{p}{(}\PY{p}{)}
\end{Verbatim}
\end{tcolorbox}

            \begin{tcolorbox}[breakable, size=fbox, boxrule=.5pt, pad at break*=1mm, opacityfill=0]
\prompt{Out}{outcolor}{43}{\boxspacing}
\begin{Verbatim}[commandchars=\\\{\}]
   movieId                                title                        genres  \textbackslash{}
9     3593                   Battlefield Earth               [action, sci-fi]
3     1707                        Home Alone 3             [children, comedy]
6     2383  Police Academy 6: City Under Siege                [comedy, crime]
4     1760                         Spice World                       [comedy]
0      393                      Street Fighter   [action, adventure, fantasy]

   year      mean       std
9  2000  1.601093  1.046658
3  1997  1.694915  1.025391
6  1989  1.769231  1.148919
4  1997  1.824324  1.164767
0  1994  1.834906  0.994561
\end{Verbatim}
\end{tcolorbox}
        
    \hypertarget{najlepsze-filmy}{%
\paragraph{Najlepsze filmy}\label{najlepsze-filmy}}

    \begin{tcolorbox}[breakable, size=fbox, boxrule=1pt, pad at break*=1mm,colback=cellbackground, colframe=cellborder]
\prompt{In}{incolor}{44}{\boxspacing}
\begin{Verbatim}[commandchars=\\\{\}]
\PY{c+c1}{\PYZsh{} poniżej lista najwyżej ocenianych filmów}
\PY{c+c1}{\PYZsh{} sortowanie względem średniej i w sposób malejący}

\PY{c+c1}{\PYZsh{} najlepszym był \PYZdq{}The Shawshank Redemption\PYZdq{}}

\PY{n}{dfm}\PY{o}{.}\PY{n}{merge}\PY{p}{(}\PY{n}{good\PYZus{}movies}\PY{p}{,} \PY{n}{on}\PY{o}{=}\PY{l+s+s1}{\PYZsq{}}\PY{l+s+s1}{movieId}\PY{l+s+s1}{\PYZsq{}}\PY{p}{)} \PYZbs{}
    \PY{o}{.}\PY{n}{sort\PYZus{}values}\PY{p}{(}\PY{n}{by}\PY{o}{=}\PY{l+s+s1}{\PYZsq{}}\PY{l+s+s1}{mean}\PY{l+s+s1}{\PYZsq{}}\PY{p}{,} \PY{n}{ascending}\PY{o}{=}\PY{k+kc}{False}\PY{p}{)} \PYZbs{}
    \PY{o}{.}\PY{n}{head}\PY{p}{(}\PY{p}{)}
\end{Verbatim}
\end{tcolorbox}

            \begin{tcolorbox}[breakable, size=fbox, boxrule=.5pt, pad at break*=1mm, opacityfill=0]
\prompt{Out}{outcolor}{44}{\boxspacing}
\begin{Verbatim}[commandchars=\\\{\}]
   movieId                       title                      genres  year  \textbackslash{}
1      318  Shawshank Redemption, The               [crime, drama]  1994
4      858             Godfather, The               [crime, drama]  1972
0       50        Usual Suspects, The   [crime, mystery, thriller]  1995
2      527           Schindler's List                 [drama, war]  1993
6     1178             Paths of Glory                 [drama, war]  1957

       mean       std
1  4.443080  0.734277
4  4.339376  0.867292
0  4.322271  0.779590
2  4.301654  0.838246
6  4.294286  0.802633
\end{Verbatim}
\end{tcolorbox}
        
    \hypertarget{najbardziej-popularne-filmy}{%
\paragraph{Najbardziej popularne
filmy}\label{najbardziej-popularne-filmy}}

    \begin{tcolorbox}[breakable, size=fbox, boxrule=1pt, pad at break*=1mm,colback=cellbackground, colframe=cellborder]
\prompt{In}{incolor}{45}{\boxspacing}
\begin{Verbatim}[commandchars=\\\{\}]
\PY{c+c1}{\PYZsh{} lista głosów najpopularniejszych filmów}

\PY{n}{df\PYZus{}ratings}\PY{p}{[}\PY{p}{[}\PY{l+s+s1}{\PYZsq{}}\PY{l+s+s1}{movieId}\PY{l+s+s1}{\PYZsq{}}\PY{p}{,} \PY{l+s+s1}{\PYZsq{}}\PY{l+s+s1}{rating}\PY{l+s+s1}{\PYZsq{}}\PY{p}{]}\PY{p}{]} \PYZbs{}
    \PY{o}{.}\PY{n}{groupby}\PY{p}{(}\PY{l+s+s1}{\PYZsq{}}\PY{l+s+s1}{movieId}\PY{l+s+s1}{\PYZsq{}}\PY{p}{)} \PYZbs{}
    \PY{o}{.}\PY{n}{count}\PY{p}{(}\PY{p}{)} \PYZbs{}
    \PY{o}{.}\PY{n}{sort\PYZus{}values}\PY{p}{(}\PY{n}{by}\PY{o}{=}\PY{l+s+s1}{\PYZsq{}}\PY{l+s+s1}{rating}\PY{l+s+s1}{\PYZsq{}}\PY{p}{,} \PY{n}{ascending}\PY{o}{=}\PY{k+kc}{False}\PY{p}{)} \PYZbs{}
    \PY{o}{.}\PY{n}{rename}\PY{p}{(}\PY{n}{columns}\PY{o}{=}\PY{p}{\PYZob{}}\PY{l+s+s1}{\PYZsq{}}\PY{l+s+s1}{rating}\PY{l+s+s1}{\PYZsq{}}\PY{p}{:} \PY{l+s+s1}{\PYZsq{}}\PY{l+s+s1}{\PYZsh{}votes}\PY{l+s+s1}{\PYZsq{}}\PY{p}{\PYZcb{}}\PY{p}{)} \PYZbs{}
    \PY{o}{.}\PY{n}{head}\PY{p}{(}\PY{p}{)}
\end{Verbatim}
\end{tcolorbox}

            \begin{tcolorbox}[breakable, size=fbox, boxrule=.5pt, pad at break*=1mm, opacityfill=0]
\prompt{Out}{outcolor}{45}{\boxspacing}
\begin{Verbatim}[commandchars=\\\{\}]
         \#votes
movieId
356        3504
296        3394
318        3338
593        3331
480        3010
\end{Verbatim}
\end{tcolorbox}
        
    \begin{tcolorbox}[breakable, size=fbox, boxrule=1pt, pad at break*=1mm,colback=cellbackground, colframe=cellborder]
\prompt{In}{incolor}{46}{\boxspacing}
\begin{Verbatim}[commandchars=\\\{\}]
\PY{c+c1}{\PYZsh{} za pomocą \PYZsq{}merge\PYZsq{} można dodać \PYZsq{}dfm\PYZsq{} (plik tytułów filmów)}

\PY{c+c1}{\PYZsh{} najpopularniejszym był \PYZdq{}Forrest Gump\PYZdq{}}

\PY{n}{df\PYZus{}ratings}\PY{p}{[}\PY{p}{[}\PY{l+s+s1}{\PYZsq{}}\PY{l+s+s1}{movieId}\PY{l+s+s1}{\PYZsq{}}\PY{p}{,} \PY{l+s+s1}{\PYZsq{}}\PY{l+s+s1}{rating}\PY{l+s+s1}{\PYZsq{}}\PY{p}{]}\PY{p}{]} \PYZbs{}
    \PY{o}{.}\PY{n}{groupby}\PY{p}{(}\PY{l+s+s1}{\PYZsq{}}\PY{l+s+s1}{movieId}\PY{l+s+s1}{\PYZsq{}}\PY{p}{)} \PYZbs{}
    \PY{o}{.}\PY{n}{count}\PY{p}{(}\PY{p}{)} \PYZbs{}
    \PY{o}{.}\PY{n}{sort\PYZus{}values}\PY{p}{(}\PY{n}{by}\PY{o}{=}\PY{l+s+s1}{\PYZsq{}}\PY{l+s+s1}{rating}\PY{l+s+s1}{\PYZsq{}}\PY{p}{,} \PY{n}{ascending}\PY{o}{=}\PY{k+kc}{False}\PY{p}{)} \PYZbs{}
    \PY{o}{.}\PY{n}{rename}\PY{p}{(}\PY{n}{columns}\PY{o}{=}\PY{p}{\PYZob{}}\PY{l+s+s1}{\PYZsq{}}\PY{l+s+s1}{rating}\PY{l+s+s1}{\PYZsq{}}\PY{p}{:} \PY{l+s+s1}{\PYZsq{}}\PY{l+s+s1}{\PYZsh{}votes}\PY{l+s+s1}{\PYZsq{}}\PY{p}{\PYZcb{}}\PY{p}{)} \PYZbs{}
    \PY{o}{.}\PY{n}{merge}\PY{p}{(}\PY{n}{dfm}\PY{p}{,} \PY{n}{left\PYZus{}index}\PY{o}{=}\PY{k+kc}{True}\PY{p}{,} \PY{n}{right\PYZus{}on}\PY{o}{=}\PY{l+s+s1}{\PYZsq{}}\PY{l+s+s1}{movieId}\PY{l+s+s1}{\PYZsq{}}\PY{p}{,} \PY{n}{how}\PY{o}{=}\PY{l+s+s1}{\PYZsq{}}\PY{l+s+s1}{left}\PY{l+s+s1}{\PYZsq{}}\PY{p}{)} \PYZbs{}
    \PY{o}{.}\PY{n}{head}\PY{p}{(}\PY{p}{)}
\end{Verbatim}
\end{tcolorbox}

            \begin{tcolorbox}[breakable, size=fbox, boxrule=.5pt, pad at break*=1mm, opacityfill=0]
\prompt{Out}{outcolor}{46}{\boxspacing}
\begin{Verbatim}[commandchars=\\\{\}]
     \#votes  movieId                       title  \textbackslash{}
352    3504      356               Forrest Gump
293    3394      296               Pulp Fiction
315    3338      318  Shawshank Redemption, The
587    3331      593  Silence of the Lambs, The
476    3010      480              Jurassic Park

                                    genres  year
352          [comedy, drama, romance, war]  1994
293       [comedy, crime, drama, thriller]  1994
315                         [crime, drama]  1994
587              [crime, horror, thriller]  1991
476  [action, adventure, sci-fi, thriller]  1993
\end{Verbatim}
\end{tcolorbox}
        
    \hypertarget{ocena-a-popularnoux15bux107}{%
\paragraph{Ocena a popularność}\label{ocena-a-popularnoux15bux107}}

    \begin{tcolorbox}[breakable, size=fbox, boxrule=1pt, pad at break*=1mm,colback=cellbackground, colframe=cellborder]
\prompt{In}{incolor}{47}{\boxspacing}
\begin{Verbatim}[commandchars=\\\{\}]
\PY{c+c1}{\PYZsh{} nie były to najwyżej oceniane filmy lecz z największą liczbą głosów}
\PY{c+c1}{\PYZsh{} nie wiemy czy najlepiej oceniane te najbardziej popularne}

\PY{c+c1}{\PYZsh{} do \PYZsq{}dfm\PYZsq{} (tytuły) za pomocą \PYZsq{}merge\PYZsq{} można też dodać \PYZsq{}dfq\PYZsq{} (plik z ocenami)}
\PY{c+c1}{\PYZsh{} sortowanie przez \PYZsq{}std\PYZsq{} daje pojęcie o spójności ocen}

\PY{c+c1}{\PYZsh{} czyli jednym się podobało, innym nie, ale ogólnie dobre oceny, popularność}

\PY{n}{dfc} \PY{o}{=} \PY{n}{df\PYZus{}ratings}\PY{p}{[}\PY{p}{[}\PY{l+s+s1}{\PYZsq{}}\PY{l+s+s1}{movieId}\PY{l+s+s1}{\PYZsq{}}\PY{p}{,} \PY{l+s+s1}{\PYZsq{}}\PY{l+s+s1}{rating}\PY{l+s+s1}{\PYZsq{}}\PY{p}{]}\PY{p}{]} \PYZbs{}
    \PY{o}{.}\PY{n}{groupby}\PY{p}{(}\PY{l+s+s1}{\PYZsq{}}\PY{l+s+s1}{movieId}\PY{l+s+s1}{\PYZsq{}}\PY{p}{)} \PYZbs{}
    \PY{o}{.}\PY{n}{count}\PY{p}{(}\PY{p}{)} \PYZbs{}
    \PY{o}{.}\PY{n}{sort\PYZus{}values}\PY{p}{(}\PY{n}{by}\PY{o}{=}\PY{l+s+s1}{\PYZsq{}}\PY{l+s+s1}{rating}\PY{l+s+s1}{\PYZsq{}}\PY{p}{,} \PY{n}{ascending}\PY{o}{=}\PY{k+kc}{False}\PY{p}{)} \PYZbs{}
    \PY{o}{.}\PY{n}{rename}\PY{p}{(}\PY{n}{columns}\PY{o}{=}\PY{p}{\PYZob{}}\PY{l+s+s1}{\PYZsq{}}\PY{l+s+s1}{rating}\PY{l+s+s1}{\PYZsq{}}\PY{p}{:} \PY{l+s+s1}{\PYZsq{}}\PY{l+s+s1}{\PYZsh{}votes}\PY{l+s+s1}{\PYZsq{}}\PY{p}{\PYZcb{}}\PY{p}{)} \PYZbs{}
    \PY{o}{.}\PY{n}{merge}\PY{p}{(}\PY{n}{dfm}\PY{p}{,} \PY{n}{left\PYZus{}index}\PY{o}{=}\PY{k+kc}{True}\PY{p}{,} \PY{n}{right\PYZus{}on}\PY{o}{=}\PY{l+s+s1}{\PYZsq{}}\PY{l+s+s1}{movieId}\PY{l+s+s1}{\PYZsq{}}\PY{p}{,} \PY{n}{how}\PY{o}{=}\PY{l+s+s1}{\PYZsq{}}\PY{l+s+s1}{left}\PY{l+s+s1}{\PYZsq{}}\PY{p}{)} \PYZbs{}
    \PY{o}{.}\PY{n}{merge}\PY{p}{(}\PY{n}{dfq}\PY{p}{,} \PY{n}{on}\PY{o}{=}\PY{l+s+s1}{\PYZsq{}}\PY{l+s+s1}{movieId}\PY{l+s+s1}{\PYZsq{}}\PY{p}{,} \PY{n}{how}\PY{o}{=}\PY{l+s+s1}{\PYZsq{}}\PY{l+s+s1}{inner}\PY{l+s+s1}{\PYZsq{}}\PY{p}{)} \PYZbs{}
    \PY{o}{.}\PY{n}{sort\PYZus{}values}\PY{p}{(}\PY{n}{by}\PY{o}{=}\PY{l+s+s1}{\PYZsq{}}\PY{l+s+s1}{std}\PY{l+s+s1}{\PYZsq{}}\PY{p}{,} \PY{n}{ascending}\PY{o}{=}\PY{k+kc}{False}\PY{p}{)}
\PY{n}{dfc}
\end{Verbatim}
\end{tcolorbox}

            \begin{tcolorbox}[breakable, size=fbox, boxrule=.5pt, pad at break*=1mm, opacityfill=0]
\prompt{Out}{outcolor}{47}{\boxspacing}
\begin{Verbatim}[commandchars=\\\{\}]
      \#votes  movieId                                              title  \textbackslash{}
2077     101     1924                           Plan 9 from Outer Space
1604     146     2459                      Texas Chainsaw Massacre, The
1455     166    63992                                          Twilight
1557     152     1483                                             Crash
2045     103      891  Halloween: The Curse of Michael Myers (Hallowe{\ldots}
{\ldots}      {\ldots}      {\ldots}                                                {\ldots}
1221     208    48738                        Last King of Scotland, The
2061     102      259                                     Kiss of Death
2058     102   104879                                         Prisoners
1676     139    81932                                      Fighter, The
1834     121     8340                              Escape from Alcatraz

                                   genres  year      mean       std
2077                     [horror, sci-fi]  1959  2.658416  1.419385
1604                             [horror]  1974  2.965753  1.394147
1455  [drama, fantasy, romance, thriller]  2008  2.436747  1.363120
1557                    [drama, thriller]  1996  2.904605  1.354088
2045                   [horror, thriller]  1995  2.616505  1.321400
{\ldots}                                   {\ldots}   {\ldots}       {\ldots}       {\ldots}
1221                    [drama, thriller]  2006  3.855769  0.643335
2061             [crime, drama, thriller]  1995  3.014706  0.638903
2058           [drama, mystery, thriller]  2013  3.960784  0.635915
1676                              [drama]  2010  3.920863  0.629179
1834                    [drama, thriller]  1979  3.871901  0.600447

[2081 rows x 7 columns]
\end{Verbatim}
\end{tcolorbox}
        
    \begin{tcolorbox}[breakable, size=fbox, boxrule=1pt, pad at break*=1mm,colback=cellbackground, colframe=cellborder]
\prompt{In}{incolor}{48}{\boxspacing}
\begin{Verbatim}[commandchars=\\\{\}]
\PY{c+c1}{\PYZsh{} średnia ocen}
\PY{n}{dfq}\PY{p}{[}\PY{l+s+s1}{\PYZsq{}}\PY{l+s+s1}{mean}\PY{l+s+s1}{\PYZsq{}}\PY{p}{]}\PY{o}{.}\PY{n}{mean}\PY{p}{(}\PY{p}{)}
\end{Verbatim}
\end{tcolorbox}

            \begin{tcolorbox}[breakable, size=fbox, boxrule=.5pt, pad at break*=1mm, opacityfill=0]
\prompt{Out}{outcolor}{48}{\boxspacing}
\begin{Verbatim}[commandchars=\\\{\}]
3.454050593287785
\end{Verbatim}
\end{tcolorbox}
        
    \hypertarget{poruxf3wnanie-2-najpopularniejszych-gatunkuxf3w}{%
\paragraph{Porównanie 2 najpopularniejszych
gatunków}\label{poruxf3wnanie-2-najpopularniejszych-gatunkuxf3w}}

    \begin{tcolorbox}[breakable, size=fbox, boxrule=1pt, pad at break*=1mm,colback=cellbackground, colframe=cellborder]
\prompt{In}{incolor}{49}{\boxspacing}
\begin{Verbatim}[commandchars=\\\{\}]
\PY{c+c1}{\PYZsh{} kopiowanie wartości do nowego DataFrame, aby wartości się nie nadpisywały}
\PY{c+c1}{\PYZsh{} przez apply zmieniamy zbiory na listy}
\PY{c+c1}{\PYZsh{} eksplodowanie tych gatunków, by ustawiły się jako rzędy jeden pod drugim}

\PY{n}{dfg} \PY{o}{=} \PY{n}{dfc}\PY{p}{[}\PY{p}{[}\PY{l+s+s1}{\PYZsq{}}\PY{l+s+s1}{\PYZsh{}votes}\PY{l+s+s1}{\PYZsq{}}\PY{p}{,} \PY{l+s+s1}{\PYZsq{}}\PY{l+s+s1}{mean}\PY{l+s+s1}{\PYZsq{}}\PY{p}{,} \PY{l+s+s1}{\PYZsq{}}\PY{l+s+s1}{std}\PY{l+s+s1}{\PYZsq{}}\PY{p}{,} \PY{l+s+s1}{\PYZsq{}}\PY{l+s+s1}{genres}\PY{l+s+s1}{\PYZsq{}}\PY{p}{]}\PY{p}{]}\PY{o}{.}\PY{n}{copy}\PY{p}{(}\PY{p}{)}
\PY{n}{dfg}\PY{p}{[}\PY{l+s+s1}{\PYZsq{}}\PY{l+s+s1}{genres}\PY{l+s+s1}{\PYZsq{}}\PY{p}{]} \PY{o}{=} \PY{n}{dfg}\PY{p}{[}\PY{l+s+s1}{\PYZsq{}}\PY{l+s+s1}{genres}\PY{l+s+s1}{\PYZsq{}}\PY{p}{]}\PY{o}{.}\PY{n}{apply}\PY{p}{(}\PY{k}{lambda} \PY{n}{x}\PY{p}{:} \PY{n+nb}{list}\PY{p}{(}\PY{n}{x}\PY{p}{)}\PY{p}{)}
\PY{n}{dfg} \PY{o}{=} \PY{n}{dfg}\PY{o}{.}\PY{n}{explode}\PY{p}{(}\PY{l+s+s1}{\PYZsq{}}\PY{l+s+s1}{genres}\PY{l+s+s1}{\PYZsq{}}\PY{p}{)}
\PY{n}{dfg}
\end{Verbatim}
\end{tcolorbox}

            \begin{tcolorbox}[breakable, size=fbox, boxrule=.5pt, pad at break*=1mm, opacityfill=0]
\prompt{Out}{outcolor}{49}{\boxspacing}
\begin{Verbatim}[commandchars=\\\{\}]
      \#votes      mean       std    genres
2077     101  2.658416  1.419385    horror
2077     101  2.658416  1.419385    sci-fi
1604     146  2.965753  1.394147    horror
1455     166  2.436747  1.363120     drama
1455     166  2.436747  1.363120   fantasy
{\ldots}      {\ldots}       {\ldots}       {\ldots}       {\ldots}
2058     102  3.960784  0.635915   mystery
2058     102  3.960784  0.635915  thriller
1676     139  3.920863  0.629179     drama
1834     121  3.871901  0.600447     drama
1834     121  3.871901  0.600447  thriller

[5520 rows x 4 columns]
\end{Verbatim}
\end{tcolorbox}
        
    \begin{tcolorbox}[breakable, size=fbox, boxrule=1pt, pad at break*=1mm,colback=cellbackground, colframe=cellborder]
\prompt{In}{incolor}{50}{\boxspacing}
\begin{Verbatim}[commandchars=\\\{\}]
\PY{c+c1}{\PYZsh{} porównanie 2 najpopularniejszych gatunków: \PYZsq{}drama\PYZsq{}, \PYZsq{}comedy\PYZsq{}}

\PY{n}{dfg}\PY{p}{[}\PY{n}{dfg}\PY{p}{[}\PY{l+s+s1}{\PYZsq{}}\PY{l+s+s1}{genres}\PY{l+s+s1}{\PYZsq{}}\PY{p}{]}\PY{o}{.}\PY{n}{isin}\PY{p}{(}\PY{p}{[}\PY{l+s+s1}{\PYZsq{}}\PY{l+s+s1}{drama}\PY{l+s+s1}{\PYZsq{}}\PY{p}{,} \PY{l+s+s1}{\PYZsq{}}\PY{l+s+s1}{comedy}\PY{l+s+s1}{\PYZsq{}}\PY{p}{]}\PY{p}{)}\PY{p}{]}
\end{Verbatim}
\end{tcolorbox}

            \begin{tcolorbox}[breakable, size=fbox, boxrule=.5pt, pad at break*=1mm, opacityfill=0]
\prompt{Out}{outcolor}{50}{\boxspacing}
\begin{Verbatim}[commandchars=\\\{\}]
      \#votes      mean       std  genres
1455     166  2.436747  1.363120   drama
1557     152  2.904605  1.354088   drama
1027     248  2.473790  1.301011  comedy
900      284  3.149648  1.300163   drama
1358     181  3.204420  1.289781   drama
{\ldots}      {\ldots}       {\ldots}       {\ldots}     {\ldots}
1221     208  3.855769  0.643335   drama
2061     102  3.014706  0.638903   drama
2058     102  3.960784  0.635915   drama
1676     139  3.920863  0.629179   drama
1834     121  3.871901  0.600447   drama

[1747 rows x 4 columns]
\end{Verbatim}
\end{tcolorbox}
        
    \begin{tcolorbox}[breakable, size=fbox, boxrule=1pt, pad at break*=1mm,colback=cellbackground, colframe=cellborder]
\prompt{In}{incolor}{51}{\boxspacing}
\begin{Verbatim}[commandchars=\\\{\}]
\PY{c+c1}{\PYZsh{} można pokazać na \PYZsq{}pairplot\PYZsq{}}
\PY{c+c1}{\PYZsh{} parametr \PYZsq{}hue\PYZsq{} podzeli ten wykres na 2 podwykresy względem gatunku}

\PY{c+c1}{\PYZsh{} komedie i dramaty tak samo popularne}
\PY{c+c1}{\PYZsh{} jednak ocena dramatu statystycznie wyższa niż komedii}
\PY{c+c1}{\PYZsh{} oceniający też bardziej spójni przy dramatach (mniejsze std) niż komediach}
\PY{c+c1}{\PYZsh{} wg widzów lepiej obejrzeć dobry dramat niż średnią komedię}

\PY{n}{sns}\PY{o}{.}\PY{n}{pairplot}\PY{p}{(}\PY{n}{dfg}\PY{p}{[}\PY{n}{dfg}\PY{p}{[}\PY{l+s+s1}{\PYZsq{}}\PY{l+s+s1}{genres}\PY{l+s+s1}{\PYZsq{}}\PY{p}{]}\PY{o}{.}\PY{n}{isin}\PY{p}{(}\PY{p}{[}\PY{l+s+s1}{\PYZsq{}}\PY{l+s+s1}{drama}\PY{l+s+s1}{\PYZsq{}}\PY{p}{,} \PY{l+s+s1}{\PYZsq{}}\PY{l+s+s1}{comedy}\PY{l+s+s1}{\PYZsq{}}\PY{p}{]}\PY{p}{)}\PY{p}{]}\PY{p}{,} \PY{n}{hue}\PY{o}{=}\PY{l+s+s1}{\PYZsq{}}\PY{l+s+s1}{genres}\PY{l+s+s1}{\PYZsq{}}\PY{p}{)}
\end{Verbatim}
\end{tcolorbox}

            \begin{tcolorbox}[breakable, size=fbox, boxrule=.5pt, pad at break*=1mm, opacityfill=0]
\prompt{Out}{outcolor}{51}{\boxspacing}
\begin{Verbatim}[commandchars=\\\{\}]
<seaborn.axisgrid.PairGrid at 0x20199f7c100>
\end{Verbatim}
\end{tcolorbox}
        
    \begin{center}
    \adjustimage{max size={0.9\linewidth}{0.9\paperheight}}{output_61_1.png}
    \end{center}
    { \hspace*{\fill} \\}
    
    \hypertarget{wnioski}{%
\paragraph{Wnioski:}\label{wnioski}}

\begin{enumerate}
\def\labelenumi{\arabic{enumi}.}
\tightlist
\item
  czyli niekoniecznie wysoka popularność jest zgodna z oceną
\item
  widzowie mogą też zwracać uwagę na różne elementy
\item
  np. widzowie mają ulubione gatunki filmów, a unikają innych
\end{enumerate}

    \begin{tcolorbox}[breakable, size=fbox, boxrule=1pt, pad at break*=1mm,colback=cellbackground, colframe=cellborder]
\prompt{In}{incolor}{52}{\boxspacing}
\begin{Verbatim}[commandchars=\\\{\}]
\PY{c+c1}{\PYZsh{} jak zmieniały się oceny filmów w miesięcznych przedziałach czasu}

\PY{c+c1}{\PYZsh{} rozkład głosów względem czasu za pomocą funkcji mean}
\PY{c+c1}{\PYZsh{} średnia między 3, a 4}

\PY{n}{df\PYZus{}ratings}\PY{p}{[}\PY{p}{[}\PY{l+s+s1}{\PYZsq{}}\PY{l+s+s1}{rating}\PY{l+s+s1}{\PYZsq{}}\PY{p}{,} \PY{l+s+s1}{\PYZsq{}}\PY{l+s+s1}{timestamp}\PY{l+s+s1}{\PYZsq{}}\PY{p}{]}\PY{p}{]}\PY{o}{.}\PY{n}{resample}\PY{p}{(}\PY{l+s+s1}{\PYZsq{}}\PY{l+s+s1}{M}\PY{l+s+s1}{\PYZsq{}}\PY{p}{,} \PY{n}{on}\PY{o}{=}\PY{l+s+s1}{\PYZsq{}}\PY{l+s+s1}{timestamp}\PY{l+s+s1}{\PYZsq{}}\PY{p}{)}\PY{o}{.}\PY{n}{mean}\PY{p}{(}\PY{p}{)}\PY{o}{.}\PY{n}{plot}\PY{p}{(}\PY{n}{figsize}\PY{o}{=}\PY{p}{(}\PY{l+m+mi}{14}\PY{p}{,} \PY{l+m+mi}{4}\PY{p}{)}\PY{p}{)}
\PY{n}{plt}\PY{o}{.}\PY{n}{title}\PY{p}{(}\PY{l+s+s1}{\PYZsq{}}\PY{l+s+s1}{Oceny filmów w czasie}\PY{l+s+s1}{\PYZsq{}}\PY{p}{)}
\end{Verbatim}
\end{tcolorbox}

            \begin{tcolorbox}[breakable, size=fbox, boxrule=.5pt, pad at break*=1mm, opacityfill=0]
\prompt{Out}{outcolor}{52}{\boxspacing}
\begin{Verbatim}[commandchars=\\\{\}]
Text(0.5, 1.0, 'Oceny filmów w czasie')
\end{Verbatim}
\end{tcolorbox}
        
    \begin{center}
    \adjustimage{max size={0.9\linewidth}{0.9\paperheight}}{output_63_1.png}
    \end{center}
    { \hspace*{\fill} \\}
    
    \begin{tcolorbox}[breakable, size=fbox, boxrule=1pt, pad at break*=1mm,colback=cellbackground, colframe=cellborder]
\prompt{In}{incolor}{53}{\boxspacing}
\begin{Verbatim}[commandchars=\\\{\}]
\PY{c+c1}{\PYZsh{} jak zmieniała się  popularność platformy w przedziałach czasu}

\PY{c+c1}{\PYZsh{} \PYZsq{}count\PYZsq{} zlicza głosy w tych przedziałach}
\PY{c+c1}{\PYZsh{} mamy film i czas, w którym była ocena}
\PY{c+c1}{\PYZsh{} funkcja \PYZsq{}resample\PYZsq{} działa podobnie jak \PYZsq{}groupby\PYZsq{}}
\PY{c+c1}{\PYZsh{} agreguje dane w czasie (\PYZsq{}M\PYZsq{} to miesiąc, \PYZsq{}timestamp\PYZsq{} to funkcja agregowana)}

\PY{n}{df\PYZus{}ratings}\PY{p}{[}\PY{p}{[}\PY{l+s+s1}{\PYZsq{}}\PY{l+s+s1}{movieId}\PY{l+s+s1}{\PYZsq{}}\PY{p}{,} \PY{l+s+s1}{\PYZsq{}}\PY{l+s+s1}{timestamp}\PY{l+s+s1}{\PYZsq{}}\PY{p}{]}\PY{p}{]}\PY{o}{.}\PY{n}{resample}\PY{p}{(}\PY{l+s+s1}{\PYZsq{}}\PY{l+s+s1}{M}\PY{l+s+s1}{\PYZsq{}}\PY{p}{,} \PY{n}{on}\PY{o}{=}\PY{l+s+s1}{\PYZsq{}}\PY{l+s+s1}{timestamp}\PY{l+s+s1}{\PYZsq{}}\PY{p}{)}\PY{o}{.}\PY{n}{count}\PY{p}{(}\PY{p}{)}\PY{o}{.}\PY{n}{plot}\PY{p}{(}\PY{n}{figsize}\PY{o}{=}\PY{p}{(}\PY{l+m+mi}{14}\PY{p}{,} \PY{l+m+mi}{4}\PY{p}{)}\PY{p}{)}
\PY{n}{plt}\PY{o}{.}\PY{n}{title}\PY{p}{(}\PY{l+s+s1}{\PYZsq{}}\PY{l+s+s1}{Popularność platformy w czasie}\PY{l+s+s1}{\PYZsq{}}\PY{p}{)}
\end{Verbatim}
\end{tcolorbox}

            \begin{tcolorbox}[breakable, size=fbox, boxrule=.5pt, pad at break*=1mm, opacityfill=0]
\prompt{Out}{outcolor}{53}{\boxspacing}
\begin{Verbatim}[commandchars=\\\{\}]
Text(0.5, 1.0, 'Popularność platformy w czasie')
\end{Verbatim}
\end{tcolorbox}
        
    \begin{center}
    \adjustimage{max size={0.9\linewidth}{0.9\paperheight}}{output_64_1.png}
    \end{center}
    { \hspace*{\fill} \\}
    
    \hypertarget{star-wars-gwiezdne-wojny}{%
\paragraph{``Star Wars'' -- ``Gwiezdne
Wojny''}\label{star-wars-gwiezdne-wojny}}

    \begin{tcolorbox}[breakable, size=fbox, boxrule=1pt, pad at break*=1mm,colback=cellbackground, colframe=cellborder]
\prompt{In}{incolor}{54}{\boxspacing}
\begin{Verbatim}[commandchars=\\\{\}]
\PY{c+c1}{\PYZsh{} ograniczenie się tylko do serii \PYZdq{}Gwiezdne Wojny\PYZdq{}}

\PY{c+c1}{\PYZsh{} \PYZsq{}Star Wars: Episode\PYZsq{} ogranicza się do 7 filmów}
\PY{c+c1}{\PYZsh{} str.contains() sprawdza czy wyrażenie regularne jest zawarte w ciągu}

\PY{n}{dfsw} \PY{o}{=} \PY{n}{dfm}\PY{p}{[}\PY{n}{dfm}\PY{p}{[}\PY{l+s+s1}{\PYZsq{}}\PY{l+s+s1}{title}\PY{l+s+s1}{\PYZsq{}}\PY{p}{]}\PY{o}{.}\PY{n}{str}\PY{o}{.}\PY{n}{contains}\PY{p}{(}\PY{l+s+s1}{\PYZsq{}}\PY{l+s+s1}{Star Wars: Episode}\PY{l+s+s1}{\PYZsq{}}\PY{p}{)}\PY{p}{]}\PY{o}{.}\PY{n}{sort\PYZus{}values}\PY{p}{(}\PY{n}{by}\PY{o}{=}\PY{l+s+s1}{\PYZsq{}}\PY{l+s+s1}{title}\PY{l+s+s1}{\PYZsq{}}\PY{p}{)}
\PY{n}{dfsw}
\end{Verbatim}
\end{tcolorbox}

            \begin{tcolorbox}[breakable, size=fbox, boxrule=.5pt, pad at break*=1mm, opacityfill=0]
\prompt{Out}{outcolor}{54}{\boxspacing}
\begin{Verbatim}[commandchars=\\\{\}]
       movieId                                            title  \textbackslash{}
2544      2628       Star Wars: Episode I - The Phantom Menace
5282      5378    Star Wars: Episode II - Attack of the Clones
10123    33493    Star Wars: Episode III - Revenge of the Sith
257        260              Star Wars: Episode IV - A New Hope
1171      1196  Star Wars: Episode V - The Empire Strikes Back
1184      1210      Star Wars: Episode VI - Return of the Jedi
26470   122886      Star Wars: Episode VII - The Force Awakens

                                     genres  year
2544            [action, adventure, sci-fi]  1999
5282      [action, adventure, sci-fi, imax]  2002
10123           [action, adventure, sci-fi]  2005
257             [action, adventure, sci-fi]  1977
1171            [action, adventure, sci-fi]  1980
1184            [action, adventure, sci-fi]  1983
26470  [action, adventure, fantasy, sci-fi]  2015
\end{Verbatim}
\end{tcolorbox}
        
    \begin{tcolorbox}[breakable, size=fbox, boxrule=1pt, pad at break*=1mm,colback=cellbackground, colframe=cellborder]
\prompt{In}{incolor}{55}{\boxspacing}
\begin{Verbatim}[commandchars=\\\{\}]
\PY{c+c1}{\PYZsh{} przez \PYZsq{}range\PYZsq{} (przedział, zakres) numerowanie części \PYZsq{}Star Wars: Episode\PYZsq{}}
\PY{c+c1}{\PYZsh{} tworzenie nowej kolumny \PYZsq{}episode\PYZsq{} z numerami 7 części filmu}
\PY{c+c1}{\PYZsh{} odrzucenie kolumny \PYZsq{}genres\PYZsq{}(gatunki), zmiana kolejności pozostałych }

\PY{n}{dfsw} \PY{o}{=} \PY{n}{dfm}\PY{p}{[}\PY{n}{dfm}\PY{p}{[}\PY{l+s+s1}{\PYZsq{}}\PY{l+s+s1}{title}\PY{l+s+s1}{\PYZsq{}}\PY{p}{]}\PY{o}{.}\PY{n}{str}\PY{o}{.}\PY{n}{contains}\PY{p}{(}\PY{l+s+s1}{\PYZsq{}}\PY{l+s+s1}{Star Wars: Episode}\PY{l+s+s1}{\PYZsq{}}\PY{p}{)}\PY{p}{]}\PY{o}{.}\PY{n}{sort\PYZus{}values}\PY{p}{(}\PY{n}{by}\PY{o}{=}\PY{l+s+s1}{\PYZsq{}}\PY{l+s+s1}{title}\PY{l+s+s1}{\PYZsq{}}\PY{p}{)}
\PY{n}{dfsw}\PY{p}{[}\PY{l+s+s1}{\PYZsq{}}\PY{l+s+s1}{episode}\PY{l+s+s1}{\PYZsq{}}\PY{p}{]} \PY{o}{=} \PY{n+nb}{range}\PY{p}{(}\PY{l+m+mi}{1}\PY{p}{,} \PY{l+m+mi}{8}\PY{p}{)}
\PY{n}{dfsw}\PY{p}{[}\PY{p}{[}\PY{l+s+s1}{\PYZsq{}}\PY{l+s+s1}{movieId}\PY{l+s+s1}{\PYZsq{}}\PY{p}{,} \PY{l+s+s1}{\PYZsq{}}\PY{l+s+s1}{episode}\PY{l+s+s1}{\PYZsq{}}\PY{p}{,} \PY{l+s+s1}{\PYZsq{}}\PY{l+s+s1}{title}\PY{l+s+s1}{\PYZsq{}}\PY{p}{,} \PY{l+s+s1}{\PYZsq{}}\PY{l+s+s1}{year}\PY{l+s+s1}{\PYZsq{}}\PY{p}{]}\PY{p}{]}
\end{Verbatim}
\end{tcolorbox}

            \begin{tcolorbox}[breakable, size=fbox, boxrule=.5pt, pad at break*=1mm, opacityfill=0]
\prompt{Out}{outcolor}{55}{\boxspacing}
\begin{Verbatim}[commandchars=\\\{\}]
       movieId  episode                                            title  year
2544      2628        1       Star Wars: Episode I - The Phantom Menace   1999
5282      5378        2    Star Wars: Episode II - Attack of the Clones   2002
10123    33493        3    Star Wars: Episode III - Revenge of the Sith   2005
257        260        4              Star Wars: Episode IV - A New Hope   1977
1171      1196        5  Star Wars: Episode V - The Empire Strikes Back   1980
1184      1210        6      Star Wars: Episode VI - Return of the Jedi   1983
26470   122886        7      Star Wars: Episode VII - The Force Awakens   2015
\end{Verbatim}
\end{tcolorbox}
        
    \begin{tcolorbox}[breakable, size=fbox, boxrule=1pt, pad at break*=1mm,colback=cellbackground, colframe=cellborder]
\prompt{In}{incolor}{56}{\boxspacing}
\begin{Verbatim}[commandchars=\\\{\}]
\PY{c+c1}{\PYZsh{} przez \PYZsq{}merge\PYZsq{} łączenie dwóch DataFrame}
\PY{c+c1}{\PYZsh{} dodanie ocen tych filmów łącząc z tabelą \PYZsq{}df\PYZsq{} na wspólnym kluczu \PYZsq{}movieId\PYZsq{}}
\PY{c+c1}{\PYZsh{} mamy dane wszystkich części \PYZsq{}Star Wars\PYZsq{} z ocenami i czasem ich wystawienia}

\PY{n}{dfsw} \PY{o}{=} \PY{n}{dfm}\PY{p}{[}\PY{n}{dfm}\PY{p}{[}\PY{l+s+s1}{\PYZsq{}}\PY{l+s+s1}{title}\PY{l+s+s1}{\PYZsq{}}\PY{p}{]}\PY{o}{.}\PY{n}{str}\PY{o}{.}\PY{n}{contains}\PY{p}{(}\PY{l+s+s1}{\PYZsq{}}\PY{l+s+s1}{Star Wars: Episode}\PY{l+s+s1}{\PYZsq{}}\PY{p}{)}\PY{p}{]}\PY{o}{.}\PY{n}{sort\PYZus{}values}\PY{p}{(}\PY{n}{by}\PY{o}{=}\PY{l+s+s1}{\PYZsq{}}\PY{l+s+s1}{title}\PY{l+s+s1}{\PYZsq{}}\PY{p}{)}
\PY{n}{dfsw}\PY{p}{[}\PY{l+s+s1}{\PYZsq{}}\PY{l+s+s1}{episode}\PY{l+s+s1}{\PYZsq{}}\PY{p}{]} \PY{o}{=} \PY{n+nb}{range}\PY{p}{(}\PY{l+m+mi}{1}\PY{p}{,} \PY{l+m+mi}{8}\PY{p}{)}
\PY{n}{dfsw} \PY{o}{=} \PY{n}{dfsw}\PY{p}{[}\PY{p}{[}\PY{l+s+s1}{\PYZsq{}}\PY{l+s+s1}{movieId}\PY{l+s+s1}{\PYZsq{}}\PY{p}{,} \PY{l+s+s1}{\PYZsq{}}\PY{l+s+s1}{episode}\PY{l+s+s1}{\PYZsq{}}\PY{p}{,} \PY{l+s+s1}{\PYZsq{}}\PY{l+s+s1}{title}\PY{l+s+s1}{\PYZsq{}}\PY{p}{,} \PY{l+s+s1}{\PYZsq{}}\PY{l+s+s1}{year}\PY{l+s+s1}{\PYZsq{}}\PY{p}{]}\PY{p}{]}\PY{o}{.}\PY{n}{merge}\PY{p}{(}\PY{n}{df\PYZus{}ratings}\PY{p}{,} \PY{n}{on}\PY{o}{=}\PY{l+s+s1}{\PYZsq{}}\PY{l+s+s1}{movieId}\PY{l+s+s1}{\PYZsq{}}\PY{p}{)}
\PY{n}{dfsw}
\end{Verbatim}
\end{tcolorbox}

            \begin{tcolorbox}[breakable, size=fbox, boxrule=.5pt, pad at break*=1mm, opacityfill=0]
\prompt{Out}{outcolor}{56}{\boxspacing}
\begin{Verbatim}[commandchars=\\\{\}]
       movieId  episode                                        title  year  \textbackslash{}
0         2628        1   Star Wars: Episode I - The Phantom Menace   1999
1         2628        1   Star Wars: Episode I - The Phantom Menace   1999
2         2628        1   Star Wars: Episode I - The Phantom Menace   1999
3         2628        1   Star Wars: Episode I - The Phantom Menace   1999
4         2628        1   Star Wars: Episode I - The Phantom Menace   1999
{\ldots}        {\ldots}      {\ldots}                                          {\ldots}   {\ldots}
10608   122886        7  Star Wars: Episode VII - The Force Awakens   2015
10609   122886        7  Star Wars: Episode VII - The Force Awakens   2015
10610   122886        7  Star Wars: Episode VII - The Force Awakens   2015
10611   122886        7  Star Wars: Episode VII - The Force Awakens   2015
10612   122886        7  Star Wars: Episode VII - The Force Awakens   2015

       rating           timestamp
0         3.0 2000-11-22 04:34:41
1         2.0 2005-11-28 03:16:45
2         5.0 1999-12-12 09:10:34
3         4.0 2002-05-21 08:26:40
4         4.0 2000-11-26 21:49:39
{\ldots}       {\ldots}                 {\ldots}
10608     4.5 2016-01-05 08:16:27
10609     4.0 2016-01-03 18:23:09
10610     5.0 2016-01-25 17:40:31
10611     2.5 2015-10-25 19:01:58
10612     5.0 2016-01-02 22:49:43

[10613 rows x 6 columns]
\end{Verbatim}
\end{tcolorbox}
        
    \begin{tcolorbox}[breakable, size=fbox, boxrule=1pt, pad at break*=1mm,colback=cellbackground, colframe=cellborder]
\prompt{In}{incolor}{57}{\boxspacing}
\begin{Verbatim}[commandchars=\\\{\}]
\PY{c+c1}{\PYZsh{} stworzenie pustego słownika}
\PY{c+c1}{\PYZsh{} z tabeli będziemy wybierać mniejsze tabele odwołując się do konkretnej części filmu}
\PY{c+c1}{\PYZsh{} z otrzymanej tabeli 3 kolumny (\PYZsq{}rating\PYZsq{}, \PYZsq{}year\PYZsq{}, \PYZsq{}timestamp\PYZsq{})}
\PY{c+c1}{\PYZsh{} na tej tabeli wywołamy funkcję \PYZsq{}resample\PYZsq{}, gdzie będziemy zliczać średnie w przedziałach rocznych}
\PY{c+c1}{\PYZsh{} \PYZsq{}Y\PYZsq{} jako pierwszy argument, \PYZsq{}mean\PYZsq{} jako funkcja agregująca}
\PY{c+c1}{\PYZsh{} wynik zapisany do słownika \PYZsq{}sw[i]\PYZsq{}}

\PY{n}{sw} \PY{o}{=} \PY{p}{\PYZob{}}\PY{p}{\PYZcb{}}
\PY{k}{for} \PY{n}{i} \PY{o+ow}{in} \PY{n+nb}{range}\PY{p}{(}\PY{l+m+mi}{1}\PY{p}{,} \PY{l+m+mi}{8}\PY{p}{)}\PY{p}{:}
    \PY{n}{sw}\PY{p}{[}\PY{n}{i}\PY{p}{]} \PY{o}{=} \PY{n}{dfsw}\PY{p}{[}\PY{n}{dfsw}\PY{p}{[}\PY{l+s+s1}{\PYZsq{}}\PY{l+s+s1}{episode}\PY{l+s+s1}{\PYZsq{}}\PY{p}{]} \PY{o}{==} \PY{n}{i}\PY{p}{]}\PY{p}{[}\PY{p}{[}\PY{l+s+s1}{\PYZsq{}}\PY{l+s+s1}{rating}\PY{l+s+s1}{\PYZsq{}}\PY{p}{,} \PY{l+s+s1}{\PYZsq{}}\PY{l+s+s1}{year}\PY{l+s+s1}{\PYZsq{}}\PY{p}{,} \PY{l+s+s1}{\PYZsq{}}\PY{l+s+s1}{timestamp}\PY{l+s+s1}{\PYZsq{}}\PY{p}{]}\PY{p}{]} \PYZbs{}
        \PY{o}{.}\PY{n}{resample}\PY{p}{(}\PY{l+s+s1}{\PYZsq{}}\PY{l+s+s1}{Y}\PY{l+s+s1}{\PYZsq{}}\PY{p}{,} \PY{n}{on}\PY{o}{=}\PY{l+s+s1}{\PYZsq{}}\PY{l+s+s1}{timestamp}\PY{l+s+s1}{\PYZsq{}}\PY{p}{)}\PY{o}{.}\PY{n}{mean}\PY{p}{(}\PY{p}{)}
\end{Verbatim}
\end{tcolorbox}

    \begin{tcolorbox}[breakable, size=fbox, boxrule=1pt, pad at break*=1mm,colback=cellbackground, colframe=cellborder]
\prompt{In}{incolor}{58}{\boxspacing}
\begin{Verbatim}[commandchars=\\\{\}]
\PY{c+c1}{\PYZsh{} mamy więc tabelę, w której kluczem będzie \PYZsq{}timestamp\PYZsq{}}
\PY{c+c1}{\PYZsh{} następnie będzie \PYZsq{}rating\PYZsq{} i rok produkcji \PYZsq{}year\PYZsq{}}

\PY{n}{sw}\PY{p}{[}\PY{l+m+mi}{1}\PY{p}{]}
\end{Verbatim}
\end{tcolorbox}

            \begin{tcolorbox}[breakable, size=fbox, boxrule=.5pt, pad at break*=1mm, opacityfill=0]
\prompt{Out}{outcolor}{58}{\boxspacing}
\begin{Verbatim}[commandchars=\\\{\}]
              rating  year
timestamp
1999-12-31  3.297619  1999
2000-12-31  3.440594  1999
2001-12-31  3.322835  1999
2002-12-31  3.156250  1999
2003-12-31  3.054054  1999
2004-12-31  2.916667  1999
2005-12-31  2.885246  1999
2006-12-31  2.812500  1999
2007-12-31  2.824176  1999
2008-12-31  3.060811  1999
2009-12-31  3.157407  1999
2010-12-31  2.984848  1999
2011-12-31  2.905660  1999
2012-12-31  2.892857  1999
2013-12-31  3.038462  1999
2014-12-31  2.964286  1999
2015-12-31  3.074627  1999
2016-12-31  2.200000  1999
\end{Verbatim}
\end{tcolorbox}
        
    \begin{tcolorbox}[breakable, size=fbox, boxrule=1pt, pad at break*=1mm,colback=cellbackground, colframe=cellborder]
\prompt{In}{incolor}{59}{\boxspacing}
\begin{Verbatim}[commandchars=\\\{\}]
\PY{c+c1}{\PYZsh{} następnie zresetowanie tego indeksu i stworzenie nowej kolumny}
\PY{c+c1}{\PYZsh{} stworzenie kolumny \PYZsq{}year\PYZus{}since\PYZsq{} (ile lat upłynęło od ukazania się filmu)}
\PY{c+c1}{\PYZsh{} wybierzemy rok i odejmujemy rok produkcji}

\PY{n}{sw} \PY{o}{=} \PY{p}{\PYZob{}}\PY{p}{\PYZcb{}}
\PY{k}{for} \PY{n}{i} \PY{o+ow}{in} \PY{n+nb}{range}\PY{p}{(}\PY{l+m+mi}{1}\PY{p}{,} \PY{l+m+mi}{8}\PY{p}{)}\PY{p}{:}
    \PY{n}{sw}\PY{p}{[}\PY{n}{i}\PY{p}{]} \PY{o}{=} \PY{n}{dfsw}\PY{p}{[}\PY{n}{dfsw}\PY{p}{[}\PY{l+s+s1}{\PYZsq{}}\PY{l+s+s1}{episode}\PY{l+s+s1}{\PYZsq{}}\PY{p}{]} \PY{o}{==} \PY{n}{i}\PY{p}{]}\PY{p}{[}\PY{p}{[}\PY{l+s+s1}{\PYZsq{}}\PY{l+s+s1}{rating}\PY{l+s+s1}{\PYZsq{}}\PY{p}{,} \PY{l+s+s1}{\PYZsq{}}\PY{l+s+s1}{year}\PY{l+s+s1}{\PYZsq{}}\PY{p}{,} \PY{l+s+s1}{\PYZsq{}}\PY{l+s+s1}{timestamp}\PY{l+s+s1}{\PYZsq{}}\PY{p}{]}\PY{p}{]} \PYZbs{}
            \PY{o}{.}\PY{n}{resample}\PY{p}{(}\PY{l+s+s1}{\PYZsq{}}\PY{l+s+s1}{Y}\PY{l+s+s1}{\PYZsq{}}\PY{p}{,} \PY{n}{on}\PY{o}{=}\PY{l+s+s1}{\PYZsq{}}\PY{l+s+s1}{timestamp}\PY{l+s+s1}{\PYZsq{}}\PY{p}{)}\PY{o}{.}\PY{n}{mean}\PY{p}{(}\PY{p}{)}
    \PY{n}{sw}\PY{p}{[}\PY{n}{i}\PY{p}{]} \PY{o}{=} \PY{n}{sw}\PY{p}{[}\PY{n}{i}\PY{p}{]}\PY{o}{.}\PY{n}{reset\PYZus{}index}\PY{p}{(}\PY{p}{)}
    \PY{n}{sw}\PY{p}{[}\PY{n}{i}\PY{p}{]}\PY{p}{[}\PY{l+s+s1}{\PYZsq{}}\PY{l+s+s1}{years\PYZus{}since}\PY{l+s+s1}{\PYZsq{}}\PY{p}{]} \PY{o}{=} \PY{n}{sw}\PY{p}{[}\PY{n}{i}\PY{p}{]}\PY{p}{[}\PY{l+s+s1}{\PYZsq{}}\PY{l+s+s1}{timestamp}\PY{l+s+s1}{\PYZsq{}}\PY{p}{]}\PY{o}{.}\PY{n}{dt}\PY{o}{.}\PY{n}{year} \PY{o}{\PYZhy{}} \PY{n}{sw}\PY{p}{[}\PY{n}{i}\PY{p}{]}\PY{p}{[}\PY{l+s+s1}{\PYZsq{}}\PY{l+s+s1}{year}\PY{l+s+s1}{\PYZsq{}}\PY{p}{]}
\end{Verbatim}
\end{tcolorbox}

    \begin{tcolorbox}[breakable, size=fbox, boxrule=1pt, pad at break*=1mm,colback=cellbackground, colframe=cellborder]
\prompt{In}{incolor}{60}{\boxspacing}
\begin{Verbatim}[commandchars=\\\{\}]
\PY{n}{sw}\PY{p}{[}\PY{l+m+mi}{1}\PY{p}{]}
\end{Verbatim}
\end{tcolorbox}

            \begin{tcolorbox}[breakable, size=fbox, boxrule=.5pt, pad at break*=1mm, opacityfill=0]
\prompt{Out}{outcolor}{60}{\boxspacing}
\begin{Verbatim}[commandchars=\\\{\}]
    timestamp    rating  year  years\_since
0  1999-12-31  3.297619  1999            0
1  2000-12-31  3.440594  1999            1
2  2001-12-31  3.322835  1999            2
3  2002-12-31  3.156250  1999            3
4  2003-12-31  3.054054  1999            4
5  2004-12-31  2.916667  1999            5
6  2005-12-31  2.885246  1999            6
7  2006-12-31  2.812500  1999            7
8  2007-12-31  2.824176  1999            8
9  2008-12-31  3.060811  1999            9
10 2009-12-31  3.157407  1999           10
11 2010-12-31  2.984848  1999           11
12 2011-12-31  2.905660  1999           12
13 2012-12-31  2.892857  1999           13
14 2013-12-31  3.038462  1999           14
15 2014-12-31  2.964286  1999           15
16 2015-12-31  3.074627  1999           16
17 2016-12-31  2.200000  1999           17
\end{Verbatim}
\end{tcolorbox}
        
    \begin{tcolorbox}[breakable, size=fbox, boxrule=1pt, pad at break*=1mm,colback=cellbackground, colframe=cellborder]
\prompt{In}{incolor}{61}{\boxspacing}
\begin{Verbatim}[commandchars=\\\{\}]
\PY{c+c1}{\PYZsh{} kolumna \PYZsq{}year\PYZsq{} już nieistotna}
\PY{c+c1}{\PYZsh{} indeksem nowej tabeli będzie \PYZsq{}years\PYZus{}since\PYZsq{}}
\PY{c+c1}{\PYZsh{} otrzymamy tabelą z indeksem, którym jest ilość lat od produkcji}
\PY{c+c1}{\PYZsh{} drugą kolumną \PYZsq{}rating\PYZsq{} (ocena średnia w tym roku)}

\PY{n}{sw} \PY{o}{=} \PY{p}{\PYZob{}}\PY{p}{\PYZcb{}}
\PY{k}{for} \PY{n}{i} \PY{o+ow}{in} \PY{n+nb}{range}\PY{p}{(}\PY{l+m+mi}{1}\PY{p}{,} \PY{l+m+mi}{8}\PY{p}{)}\PY{p}{:}
    \PY{n}{sw}\PY{p}{[}\PY{n}{i}\PY{p}{]} \PY{o}{=} \PY{n}{dfsw}\PY{p}{[}\PY{n}{dfsw}\PY{p}{[}\PY{l+s+s1}{\PYZsq{}}\PY{l+s+s1}{episode}\PY{l+s+s1}{\PYZsq{}}\PY{p}{]} \PY{o}{==} \PY{n}{i}\PY{p}{]}\PY{p}{[}\PY{p}{[}\PY{l+s+s1}{\PYZsq{}}\PY{l+s+s1}{rating}\PY{l+s+s1}{\PYZsq{}}\PY{p}{,} \PY{l+s+s1}{\PYZsq{}}\PY{l+s+s1}{year}\PY{l+s+s1}{\PYZsq{}}\PY{p}{,} \PY{l+s+s1}{\PYZsq{}}\PY{l+s+s1}{timestamp}\PY{l+s+s1}{\PYZsq{}}\PY{p}{]}\PY{p}{]} \PYZbs{}
            \PY{o}{.}\PY{n}{resample}\PY{p}{(}\PY{l+s+s1}{\PYZsq{}}\PY{l+s+s1}{Y}\PY{l+s+s1}{\PYZsq{}}\PY{p}{,} \PY{n}{on}\PY{o}{=}\PY{l+s+s1}{\PYZsq{}}\PY{l+s+s1}{timestamp}\PY{l+s+s1}{\PYZsq{}}\PY{p}{)}\PY{o}{.}\PY{n}{mean}\PY{p}{(}\PY{p}{)}
    \PY{n}{sw}\PY{p}{[}\PY{n}{i}\PY{p}{]} \PY{o}{=} \PY{n}{sw}\PY{p}{[}\PY{n}{i}\PY{p}{]}\PY{o}{.}\PY{n}{reset\PYZus{}index}\PY{p}{(}\PY{p}{)}
    \PY{n}{sw}\PY{p}{[}\PY{n}{i}\PY{p}{]}\PY{p}{[}\PY{l+s+s1}{\PYZsq{}}\PY{l+s+s1}{years\PYZus{}since}\PY{l+s+s1}{\PYZsq{}}\PY{p}{]} \PY{o}{=} \PY{n}{sw}\PY{p}{[}\PY{n}{i}\PY{p}{]}\PY{p}{[}\PY{l+s+s1}{\PYZsq{}}\PY{l+s+s1}{timestamp}\PY{l+s+s1}{\PYZsq{}}\PY{p}{]}\PY{o}{.}\PY{n}{dt}\PY{o}{.}\PY{n}{year} \PY{o}{\PYZhy{}} \PY{n}{sw}\PY{p}{[}\PY{n}{i}\PY{p}{]}\PY{p}{[}\PY{l+s+s1}{\PYZsq{}}\PY{l+s+s1}{year}\PY{l+s+s1}{\PYZsq{}}\PY{p}{]}
    \PY{n}{sw}\PY{p}{[}\PY{n}{i}\PY{p}{]} \PY{o}{=} \PY{n}{sw}\PY{p}{[}\PY{n}{i}\PY{p}{]}\PY{p}{[}\PY{p}{[}\PY{l+s+s1}{\PYZsq{}}\PY{l+s+s1}{rating}\PY{l+s+s1}{\PYZsq{}}\PY{p}{,} \PY{l+s+s1}{\PYZsq{}}\PY{l+s+s1}{years\PYZus{}since}\PY{l+s+s1}{\PYZsq{}}\PY{p}{]}\PY{p}{]}\PY{o}{.}\PY{n}{set\PYZus{}index}\PY{p}{(}\PY{l+s+s1}{\PYZsq{}}\PY{l+s+s1}{years\PYZus{}since}\PY{l+s+s1}{\PYZsq{}}\PY{p}{)}
\end{Verbatim}
\end{tcolorbox}

    \begin{tcolorbox}[breakable, size=fbox, boxrule=1pt, pad at break*=1mm,colback=cellbackground, colframe=cellborder]
\prompt{In}{incolor}{62}{\boxspacing}
\begin{Verbatim}[commandchars=\\\{\}]
\PY{n}{sw}\PY{p}{[}\PY{l+m+mi}{1}\PY{p}{]}
\end{Verbatim}
\end{tcolorbox}

            \begin{tcolorbox}[breakable, size=fbox, boxrule=.5pt, pad at break*=1mm, opacityfill=0]
\prompt{Out}{outcolor}{62}{\boxspacing}
\begin{Verbatim}[commandchars=\\\{\}]
               rating
years\_since
0            3.297619
1            3.440594
2            3.322835
3            3.156250
4            3.054054
5            2.916667
6            2.885246
7            2.812500
8            2.824176
9            3.060811
10           3.157407
11           2.984848
12           2.905660
13           2.892857
14           3.038462
15           2.964286
16           3.074627
17           2.200000
\end{Verbatim}
\end{tcolorbox}
        
    \begin{tcolorbox}[breakable, size=fbox, boxrule=1pt, pad at break*=1mm,colback=cellbackground, colframe=cellborder]
\prompt{In}{incolor}{63}{\boxspacing}
\begin{Verbatim}[commandchars=\\\{\}]
\PY{c+c1}{\PYZsh{} takie tabele możemy złączyć w jedną używając funkcji \PYZsq{}concat\PYZsq{}}
\PY{c+c1}{\PYZsh{} uzyskujemy podobny wynik dla każdej części filmu \PYZdq{}Star Wars\PYZdq{}}
\PY{c+c1}{\PYZsh{} dla ostatniej kolumny tylko dwie wartości, bo film ukazał się najpóźniej}
\PY{c+c1}{\PYZsh{} część 4,5,6 powstały najwcześniej}

\PY{n}{pd}\PY{o}{.}\PY{n}{concat}\PY{p}{(}\PY{n}{sw}\PY{p}{,} \PY{n}{axis}\PY{o}{=}\PY{l+m+mi}{1}\PY{p}{)}
\end{Verbatim}
\end{tcolorbox}

            \begin{tcolorbox}[breakable, size=fbox, boxrule=.5pt, pad at break*=1mm, opacityfill=0]
\prompt{Out}{outcolor}{63}{\boxspacing}
\begin{Verbatim}[commandchars=\\\{\}]
                    1         2         3         4         5         6  \textbackslash{}
               rating    rating    rating    rating    rating    rating
years\_since
0            3.297619  3.448718  3.739496       NaN       NaN       NaN
1            3.440594  3.280000  3.561538       NaN       NaN       NaN
2            3.322835  3.253521  3.396341       NaN       NaN       NaN
3            3.156250  3.042857  3.593750       NaN       NaN       NaN
4            3.054054  3.000000  3.622642       NaN       NaN       NaN
5            2.916667  3.111111  3.254902       NaN       NaN       NaN
6            2.885246  3.246269  3.571429       NaN       NaN       NaN
7            2.812500  3.166667  3.108696       NaN       NaN       NaN
8            2.824176  2.962963  3.794118       NaN       NaN       NaN
9            3.060811  3.181818  3.108696       NaN       NaN       NaN
10           3.157407  3.000000  3.252475       NaN       NaN       NaN
11           2.984848  3.441176  3.590909       NaN       NaN       NaN
12           2.905660  2.525000       NaN       NaN       NaN       NaN
13           2.892857  2.835616       NaN       NaN       NaN  4.272727
14           3.038462  2.500000       NaN       NaN       NaN  4.122378
15           2.964286       NaN       NaN       NaN       NaN  4.080000
16           3.074627       NaN       NaN       NaN  4.257143  3.962162
17           2.200000       NaN       NaN       NaN  4.220930  3.950725
18                NaN       NaN       NaN       NaN  4.090909  3.882051
19                NaN       NaN       NaN  4.049505  4.172249  4.012987
20                NaN       NaN       NaN  4.256018  4.250000  3.950000
21                NaN       NaN       NaN  4.288462  4.137725  3.983871
22                NaN       NaN       NaN  4.292237  4.302752  3.944444
23                NaN       NaN       NaN  4.373702  4.195946  3.829545
24                NaN       NaN       NaN  4.314917  4.275862  4.014423
25                NaN       NaN       NaN  4.301724  4.169643  4.119048
26                NaN       NaN       NaN  4.260274  4.014286  3.966216
27                NaN       NaN       NaN  4.183333  4.257009  4.084270
28                NaN       NaN       NaN  3.986339  4.205882  3.784722
29                NaN       NaN       NaN  4.056122  4.000000  4.125000
30                NaN       NaN       NaN  4.043478  4.146739  4.204082
31                NaN       NaN       NaN  4.166667  4.112500  4.144737
32                NaN       NaN       NaN  4.006173  4.287037  3.958084
33                NaN       NaN       NaN  4.125000  4.214286  4.027778
34                NaN       NaN       NaN  4.000000  4.416667       NaN
35                NaN       NaN       NaN  4.203704  4.003906       NaN
36                NaN       NaN       NaN  4.149254  4.314815       NaN
37                NaN       NaN       NaN  4.230769       NaN       NaN
38                NaN       NaN       NaN  3.906757       NaN       NaN
39                NaN       NaN       NaN  4.250000       NaN       NaN

                    7
               rating
years\_since
0            3.860000
1            4.038462
2                 NaN
3                 NaN
4                 NaN
5                 NaN
6                 NaN
7                 NaN
8                 NaN
9                 NaN
10                NaN
11                NaN
12                NaN
13                NaN
14                NaN
15                NaN
16                NaN
17                NaN
18                NaN
19                NaN
20                NaN
21                NaN
22                NaN
23                NaN
24                NaN
25                NaN
26                NaN
27                NaN
28                NaN
29                NaN
30                NaN
31                NaN
32                NaN
33                NaN
34                NaN
35                NaN
36                NaN
37                NaN
38                NaN
39                NaN
\end{Verbatim}
\end{tcolorbox}
        
    \begin{tcolorbox}[breakable, size=fbox, boxrule=1pt, pad at break*=1mm,colback=cellbackground, colframe=cellborder]
\prompt{In}{incolor}{64}{\boxspacing}
\begin{Verbatim}[commandchars=\\\{\}]
\PY{c+c1}{\PYZsh{} wyświetlenie wyników przez \PYZsq{}concat\PYZsq{} i dodatkowo funkcję \PYZsq{}plot\PYZsq{}}

\PY{c+c1}{\PYZsh{} nowsze części filmu oceniane gorzej niż wcześniejsze}
\PY{c+c1}{\PYZsh{} z upływem czasu widać też spadek ocen}

\PY{c+c1}{\PYZsh{} te wcześniejsze części filmu dużo lepiej oceniane}
\PY{c+c1}{\PYZsh{} również dużo bardziej stałe w swojej ocenie}
 
\PY{n}{pd}\PY{o}{.}\PY{n}{concat}\PY{p}{(}\PY{n}{sw}\PY{p}{,} \PY{n}{axis}\PY{o}{=}\PY{l+m+mi}{1}\PY{p}{)}\PY{o}{.}\PY{n}{plot}\PY{p}{(}\PY{n}{figsize}\PY{o}{=}\PY{p}{(}\PY{l+m+mi}{14}\PY{p}{,} \PY{l+m+mi}{5}\PY{p}{)}\PY{p}{)}
\PY{n}{plt}\PY{o}{.}\PY{n}{title}\PY{p}{(}\PY{l+s+s1}{\PYZsq{}}\PY{l+s+s1}{Oceny poszczególnych części filmu w czasie}\PY{l+s+s1}{\PYZsq{}}\PY{p}{)}
\end{Verbatim}
\end{tcolorbox}

            \begin{tcolorbox}[breakable, size=fbox, boxrule=.5pt, pad at break*=1mm, opacityfill=0]
\prompt{Out}{outcolor}{64}{\boxspacing}
\begin{Verbatim}[commandchars=\\\{\}]
Text(0.5, 1.0, 'Oceny poszczególnych części filmu w czasie')
\end{Verbatim}
\end{tcolorbox}
        
    \begin{center}
    \adjustimage{max size={0.9\linewidth}{0.9\paperheight}}{output_76_1.png}
    \end{center}
    { \hspace*{\fill} \\}
    
    \begin{tcolorbox}[breakable, size=fbox, boxrule=1pt, pad at break*=1mm,colback=cellbackground, colframe=cellborder]
\prompt{In}{incolor}{65}{\boxspacing}
\begin{Verbatim}[commandchars=\\\{\}]
\PY{c+c1}{\PYZsh{} części 4,5,6 powstały najwcześniej}
\PY{c+c1}{\PYZsh{} prawdopodobnie starsze wersje wyżej oceniane niż wcześniejsze}

\PY{n}{dfsw} \PY{o}{=} \PY{n}{dfm}\PY{p}{[}\PY{n}{dfm}\PY{p}{[}\PY{l+s+s1}{\PYZsq{}}\PY{l+s+s1}{title}\PY{l+s+s1}{\PYZsq{}}\PY{p}{]}\PY{o}{.}\PY{n}{str}\PY{o}{.}\PY{n}{contains}\PY{p}{(}\PY{l+s+s1}{\PYZsq{}}\PY{l+s+s1}{Star Wars: Episode}\PY{l+s+s1}{\PYZsq{}}\PY{p}{)}\PY{p}{]}
\PY{n}{dfsw} \PY{o}{=} \PY{n}{dfsw}\PY{p}{[}\PY{p}{[}\PY{l+s+s1}{\PYZsq{}}\PY{l+s+s1}{movieId}\PY{l+s+s1}{\PYZsq{}}\PY{p}{,} \PY{l+s+s1}{\PYZsq{}}\PY{l+s+s1}{title}\PY{l+s+s1}{\PYZsq{}}\PY{p}{,} \PY{l+s+s1}{\PYZsq{}}\PY{l+s+s1}{year}\PY{l+s+s1}{\PYZsq{}}\PY{p}{]}\PY{p}{]}\PY{o}{.}\PY{n}{merge}\PY{p}{(}\PY{n}{df\PYZus{}ratings}\PY{p}{,} \PY{n}{on}\PY{o}{=}\PY{l+s+s1}{\PYZsq{}}\PY{l+s+s1}{movieId}\PY{l+s+s1}{\PYZsq{}}\PY{p}{)}\PY{o}{.}\PY{n}{sort\PYZus{}values}\PY{p}{(}\PY{n}{by}\PY{o}{=}\PY{l+s+s1}{\PYZsq{}}\PY{l+s+s1}{rating}\PY{l+s+s1}{\PYZsq{}}\PY{p}{,} \PY{n}{ascending}\PY{o}{=}\PY{k+kc}{False}\PY{p}{)}
\PY{n}{dfsw}\PY{o}{.}\PY{n}{head}\PY{p}{(}\PY{p}{)}
\end{Verbatim}
\end{tcolorbox}

            \begin{tcolorbox}[breakable, size=fbox, boxrule=.5pt, pad at break*=1mm, opacityfill=0]
\prompt{Out}{outcolor}{65}{\boxspacing}
\begin{Verbatim}[commandchars=\\\{\}]
       movieId                                            title  year  rating  \textbackslash{}
10612   122886      Star Wars: Episode VII - The Force Awakens   2015     5.0
3115      1196  Star Wars: Episode V - The Empire Strikes Back   1980     5.0
3099      1196  Star Wars: Episode V - The Empire Strikes Back   1980     5.0
3100      1196  Star Wars: Episode V - The Empire Strikes Back   1980     5.0
3101      1196  Star Wars: Episode V - The Empire Strikes Back   1980     5.0

                timestamp
10612 2016-01-02 22:49:43
3115  1997-11-16 04:18:03
3099  2006-01-12 01:46:57
3100  2000-12-31 05:58:27
3101  2002-04-02 20:27:37
\end{Verbatim}
\end{tcolorbox}
        
    \hypertarget{wnioski}{%
\paragraph{Wnioski}\label{wnioski}}

\begin{enumerate}
\def\labelenumi{\arabic{enumi}.}
\tightlist
\item
  nowsze części filmu oceniane gorzej niż wcześniejsze
\item
  te wcześniejsze części filmu dużo lepiej oceniane
\item
  również dużo bardziej stałe w swojej ocenie
\end{enumerate}

    \hypertarget{zastosowanie-regresji-wielomianowej}{%
\subsubsection{Zastosowanie regresji
wielomianowej}\label{zastosowanie-regresji-wielomianowej}}

\hypertarget{przykux142ad-star-wars-episode-1-w-interpolacji}{%
\paragraph{1 przykład -- ``Star Wars: Episode 1'' w
interpolacji}\label{przykux142ad-star-wars-episode-1-w-interpolacji}}

Wykorzystanie do znajdowania wartości pośrednich w obecnych czasach bez
prognozowania w przyszłości.

    \begin{tcolorbox}[breakable, size=fbox, boxrule=1pt, pad at break*=1mm,colback=cellbackground, colframe=cellborder]
\prompt{In}{incolor}{66}{\boxspacing}
\begin{Verbatim}[commandchars=\\\{\}]
\PY{c+c1}{\PYZsh{} zmiana reprezenacji danych z pandasowego DataFame na tablicę w Numpy}
\PY{c+c1}{\PYZsh{} najpierw \PYZsq{}reset\PYZus{}index\PYZsq{} i podajemy przedział indeksów w \PYZsq{}numpy\PYZsq{}}
\PY{c+c1}{\PYZsh{} \PYZsq{}x\PYZsq{} będzie liczbą lat od produkcji, a \PYZsq{}y\PYZsq{} będzie oceną }
\PY{c+c1}{\PYZsh{} wykorzystując te \PYZsq{}x\PYZsq{},\PYZsq{}y\PYZsq{} chcemy zastosować regresję wielomianową}

\PY{n}{x} \PY{o}{=} \PY{n}{sw}\PY{p}{[}\PY{l+m+mi}{1}\PY{p}{]}\PY{o}{.}\PY{n}{reset\PYZus{}index}\PY{p}{(}\PY{p}{)}\PY{o}{.}\PY{n}{to\PYZus{}numpy}\PY{p}{(}\PY{p}{)}\PY{p}{[}\PY{p}{:}\PY{p}{,} \PY{l+m+mi}{0}\PY{p}{]}
\PY{n}{y} \PY{o}{=} \PY{n}{sw}\PY{p}{[}\PY{l+m+mi}{1}\PY{p}{]}\PY{o}{.}\PY{n}{reset\PYZus{}index}\PY{p}{(}\PY{p}{)}\PY{o}{.}\PY{n}{to\PYZus{}numpy}\PY{p}{(}\PY{p}{)}\PY{p}{[}\PY{p}{:}\PY{p}{,} \PY{l+m+mi}{1}\PY{p}{]} 
\PY{n}{y}
\end{Verbatim}
\end{tcolorbox}

            \begin{tcolorbox}[breakable, size=fbox, boxrule=.5pt, pad at break*=1mm, opacityfill=0]
\prompt{Out}{outcolor}{66}{\boxspacing}
\begin{Verbatim}[commandchars=\\\{\}]
array([3.29761905, 3.44059406, 3.32283465, 3.15625   , 3.05405405,
       2.91666667, 2.8852459 , 2.8125    , 2.82417582, 3.06081081,
       3.15740741, 2.98484848, 2.90566038, 2.89285714, 3.03846154,
       2.96428571, 3.07462687, 2.2       ])
\end{Verbatim}
\end{tcolorbox}
        
    \begin{tcolorbox}[breakable, size=fbox, boxrule=1pt, pad at break*=1mm,colback=cellbackground, colframe=cellborder]
\prompt{In}{incolor}{67}{\boxspacing}
\begin{Verbatim}[commandchars=\\\{\}]
\PY{c+c1}{\PYZsh{} współczynniki wielomianu uzyskujemy przez funkcję \PYZsq{}polyfit\PYZsq{}}
\PY{c+c1}{\PYZsh{} robimy to na \PYZsq{}x\PYZsq{},\PYZsq{}y\PYZsq{} przekazując \PYZsq{}deg\PYZsq{} (stopień wielomianu)}
\PY{c+c1}{\PYZsh{} \PYZsq{}t\PYZsq{} podnosimy do każdej kolejnej potęgi, stąd \PYZsq{}deg+1\PYZsq{}}
\PY{c+c1}{\PYZsh{} będą więc 4 współczynniki zaczynając od 0}

\PY{c+c1}{\PYZsh{} zastosować można poniżej przedstawiony algorytm}

\PY{k+kn}{from} \PY{n+nn}{numpy} \PY{k+kn}{import} \PY{n}{polyfit}

\PY{c+c1}{\PYZsh{} \PYZsq{}linspace\PYZsq{} zwraca liczby w określonym przedziale}
\PY{c+c1}{\PYZsh{} polyfit \PYZhy{} dopasowanie wielomianowe metodą najmniejszych kwadratów}
\PY{c+c1}{\PYZsh{} dopasowuje wielomian stopnia \PYZsq{}deg\PYZsq{} do (x,y)}
\PY{c+c1}{\PYZsh{} concatenate \PYZhy{} łączy sekwencję wzdłuż istniejącej osi}
\PY{c+c1}{\PYZsh{} sprawdzimy optymalny stopień wielomianu (np. 2,3,4)}

\PY{n}{t} \PY{o}{=} \PY{n}{np}\PY{o}{.}\PY{n}{linspace}\PY{p}{(}\PY{l+m+mi}{0}\PY{p}{,} \PY{l+m+mi}{20}\PY{p}{,} \PY{n}{num}\PY{o}{=}\PY{l+m+mi}{21}\PY{p}{)}

\PY{c+c1}{\PYZsh{}deg = 2}
\PY{n}{deg} \PY{o}{=} \PY{l+m+mi}{3}
\PY{c+c1}{\PYZsh{}deg = 4}

\PY{n}{coeffs} \PY{o}{=} \PY{n}{polyfit}\PY{p}{(}\PY{n}{x}\PY{p}{,} \PY{n}{y}\PY{p}{,} \PY{n}{deg}\PY{o}{=}\PY{n}{deg}\PY{p}{)}\PY{o}{.}\PY{n}{reshape}\PY{p}{(}\PY{o}{\PYZhy{}}\PY{l+m+mi}{1}\PY{p}{,} \PY{l+m+mi}{1}\PY{p}{)}
\PY{n}{X} \PY{o}{=} \PY{n}{np}\PY{o}{.}\PY{n}{concatenate}\PY{p}{(}\PY{n+nb}{list}\PY{p}{(}\PY{n+nb}{map}\PY{p}{(}\PY{k}{lambda} \PY{n}{i}\PY{p}{:} \PY{n}{t}\PY{o}{.}\PY{n}{reshape}\PY{p}{(}\PY{l+m+mi}{1}\PY{p}{,} \PY{o}{\PYZhy{}}\PY{l+m+mi}{1}\PY{p}{)} \PY{o}{*}\PY{o}{*} \PY{n}{i}\PY{p}{,} \PY{n+nb}{reversed}\PY{p}{(}\PY{n+nb}{range}\PY{p}{(}\PY{n}{deg} \PY{o}{+} \PY{l+m+mi}{1}\PY{p}{)}\PY{p}{)}\PY{p}{)}\PY{p}{)}\PY{p}{)}
\PY{n}{Y} \PY{o}{=} \PY{p}{(}\PY{n}{coeffs}\PY{o}{.}\PY{n}{T} \PY{o}{@} \PY{n}{X}\PY{p}{)}\PY{o}{.}\PY{n}{flatten}\PY{p}{(}\PY{p}{)}
\end{Verbatim}
\end{tcolorbox}

    \begin{tcolorbox}[breakable, size=fbox, boxrule=1pt, pad at break*=1mm,colback=cellbackground, colframe=cellborder]
\prompt{In}{incolor}{68}{\boxspacing}
\begin{Verbatim}[commandchars=\\\{\}]
\PY{c+c1}{\PYZsh{} na wykresie oceny widzów dla 1 części \PYZdq{}Star Wars\PYZdq{}}

\PY{c+c1}{\PYZsh{} szkielet wykresu przy użyciu funkcji \PYZsq{}subplots\PYZsq{}}
\PY{c+c1}{\PYZsh{} wywołamy \PYZsq{}ax.plot\PYZsq{}, potem sw[1] dla 1 części i oznaczymy dane przez kółko}

\PY{n}{fig}\PY{p}{,} \PY{n}{ax} \PY{o}{=} \PY{n}{plt}\PY{o}{.}\PY{n}{subplots}\PY{p}{(}\PY{l+m+mi}{1}\PY{p}{,} \PY{l+m+mi}{1}\PY{p}{,} \PY{n}{figsize}\PY{o}{=}\PY{p}{(}\PY{l+m+mi}{12}\PY{p}{,} \PY{l+m+mi}{3}\PY{p}{)}\PY{p}{)}
\PY{n}{ax}\PY{o}{.}\PY{n}{plot}\PY{p}{(}\PY{n}{sw}\PY{p}{[}\PY{l+m+mi}{1}\PY{p}{]}\PY{p}{,} \PY{l+s+s1}{\PYZsq{}}\PY{l+s+s1}{o}\PY{l+s+s1}{\PYZsq{}}\PY{p}{)}
\PY{n}{plt}\PY{o}{.}\PY{n}{title}\PY{p}{(}\PY{l+s+s1}{\PYZsq{}}\PY{l+s+s1}{Przedziały ocen od kilkunastu lat}\PY{l+s+s1}{\PYZsq{}}\PY{p}{)}
\end{Verbatim}
\end{tcolorbox}

            \begin{tcolorbox}[breakable, size=fbox, boxrule=.5pt, pad at break*=1mm, opacityfill=0]
\prompt{Out}{outcolor}{68}{\boxspacing}
\begin{Verbatim}[commandchars=\\\{\}]
Text(0.5, 1.0, 'Przedziały ocen od kilkunastu lat')
\end{Verbatim}
\end{tcolorbox}
        
    \begin{center}
    \adjustimage{max size={0.9\linewidth}{0.9\paperheight}}{output_82_1.png}
    \end{center}
    { \hspace*{\fill} \\}
    
    \begin{tcolorbox}[breakable, size=fbox, boxrule=1pt, pad at break*=1mm,colback=cellbackground, colframe=cellborder]
\prompt{In}{incolor}{69}{\boxspacing}
\begin{Verbatim}[commandchars=\\\{\}]
\PY{c+c1}{\PYZsh{} następnie nasze przewidywania na podstawie regresji wielomianowej}

\PY{c+c1}{\PYZsh{} widzimy, że wielomian 3\PYZhy{}go stopnia był dobrym dopasowaniem}
\PY{c+c1}{\PYZsh{} gdy zmienimy na 2\PYZhy{}go stopnia, to nie mamy dobrego dopasowania}
\PY{c+c1}{\PYZsh{} dla 4\PYZhy{}go stopnia dopasowanie staje się tak dobre, że mało realistyczne}
\PY{c+c1}{\PYZsh{} nie dałoby się przewidzieć co stałoby się z głosami widzów później}

\PY{n}{fig}\PY{p}{,} \PY{n}{ax} \PY{o}{=} \PY{n}{plt}\PY{o}{.}\PY{n}{subplots}\PY{p}{(}\PY{l+m+mi}{1}\PY{p}{,} \PY{l+m+mi}{1}\PY{p}{,} \PY{n}{figsize}\PY{o}{=}\PY{p}{(}\PY{l+m+mi}{12}\PY{p}{,} \PY{l+m+mi}{3}\PY{p}{)}\PY{p}{)}
\PY{n}{ax}\PY{o}{.}\PY{n}{plot}\PY{p}{(}\PY{n}{sw}\PY{p}{[}\PY{l+m+mi}{1}\PY{p}{]}\PY{p}{,} \PY{l+s+s1}{\PYZsq{}}\PY{l+s+s1}{o}\PY{l+s+s1}{\PYZsq{}}\PY{p}{)}
\PY{n}{ax}\PY{o}{.}\PY{n}{plot}\PY{p}{(}\PY{n}{t}\PY{p}{,} \PY{n}{Y}\PY{p}{)}
\PY{n}{plt}\PY{o}{.}\PY{n}{title}\PY{p}{(}\PY{l+s+s1}{\PYZsq{}}\PY{l+s+s1}{Przewidywane wyniki między znanymi węzłami danych}\PY{l+s+s1}{\PYZsq{}}\PY{p}{)}
\end{Verbatim}
\end{tcolorbox}

            \begin{tcolorbox}[breakable, size=fbox, boxrule=.5pt, pad at break*=1mm, opacityfill=0]
\prompt{Out}{outcolor}{69}{\boxspacing}
\begin{Verbatim}[commandchars=\\\{\}]
Text(0.5, 1.0, 'Przewidywane wyniki między znanymi węzłami danych')
\end{Verbatim}
\end{tcolorbox}
        
    \begin{center}
    \adjustimage{max size={0.9\linewidth}{0.9\paperheight}}{output_83_1.png}
    \end{center}
    { \hspace*{\fill} \\}
    
    \hypertarget{przykux142ad-wszystkie-dane-w-ekstrapolacji}{%
\paragraph{2 przykład -- wszystkie dane w
ekstrapolacji}\label{przykux142ad-wszystkie-dane-w-ekstrapolacji}}

Prognozowanie wartości zmiennej lub funkcji poza zakresem, dla którego
mamy dane.

Dopasowanie do istniejących danych pewnej funkcji, następnie wyliczenie
jej wartości w szukanym punkcie w przyszłości.

    \begin{tcolorbox}[breakable, size=fbox, boxrule=1pt, pad at break*=1mm,colback=cellbackground, colframe=cellborder]
\prompt{In}{incolor}{70}{\boxspacing}
\begin{Verbatim}[commandchars=\\\{\}]
\PY{c+c1}{\PYZsh{} ilość filmów w poszczególnych latach}

\PY{n}{dfm}\PY{p}{[}\PY{p}{[}\PY{l+s+s1}{\PYZsq{}}\PY{l+s+s1}{movieId}\PY{l+s+s1}{\PYZsq{}}\PY{p}{,} \PY{l+s+s1}{\PYZsq{}}\PY{l+s+s1}{year}\PY{l+s+s1}{\PYZsq{}}\PY{p}{]}\PY{p}{]}\PY{o}{.}\PY{n}{groupby}\PY{p}{(}\PY{l+s+s1}{\PYZsq{}}\PY{l+s+s1}{year}\PY{l+s+s1}{\PYZsq{}}\PY{p}{)}\PY{o}{.}\PY{n}{count}\PY{p}{(}\PY{p}{)}
\end{Verbatim}
\end{tcolorbox}

            \begin{tcolorbox}[breakable, size=fbox, boxrule=.5pt, pad at break*=1mm, opacityfill=0]
\prompt{Out}{outcolor}{70}{\boxspacing}
\begin{Verbatim}[commandchars=\\\{\}]
       movieId
year
-1          68
 1874        1
 1878        1
 1887        1
 1888        2
{\ldots}        {\ldots}
 2012     1387
 2013     1476
 2014     1420
 2015     1036
 2016       23

[130 rows x 1 columns]
\end{Verbatim}
\end{tcolorbox}
        
    \begin{tcolorbox}[breakable, size=fbox, boxrule=1pt, pad at break*=1mm,colback=cellbackground, colframe=cellborder]
\prompt{In}{incolor}{71}{\boxspacing}
\begin{Verbatim}[commandchars=\\\{\}]
\PY{c+c1}{\PYZsh{} tendencje w formie wykresu}

\PY{n}{dfm}\PY{p}{[}\PY{p}{[}\PY{l+s+s1}{\PYZsq{}}\PY{l+s+s1}{movieId}\PY{l+s+s1}{\PYZsq{}}\PY{p}{,} \PY{l+s+s1}{\PYZsq{}}\PY{l+s+s1}{year}\PY{l+s+s1}{\PYZsq{}}\PY{p}{]}\PY{p}{]}\PY{o}{.}\PY{n}{groupby}\PY{p}{(}\PY{l+s+s1}{\PYZsq{}}\PY{l+s+s1}{year}\PY{l+s+s1}{\PYZsq{}}\PY{p}{)}\PY{o}{.}\PY{n}{count}\PY{p}{(}\PY{p}{)}\PY{o}{.}\PY{n}{loc}\PY{p}{[}\PY{l+m+mi}{0}\PY{p}{:}\PY{l+m+mi}{2014}\PY{p}{]}\PY{o}{.}\PY{n}{plot}\PY{p}{(}\PY{n}{figsize}\PY{o}{=}\PY{p}{(}\PY{l+m+mi}{12}\PY{p}{,} \PY{l+m+mi}{3}\PY{p}{)}\PY{p}{)}
\PY{n}{plt}\PY{o}{.}\PY{n}{title}\PY{p}{(}\PY{l+s+s1}{\PYZsq{}}\PY{l+s+s1}{Ilość filmów w poszczególnych latach}\PY{l+s+s1}{\PYZsq{}}\PY{p}{)}
\end{Verbatim}
\end{tcolorbox}

            \begin{tcolorbox}[breakable, size=fbox, boxrule=.5pt, pad at break*=1mm, opacityfill=0]
\prompt{Out}{outcolor}{71}{\boxspacing}
\begin{Verbatim}[commandchars=\\\{\}]
Text(0.5, 1.0, 'Ilość filmów w poszczególnych latach')
\end{Verbatim}
\end{tcolorbox}
        
    \begin{center}
    \adjustimage{max size={0.9\linewidth}{0.9\paperheight}}{output_86_1.png}
    \end{center}
    { \hspace*{\fill} \\}
    
    \begin{tcolorbox}[breakable, size=fbox, boxrule=1pt, pad at break*=1mm,colback=cellbackground, colframe=cellborder]
\prompt{In}{incolor}{72}{\boxspacing}
\begin{Verbatim}[commandchars=\\\{\}]
\PY{c+c1}{\PYZsh{} najpierw zamienimy tę tabelę na \PYZsq{}numpy\PYZsq{}}
\PY{c+c1}{\PYZsh{} można teraz na tych danych dokonać regresji wielomianowej}
\PY{c+c1}{\PYZsh{} naszymi \PYZsq{}x\PYZsq{} będzie rok produkcji, \PYZsq{}y\PYZsq{} liczba filmów w konkretnych latach}

\PY{n}{X} \PY{o}{=} \PY{n}{dfm}\PY{p}{[}\PY{p}{[}\PY{l+s+s1}{\PYZsq{}}\PY{l+s+s1}{movieId}\PY{l+s+s1}{\PYZsq{}}\PY{p}{,} \PY{l+s+s1}{\PYZsq{}}\PY{l+s+s1}{year}\PY{l+s+s1}{\PYZsq{}}\PY{p}{]}\PY{p}{]}\PY{o}{.}\PY{n}{groupby}\PY{p}{(}\PY{l+s+s1}{\PYZsq{}}\PY{l+s+s1}{year}\PY{l+s+s1}{\PYZsq{}}\PY{p}{)}\PY{o}{.}\PY{n}{count}\PY{p}{(}\PY{p}{)}\PY{o}{.}\PY{n}{loc}\PY{p}{[}\PY{l+m+mi}{0}\PY{p}{:}\PY{p}{]}\PY{o}{.}\PY{n}{reset\PYZus{}index}\PY{p}{(}\PY{p}{)}\PY{o}{.}\PY{n}{to\PYZus{}numpy}\PY{p}{(}\PY{p}{)}
\end{Verbatim}
\end{tcolorbox}

    \begin{tcolorbox}[breakable, size=fbox, boxrule=1pt, pad at break*=1mm,colback=cellbackground, colframe=cellborder]
\prompt{In}{incolor}{73}{\boxspacing}
\begin{Verbatim}[commandchars=\\\{\}]
\PY{n}{X}
\end{Verbatim}
\end{tcolorbox}

            \begin{tcolorbox}[breakable, size=fbox, boxrule=.5pt, pad at break*=1mm, opacityfill=0]
\prompt{Out}{outcolor}{73}{\boxspacing}
\begin{Verbatim}[commandchars=\\\{\}]
array([[1874,    1],
       [1878,    1],
       [1887,    1],
       [1888,    2],
       [1890,    3],
       [1891,    4],
       [1892,    3],
       [1893,    1],
       [1894,    2],
       [1895,    2],
       [1896,    3],
       [1897,    1],
       [1898,    6],
       [1899,    1],
       [1900,    1],
       [1901,    1],
       [1902,    1],
       [1903,    3],
       [1904,    1],
       [1905,    1],
       [1908,    2],
       [1909,    3],
       [1910,    4],
       [1911,    2],
       [1912,    5],
       [1913,    7],
       [1914,   15],
       [1915,   24],
       [1916,   28],
       [1917,   14],
       [1918,   10],
       [1919,   20],
       [1920,   24],
       [1921,   29],
       [1922,   29],
       [1923,   20],
       [1924,   33],
       [1925,   35],
       [1926,   42],
       [1927,   37],
       [1928,   53],
       [1929,   58],
       [1930,   66],
       [1931,   80],
       [1932,  111],
       [1933,  113],
       [1934,  112],
       [1935,  123],
       [1936,  116],
       [1937,  128],
       [1938,  108],
       [1939,  110],
       [1940,  122],
       [1941,  123],
       [1942,  116],
       [1943,  127],
       [1944,  111],
       [1945,  120],
       [1946,   99],
       [1947,  117],
       [1948,  111],
       [1949,  142],
       [1950,  136],
       [1951,  143],
       [1952,  150],
       [1953,  158],
       [1954,  138],
       [1955,  166],
       [1956,  164],
       [1957,  194],
       [1958,  158],
       [1959,  170],
       [1960,  174],
       [1961,  157],
       [1962,  176],
       [1963,  176],
       [1964,  197],
       [1965,  188],
       [1966,  242],
       [1967,  214],
       [1968,  246],
       [1969,  214],
       [1970,  246],
       [1971,  266],
       [1972,  282],
       [1973,  257],
       [1974,  249],
       [1975,  259],
       [1976,  255],
       [1977,  255],
       [1978,  254],
       [1979,  263],
       [1980,  295],
       [1981,  291],
       [1982,  284],
       [1983,  274],
       [1984,  282],
       [1985,  306],
       [1986,  315],
       [1987,  368],
       [1988,  369],
       [1989,  364],
       [1990,  350],
       [1991,  354],
       [1992,  379],
       [1993,  416],
       [1994,  485],
       [1995,  523],
       [1996,  566],
       [1997,  593],
       [1998,  617],
       [1999,  620],
       [2000,  714],
       [2001,  716],
       [2002,  757],
       [2003,  722],
       [2004,  817],
       [2005,  889],
       [2006, 1033],
       [2007, 1067],
       [2008, 1193],
       [2009, 1352],
       [2010, 1231],
       [2011, 1321],
       [2012, 1387],
       [2013, 1476],
       [2014, 1420],
       [2015, 1036],
       [2016,   23]], dtype=int64)
\end{Verbatim}
\end{tcolorbox}
        
    \begin{tcolorbox}[breakable, size=fbox, boxrule=1pt, pad at break*=1mm,colback=cellbackground, colframe=cellborder]
\prompt{In}{incolor}{74}{\boxspacing}
\begin{Verbatim}[commandchars=\\\{\}]
\PY{c+c1}{\PYZsh{} \PYZsq{}X\PYZsq{} to ilość filmów do roku 2014}
\PY{c+c1}{\PYZsh{} dane ograniczone do roku 2014, by uniknąć wyników niewiarygodnych}

\PY{n}{X} \PY{o}{=} \PY{n}{dfm}\PY{p}{[}\PY{p}{[}\PY{l+s+s1}{\PYZsq{}}\PY{l+s+s1}{movieId}\PY{l+s+s1}{\PYZsq{}}\PY{p}{,} \PY{l+s+s1}{\PYZsq{}}\PY{l+s+s1}{year}\PY{l+s+s1}{\PYZsq{}}\PY{p}{]}\PY{p}{]}\PY{o}{.}\PY{n}{groupby}\PY{p}{(}\PY{l+s+s1}{\PYZsq{}}\PY{l+s+s1}{year}\PY{l+s+s1}{\PYZsq{}}\PY{p}{)}\PY{o}{.}\PY{n}{count}\PY{p}{(}\PY{p}{)}\PY{o}{.}\PY{n}{loc}\PY{p}{[}\PY{l+m+mi}{0}\PY{p}{:}\PY{l+m+mi}{2014}\PY{p}{]}\PY{o}{.}\PY{n}{reset\PYZus{}index}\PY{p}{(}\PY{p}{)}\PY{o}{.}\PY{n}{to\PYZus{}numpy}\PY{p}{(}\PY{p}{)}
\end{Verbatim}
\end{tcolorbox}

    \hypertarget{co-stanie-siux119-z-filmami-po-2030-roku}{%
\subparagraph{Co stanie się z filmami po 2030
roku?}\label{co-stanie-siux119-z-filmami-po-2030-roku}}

W ekstrapolacji stosujemy algorytm podobny do interpolacji w „Star Wars:
Episode 1''. Zmiana odpowiednich parametrów, wartości zmiennych.

Na wykresach przewidywane wyniki (linia) między znanymi węzłami danych
(kropki). Sprawdzę kolejno, czy optymalny wielomian będzie stopnia
2,3,4.

    \begin{tcolorbox}[breakable, size=fbox, boxrule=1pt, pad at break*=1mm,colback=cellbackground, colframe=cellborder]
\prompt{In}{incolor}{75}{\boxspacing}
\begin{Verbatim}[commandchars=\\\{\}]
\PY{c+c1}{\PYZsh{} \PYZsq{}t\PYZsq{} będzie znowu zmienną, dzięki której będziemy przewidywać}
\PY{c+c1}{\PYZsh{} np. co stanie się z filmami w 2030 r.}
\PY{c+c1}{\PYZsh{} kopiujemy kod, który już mamy (z 1 przykładu)}
\PY{c+c1}{\PYZsh{} zamieniamy tylko \PYZsq{}x\PYZsq{},\PYZsq{}y\PYZsq{} na \PYZsq{}X\PYZsq{}}
\PY{c+c1}{\PYZsh{} zerowy indeks to rok produkcji, a pierwszy indeks to ilość filmów}
\PY{c+c1}{\PYZsh{} \PYZsq{}Y\PYZsq{} to ilość filmów wyświetlanych w konkretnych latach}

\PY{c+c1}{\PYZsh{} następnie sprawdźmy kolejno, czy optymalny wielomian będzie stopnia 2,3,4}

\PY{n}{deg} \PY{o}{=} \PY{l+m+mi}{2}

\PY{n}{X} \PY{o}{=} \PY{n}{dfm}\PY{p}{[}\PY{p}{[}\PY{l+s+s1}{\PYZsq{}}\PY{l+s+s1}{movieId}\PY{l+s+s1}{\PYZsq{}}\PY{p}{,} \PY{l+s+s1}{\PYZsq{}}\PY{l+s+s1}{year}\PY{l+s+s1}{\PYZsq{}}\PY{p}{]}\PY{p}{]}\PY{o}{.}\PY{n}{groupby}\PY{p}{(}\PY{l+s+s1}{\PYZsq{}}\PY{l+s+s1}{year}\PY{l+s+s1}{\PYZsq{}}\PY{p}{)}\PY{o}{.}\PY{n}{count}\PY{p}{(}\PY{p}{)}\PY{o}{.}\PY{n}{loc}\PY{p}{[}\PY{l+m+mi}{0}\PY{p}{:}\PY{l+m+mi}{2014}\PY{p}{]}\PY{o}{.}\PY{n}{reset\PYZus{}index}\PY{p}{(}\PY{p}{)}\PY{o}{.}\PY{n}{to\PYZus{}numpy}\PY{p}{(}\PY{p}{)}

\PY{n}{t} \PY{o}{=} \PY{n}{np}\PY{o}{.}\PY{n}{linspace}\PY{p}{(}\PY{l+m+mi}{1900}\PY{p}{,} \PY{l+m+mi}{2030}\PY{p}{,} \PY{n}{num}\PY{o}{=}\PY{l+m+mi}{41}\PY{p}{)}

\PY{n}{coeffs} \PY{o}{=} \PY{n}{polyfit}\PY{p}{(}\PY{n}{X}\PY{p}{[}\PY{p}{:}\PY{p}{,} \PY{l+m+mi}{0}\PY{p}{]}\PY{p}{,} \PY{n}{X}\PY{p}{[}\PY{p}{:}\PY{p}{,} \PY{l+m+mi}{1}\PY{p}{]}\PY{p}{,} \PY{n}{deg}\PY{o}{=}\PY{n}{deg}\PY{p}{)}\PY{o}{.}\PY{n}{reshape}\PY{p}{(}\PY{o}{\PYZhy{}}\PY{l+m+mi}{1}\PY{p}{,} \PY{l+m+mi}{1}\PY{p}{)}
\PY{n}{X} \PY{o}{=} \PY{n}{np}\PY{o}{.}\PY{n}{concatenate}\PY{p}{(}\PY{n+nb}{list}\PY{p}{(}\PY{n+nb}{map}\PY{p}{(}\PY{k}{lambda} \PY{n}{i}\PY{p}{:} \PY{n}{t}\PY{o}{.}\PY{n}{reshape}\PY{p}{(}\PY{l+m+mi}{1}\PY{p}{,} \PY{o}{\PYZhy{}}\PY{l+m+mi}{1}\PY{p}{)} \PY{o}{*}\PY{o}{*} \PY{n}{i}\PY{p}{,} \PY{n+nb}{reversed}\PY{p}{(}\PY{n+nb}{range}\PY{p}{(}\PY{n}{deg} \PY{o}{+} \PY{l+m+mi}{1}\PY{p}{)}\PY{p}{)}\PY{p}{)}\PY{p}{)}\PY{p}{)}
\PY{n}{Y} \PY{o}{=} \PY{p}{(}\PY{n}{coeffs}\PY{o}{.}\PY{n}{T} \PY{o}{@} \PY{n}{X}\PY{p}{)}\PY{o}{.}\PY{n}{flatten}\PY{p}{(}\PY{p}{)}
\end{Verbatim}
\end{tcolorbox}

    \begin{tcolorbox}[breakable, size=fbox, boxrule=1pt, pad at break*=1mm,colback=cellbackground, colframe=cellborder]
\prompt{In}{incolor}{76}{\boxspacing}
\begin{Verbatim}[commandchars=\\\{\}]
\PY{c+c1}{\PYZsh{} dla wielomianu 2\PYZhy{}go stopnia nie mamy dobrego dopasowania (zbyt odstający)}

\PY{n}{fig}\PY{p}{,} \PY{n}{ax} \PY{o}{=} \PY{n}{plt}\PY{o}{.}\PY{n}{subplots}\PY{p}{(}\PY{l+m+mi}{1}\PY{p}{,} \PY{l+m+mi}{1}\PY{p}{,} \PY{n}{figsize}\PY{o}{=}\PY{p}{(}\PY{l+m+mi}{12}\PY{p}{,} \PY{l+m+mi}{3}\PY{p}{)}\PY{p}{)}
\PY{n}{ax}\PY{o}{.}\PY{n}{plot}\PY{p}{(}\PY{n}{dfm}\PY{p}{[}\PY{p}{[}\PY{l+s+s1}{\PYZsq{}}\PY{l+s+s1}{movieId}\PY{l+s+s1}{\PYZsq{}}\PY{p}{,} \PY{l+s+s1}{\PYZsq{}}\PY{l+s+s1}{year}\PY{l+s+s1}{\PYZsq{}}\PY{p}{]}\PY{p}{]}\PY{o}{.}\PY{n}{groupby}\PY{p}{(}\PY{l+s+s1}{\PYZsq{}}\PY{l+s+s1}{year}\PY{l+s+s1}{\PYZsq{}}\PY{p}{)}\PY{o}{.}\PY{n}{count}\PY{p}{(}\PY{p}{)}\PY{o}{.}\PY{n}{loc}\PY{p}{[}\PY{l+m+mi}{0}\PY{p}{:}\PY{l+m+mi}{2014}\PY{p}{]}\PY{p}{,} \PY{l+s+s1}{\PYZsq{}}\PY{l+s+s1}{o}\PY{l+s+s1}{\PYZsq{}}\PY{p}{)}
\PY{n}{plt}\PY{o}{.}\PY{n}{title}\PY{p}{(}\PY{l+s+s1}{\PYZsq{}}\PY{l+s+s1}{Ilość filmów do roku 2014}\PY{l+s+s1}{\PYZsq{}}\PY{p}{)}

\PY{c+c1}{\PYZsh{} dodajemy nasz wielomian (czyli \PYZsq{}t\PYZsq{} oraz \PYZsq{}Y\PYZsq{})}
\PY{c+c1}{\PYZsh{} tworzymy wykres oryginalnych danych (kropki)}
\PY{c+c1}{\PYZsh{} przewidywania czerwoną linią}

\PY{n}{fig}\PY{p}{,} \PY{n}{ax} \PY{o}{=} \PY{n}{plt}\PY{o}{.}\PY{n}{subplots}\PY{p}{(}\PY{l+m+mi}{1}\PY{p}{,} \PY{l+m+mi}{1}\PY{p}{,} \PY{n}{figsize}\PY{o}{=}\PY{p}{(}\PY{l+m+mi}{12}\PY{p}{,} \PY{l+m+mi}{3}\PY{p}{)}\PY{p}{)}
\PY{n}{ax}\PY{o}{.}\PY{n}{plot}\PY{p}{(}\PY{n}{dfm}\PY{p}{[}\PY{p}{[}\PY{l+s+s1}{\PYZsq{}}\PY{l+s+s1}{movieId}\PY{l+s+s1}{\PYZsq{}}\PY{p}{,} \PY{l+s+s1}{\PYZsq{}}\PY{l+s+s1}{year}\PY{l+s+s1}{\PYZsq{}}\PY{p}{]}\PY{p}{]}\PY{o}{.}\PY{n}{groupby}\PY{p}{(}\PY{l+s+s1}{\PYZsq{}}\PY{l+s+s1}{year}\PY{l+s+s1}{\PYZsq{}}\PY{p}{)}\PY{o}{.}\PY{n}{count}\PY{p}{(}\PY{p}{)}\PY{o}{.}\PY{n}{loc}\PY{p}{[}\PY{l+m+mi}{0}\PY{p}{:}\PY{l+m+mi}{2014}\PY{p}{]}\PY{p}{,} \PY{l+s+s1}{\PYZsq{}}\PY{l+s+s1}{o}\PY{l+s+s1}{\PYZsq{}}\PY{p}{)}
\PY{n}{ax}\PY{o}{.}\PY{n}{plot}\PY{p}{(}\PY{n}{t}\PY{p}{,} \PY{n}{Y}\PY{p}{,} \PY{l+s+s1}{\PYZsq{}}\PY{l+s+s1}{r}\PY{l+s+s1}{\PYZsq{}}\PY{p}{)}
\PY{n}{plt}\PY{o}{.}\PY{n}{title}\PY{p}{(}\PY{l+s+s1}{\PYZsq{}}\PY{l+s+s1}{Przewidywana ilość filmów do roku 2030          (wielomian 2 stopnia)}\PY{l+s+s1}{\PYZsq{}}\PY{p}{)}
\end{Verbatim}
\end{tcolorbox}

            \begin{tcolorbox}[breakable, size=fbox, boxrule=.5pt, pad at break*=1mm, opacityfill=0]
\prompt{Out}{outcolor}{76}{\boxspacing}
\begin{Verbatim}[commandchars=\\\{\}]
Text(0.5, 1.0, 'Przewidywana ilość filmów do roku 2030          (wielomian 2
stopnia)')
\end{Verbatim}
\end{tcolorbox}
        
    \begin{center}
    \adjustimage{max size={0.9\linewidth}{0.9\paperheight}}{output_92_1.png}
    \end{center}
    { \hspace*{\fill} \\}
    
    \begin{center}
    \adjustimage{max size={0.9\linewidth}{0.9\paperheight}}{output_92_2.png}
    \end{center}
    { \hspace*{\fill} \\}
    
    \begin{tcolorbox}[breakable, size=fbox, boxrule=1pt, pad at break*=1mm,colback=cellbackground, colframe=cellborder]
\prompt{In}{incolor}{77}{\boxspacing}
\begin{Verbatim}[commandchars=\\\{\}]
\PY{c+c1}{\PYZsh{} wielomian 3\PYZhy{}go stopnia wydaje się dobry}
\PY{c+c1}{\PYZsh{} mógłby symulować, w jaki sposób ilość fimów wzrastałaby w przyszłości}

\PY{n}{deg} \PY{o}{=} \PY{l+m+mi}{3}

\PY{n}{X} \PY{o}{=} \PY{n}{dfm}\PY{p}{[}\PY{p}{[}\PY{l+s+s1}{\PYZsq{}}\PY{l+s+s1}{movieId}\PY{l+s+s1}{\PYZsq{}}\PY{p}{,} \PY{l+s+s1}{\PYZsq{}}\PY{l+s+s1}{year}\PY{l+s+s1}{\PYZsq{}}\PY{p}{]}\PY{p}{]}\PY{o}{.}\PY{n}{groupby}\PY{p}{(}\PY{l+s+s1}{\PYZsq{}}\PY{l+s+s1}{year}\PY{l+s+s1}{\PYZsq{}}\PY{p}{)}\PY{o}{.}\PY{n}{count}\PY{p}{(}\PY{p}{)}\PY{o}{.}\PY{n}{loc}\PY{p}{[}\PY{l+m+mi}{0}\PY{p}{:}\PY{l+m+mi}{2014}\PY{p}{]}\PY{o}{.}\PY{n}{reset\PYZus{}index}\PY{p}{(}\PY{p}{)}\PY{o}{.}\PY{n}{to\PYZus{}numpy}\PY{p}{(}\PY{p}{)}

\PY{n}{coeffs} \PY{o}{=} \PY{n}{polyfit}\PY{p}{(}\PY{n}{X}\PY{p}{[}\PY{p}{:}\PY{p}{,} \PY{l+m+mi}{0}\PY{p}{]}\PY{p}{,} \PY{n}{X}\PY{p}{[}\PY{p}{:}\PY{p}{,} \PY{l+m+mi}{1}\PY{p}{]}\PY{p}{,} \PY{n}{deg}\PY{o}{=}\PY{n}{deg}\PY{p}{)}\PY{o}{.}\PY{n}{reshape}\PY{p}{(}\PY{o}{\PYZhy{}}\PY{l+m+mi}{1}\PY{p}{,} \PY{l+m+mi}{1}\PY{p}{)}
\PY{n}{X} \PY{o}{=} \PY{n}{np}\PY{o}{.}\PY{n}{concatenate}\PY{p}{(}\PY{n+nb}{list}\PY{p}{(}\PY{n+nb}{map}\PY{p}{(}\PY{k}{lambda} \PY{n}{i}\PY{p}{:} \PY{n}{t}\PY{o}{.}\PY{n}{reshape}\PY{p}{(}\PY{l+m+mi}{1}\PY{p}{,} \PY{o}{\PYZhy{}}\PY{l+m+mi}{1}\PY{p}{)} \PY{o}{*}\PY{o}{*} \PY{n}{i}\PY{p}{,} \PY{n+nb}{reversed}\PY{p}{(}\PY{n+nb}{range}\PY{p}{(}\PY{n}{deg} \PY{o}{+} \PY{l+m+mi}{1}\PY{p}{)}\PY{p}{)}\PY{p}{)}\PY{p}{)}\PY{p}{)}
\PY{n}{Y} \PY{o}{=} \PY{p}{(}\PY{n}{coeffs}\PY{o}{.}\PY{n}{T} \PY{o}{@} \PY{n}{X}\PY{p}{)}\PY{o}{.}\PY{n}{flatten}\PY{p}{(}\PY{p}{)}

\PY{c+c1}{\PYZsh{} dodajemy nasz wielomian (czyli \PYZsq{}t\PYZsq{} oraz \PYZsq{}Y\PYZsq{})}
\PY{c+c1}{\PYZsh{} przewidywania zaznaczone zieloną linią}

\PY{n}{fig}\PY{p}{,} \PY{n}{ax} \PY{o}{=} \PY{n}{plt}\PY{o}{.}\PY{n}{subplots}\PY{p}{(}\PY{l+m+mi}{1}\PY{p}{,} \PY{l+m+mi}{1}\PY{p}{,} \PY{n}{figsize}\PY{o}{=}\PY{p}{(}\PY{l+m+mi}{12}\PY{p}{,} \PY{l+m+mi}{3}\PY{p}{)}\PY{p}{)}
\PY{n}{ax}\PY{o}{.}\PY{n}{plot}\PY{p}{(}\PY{n}{dfm}\PY{p}{[}\PY{p}{[}\PY{l+s+s1}{\PYZsq{}}\PY{l+s+s1}{movieId}\PY{l+s+s1}{\PYZsq{}}\PY{p}{,} \PY{l+s+s1}{\PYZsq{}}\PY{l+s+s1}{year}\PY{l+s+s1}{\PYZsq{}}\PY{p}{]}\PY{p}{]}\PY{o}{.}\PY{n}{groupby}\PY{p}{(}\PY{l+s+s1}{\PYZsq{}}\PY{l+s+s1}{year}\PY{l+s+s1}{\PYZsq{}}\PY{p}{)}\PY{o}{.}\PY{n}{count}\PY{p}{(}\PY{p}{)}\PY{o}{.}\PY{n}{loc}\PY{p}{[}\PY{l+m+mi}{0}\PY{p}{:}\PY{l+m+mi}{2014}\PY{p}{]}\PY{p}{,} \PY{l+s+s1}{\PYZsq{}}\PY{l+s+s1}{o}\PY{l+s+s1}{\PYZsq{}}\PY{p}{)}
\PY{n}{ax}\PY{o}{.}\PY{n}{plot}\PY{p}{(}\PY{n}{t}\PY{p}{,} \PY{n}{Y}\PY{p}{,} \PY{l+s+s1}{\PYZsq{}}\PY{l+s+s1}{g}\PY{l+s+s1}{\PYZsq{}}\PY{p}{)}
\PY{c+c1}{\PYZsh{}ax.plot(t, Y, \PYZsq{}r\PYZsq{})}
\PY{n}{plt}\PY{o}{.}\PY{n}{title}\PY{p}{(}\PY{l+s+s1}{\PYZsq{}}\PY{l+s+s1}{Przewidywana ilość filmów do roku 2030          (wielomian 3 stopnia)}\PY{l+s+s1}{\PYZsq{}}\PY{p}{)}
\end{Verbatim}
\end{tcolorbox}

            \begin{tcolorbox}[breakable, size=fbox, boxrule=.5pt, pad at break*=1mm, opacityfill=0]
\prompt{Out}{outcolor}{77}{\boxspacing}
\begin{Verbatim}[commandchars=\\\{\}]
Text(0.5, 1.0, 'Przewidywana ilość filmów do roku 2030          (wielomian 3
stopnia)')
\end{Verbatim}
\end{tcolorbox}
        
    \begin{center}
    \adjustimage{max size={0.9\linewidth}{0.9\paperheight}}{output_93_1.png}
    \end{center}
    { \hspace*{\fill} \\}
    
    \begin{tcolorbox}[breakable, size=fbox, boxrule=1pt, pad at break*=1mm,colback=cellbackground, colframe=cellborder]
\prompt{In}{incolor}{78}{\boxspacing}
\begin{Verbatim}[commandchars=\\\{\}]
\PY{c+c1}{\PYZsh{} dla 4\PYZhy{}go stopnia dopasowanie staje się tak dobre, że mało realistyczne (model nie uczy się)}
\PY{c+c1}{\PYZsh{} przy stopniu wyższym od 3 nie dałoby się przewidzieć co stałoby się z głosami widzów później}

\PY{n}{deg} \PY{o}{=} \PY{l+m+mi}{4}

\PY{n}{X} \PY{o}{=} \PY{n}{dfm}\PY{p}{[}\PY{p}{[}\PY{l+s+s1}{\PYZsq{}}\PY{l+s+s1}{movieId}\PY{l+s+s1}{\PYZsq{}}\PY{p}{,} \PY{l+s+s1}{\PYZsq{}}\PY{l+s+s1}{year}\PY{l+s+s1}{\PYZsq{}}\PY{p}{]}\PY{p}{]}\PY{o}{.}\PY{n}{groupby}\PY{p}{(}\PY{l+s+s1}{\PYZsq{}}\PY{l+s+s1}{year}\PY{l+s+s1}{\PYZsq{}}\PY{p}{)}\PY{o}{.}\PY{n}{count}\PY{p}{(}\PY{p}{)}\PY{o}{.}\PY{n}{loc}\PY{p}{[}\PY{l+m+mi}{0}\PY{p}{:}\PY{l+m+mi}{2014}\PY{p}{]}\PY{o}{.}\PY{n}{reset\PYZus{}index}\PY{p}{(}\PY{p}{)}\PY{o}{.}\PY{n}{to\PYZus{}numpy}\PY{p}{(}\PY{p}{)}

\PY{n}{coeffs} \PY{o}{=} \PY{n}{polyfit}\PY{p}{(}\PY{n}{X}\PY{p}{[}\PY{p}{:}\PY{p}{,} \PY{l+m+mi}{0}\PY{p}{]}\PY{p}{,} \PY{n}{X}\PY{p}{[}\PY{p}{:}\PY{p}{,} \PY{l+m+mi}{1}\PY{p}{]}\PY{p}{,} \PY{n}{deg}\PY{o}{=}\PY{n}{deg}\PY{p}{)}\PY{o}{.}\PY{n}{reshape}\PY{p}{(}\PY{o}{\PYZhy{}}\PY{l+m+mi}{1}\PY{p}{,} \PY{l+m+mi}{1}\PY{p}{)}
\PY{n}{X} \PY{o}{=} \PY{n}{np}\PY{o}{.}\PY{n}{concatenate}\PY{p}{(}\PY{n+nb}{list}\PY{p}{(}\PY{n+nb}{map}\PY{p}{(}\PY{k}{lambda} \PY{n}{i}\PY{p}{:} \PY{n}{t}\PY{o}{.}\PY{n}{reshape}\PY{p}{(}\PY{l+m+mi}{1}\PY{p}{,} \PY{o}{\PYZhy{}}\PY{l+m+mi}{1}\PY{p}{)} \PY{o}{*}\PY{o}{*} \PY{n}{i}\PY{p}{,} \PY{n+nb}{reversed}\PY{p}{(}\PY{n+nb}{range}\PY{p}{(}\PY{n}{deg} \PY{o}{+} \PY{l+m+mi}{1}\PY{p}{)}\PY{p}{)}\PY{p}{)}\PY{p}{)}\PY{p}{)}
\PY{n}{Y} \PY{o}{=} \PY{p}{(}\PY{n}{coeffs}\PY{o}{.}\PY{n}{T} \PY{o}{@} \PY{n}{X}\PY{p}{)}\PY{o}{.}\PY{n}{flatten}\PY{p}{(}\PY{p}{)}

\PY{c+c1}{\PYZsh{} dodajemy nasz wielomian (czyli \PYZsq{}t\PYZsq{} oraz \PYZsq{}Y\PYZsq{})}
\PY{c+c1}{\PYZsh{} przewidywania zaznaczone czerwoną linią}

\PY{n}{fig}\PY{p}{,} \PY{n}{ax} \PY{o}{=} \PY{n}{plt}\PY{o}{.}\PY{n}{subplots}\PY{p}{(}\PY{l+m+mi}{1}\PY{p}{,} \PY{l+m+mi}{1}\PY{p}{,} \PY{n}{figsize}\PY{o}{=}\PY{p}{(}\PY{l+m+mi}{12}\PY{p}{,} \PY{l+m+mi}{3}\PY{p}{)}\PY{p}{)}
\PY{n}{ax}\PY{o}{.}\PY{n}{plot}\PY{p}{(}\PY{n}{dfm}\PY{p}{[}\PY{p}{[}\PY{l+s+s1}{\PYZsq{}}\PY{l+s+s1}{movieId}\PY{l+s+s1}{\PYZsq{}}\PY{p}{,} \PY{l+s+s1}{\PYZsq{}}\PY{l+s+s1}{year}\PY{l+s+s1}{\PYZsq{}}\PY{p}{]}\PY{p}{]}\PY{o}{.}\PY{n}{groupby}\PY{p}{(}\PY{l+s+s1}{\PYZsq{}}\PY{l+s+s1}{year}\PY{l+s+s1}{\PYZsq{}}\PY{p}{)}\PY{o}{.}\PY{n}{count}\PY{p}{(}\PY{p}{)}\PY{o}{.}\PY{n}{loc}\PY{p}{[}\PY{l+m+mi}{0}\PY{p}{:}\PY{l+m+mi}{2014}\PY{p}{]}\PY{p}{,} \PY{l+s+s1}{\PYZsq{}}\PY{l+s+s1}{o}\PY{l+s+s1}{\PYZsq{}}\PY{p}{)}
\PY{n}{ax}\PY{o}{.}\PY{n}{plot}\PY{p}{(}\PY{n}{t}\PY{p}{,} \PY{n}{Y}\PY{p}{,} \PY{l+s+s1}{\PYZsq{}}\PY{l+s+s1}{r}\PY{l+s+s1}{\PYZsq{}}\PY{p}{)}
\PY{n}{ax}\PY{o}{.}\PY{n}{plot}\PY{p}{(}\PY{n}{t}\PY{p}{,} \PY{n}{Y}\PY{p}{,} \PY{l+s+s1}{\PYZsq{}}\PY{l+s+s1}{r}\PY{l+s+s1}{\PYZsq{}}\PY{p}{)}
\PY{n}{plt}\PY{o}{.}\PY{n}{title}\PY{p}{(}\PY{l+s+s1}{\PYZsq{}}\PY{l+s+s1}{Przewidywana ilość filmów do roku 2030          (wielomian 4 stopnia)}\PY{l+s+s1}{\PYZsq{}}\PY{p}{)}
\end{Verbatim}
\end{tcolorbox}

            \begin{tcolorbox}[breakable, size=fbox, boxrule=.5pt, pad at break*=1mm, opacityfill=0]
\prompt{Out}{outcolor}{78}{\boxspacing}
\begin{Verbatim}[commandchars=\\\{\}]
Text(0.5, 1.0, 'Przewidywana ilość filmów do roku 2030          (wielomian 4
stopnia)')
\end{Verbatim}
\end{tcolorbox}
        
    \begin{center}
    \adjustimage{max size={0.9\linewidth}{0.9\paperheight}}{output_94_1.png}
    \end{center}
    { \hspace*{\fill} \\}
    
    \hypertarget{poruxf3wnanie-wynikuxf3w-na-podstawie-regresji-wielomianowej}{%
\paragraph{Porównanie wyników na podstawie regresji
wielomianowej}\label{poruxf3wnanie-wynikuxf3w-na-podstawie-regresji-wielomianowej}}

\begin{enumerate}
\def\labelenumi{\arabic{enumi}.}
\item
  Widzimy, że wielomian 3-go stopnia był dobrym dopasowaniem.
\item
  Gdy zmienimy na 2-go stopnia, to nie mamy dobrego dopasowania (zbyt
  odstający).
\item
  Dla 4-go stopnia dopasowanie staje się tak dobre, że mało
  realistyczne.
\item
  Przy stopniu wyższym od 3 nie dałoby się przewidzieć co stałoby się z
  głosami widzów później.
\end{enumerate}

\hypertarget{mieliux15bmy-2-przykux142ady-wykorzystania-regresji-wielomianowej}{%
\paragraph{Mieliśmy 2 przykłady wykorzystania regresji
wielomianowej}\label{mieliux15bmy-2-przykux142ady-wykorzystania-regresji-wielomianowej}}

Pierwszy - w interpolacji, drugi - w ekstrapolacji. 1. Pierwszy - przy
``Star Wars'' oceny można było wykorzystać do znajdowania wartości
pośrednich w obecnych czasach (interpolacja). 2. Drugi - przy ilości
wszystkich produkowanych filmów, do przewidywania szybkości wzrostu lub
spadku w przyszłości (ekstrapolacja).


    % Add a bibliography block to the postdoc
    
    
    
\end{document}
